\documentclass{report}

\usepackage{fontspec}
\usepackage{xltxtra}
\usepackage[french]{babel}
\setmainfont[Mapping=tex-text]{Gentium Book Basic}
\setsansfont{DejaVu Sans}

\usepackage[hidelinks]{hyperref}

\usepackage{xcolor}
\definecolor{light-gray}{gray}{0.8}

\usepackage{array}
\renewcommand{\arraystretch}{2}
\newcolumntype{M}[1]{>{\centering\arraybackslash}m{#1}}

\usepackage{pbox}

%See http://tex.stackexchange.com/questions/109467/footnote-in-tabular-environment#109471
\usepackage{footnote}
\makesavenoteenv{tabular}

\newcommand{\mytextfield}[1]{
    \TextField[bordercolor=,backgroundcolor=,width=#1]{}
}

\newcommand{\myfilledtextfield}[2]{
    \TextField[bordercolor=,backgroundcolor=,width=#1,value=#2,readonly]{}
}

\newcommand{\mycheckbox}{
    \mbox{\CheckBox[width=1.5em,bordercolor=]{}}
}

\newcommand{\mycheckedcheckbox}{
    \mbox{\CheckBox[width=1.5em,bordercolor=,checked,readonly]{}}
}

\newcommand{\mynumberfield}[1]{
    \TextField[bordercolor=,backgroundcolor=,align=1,width=#1]{}
}

\newcommand{\myfillednumberfield}[2]{
    \TextField[bordercolor=,backgroundcolor=,align=1,width=#1,value=#2,readonly]{}
}

\newcommand{\cross}{
    \textsf{☨}
}

\newcommand{\ankh}{
    \textsf{☥}
}

\newcommand{\caduceus}{
    \textsf{☤}
}



\usepackage{fullpage}

\setlength\parindent{0pt}
\setlength{\parskip}{1.3ex plus 0.5ex minus 0.3ex}

\titlespacing{\section}{0cm}{-0.125cm}{0.125cm}
\titleformat{\subsection}[display]{\normalfont\large\bfseries}{}{-10pt}{}

\usepackage{mdframed}

\hypersetup{
    linkcolor=blue,
    urlcolor=blue,
}

\newcommand{\gsection}[1]{
    \section{#1}
    \label{section-#1}
}

\newcommand{\glink}[1]{\hyperref[section-#1]{#1}}

\newcommand{\quest}[5]{
    \begin{mdframed}[innertopmargin=0.5cm,innerbottommargin=0.5cm,leftmargin=0.5cm,rightmargin=0.5cm]
        \begin{center}
            \textbf{#1} (#2)
        \end{center}
        \begin{description}
            \item[Type] #3
            \item[Difficulté (supposée)] #4
        \end{description}
        #5
    \end{mdframed}
}

\newcommand{\ellipse}{
    \begin{center}
        *********
    \end{center}
}

\newcommand{\success}{\emph{Succès !}}

\newcommand{\hero}[1]{\textbf{#1}}

\newcommand{\theend}{\emph{Fin d'aventure.}}

\begin{document}

\fullbleed{images/cover.jpg}

\begin{center}
{\fontsize{4cm}{0}\selectfont Y}
\end{center}

\chapter{Introduction}

Des conifères gigantesques masquant le soleil. Une enfilade de grandes collines et de petites montagnes. Des villages isolés, repliés sur eux-mêmes. Quelques forteresses éparpillées, indignes du terme château, pour les surveiller. Et bien sûr, cet épais brouillard opaque et froid qui s'installe dès le coucher du soleil et subsiste jusqu'au petit matin. Pas de doute, vous êtes en Ylèdre.

L'Ylèdre est une contrée où se déroulent mille légendes, mais aucune n'est réellement joyeuse, et bien peu se finissent bien. Des hommes qui doivent sucer le sang d'autres êtes humains pour survivre ou qui se transforment en loups affamés la nuit, des sorcières sacrifiant des bébés pour invoquer des démons, des spectres vengeurs qui errent dans les bois à la recherche de leurs assassins, voilà plutôt l'essence de ces histoires.

Ces contes sont bien sûr faux, des versions extrêmement romancées de faits divers. La vérité est bien plus frustre et violente.

Il y a des monstres en Ylèdre. Il y en a même beaucoup. Il y a des loups-garous. Il y a des vampires. Il y a des changelins. Il y a des bêtes si étranges et horribles qu'elles ne portent même pas de nom. Ils existent, ils sont présents, ils sont un élément du quotidien, combattus pied à pied par la population locale. Ici la lycanthropie, le vampirisme, ne sont pas des conséquences rarissimes d'une sorcellerie venue d'ailleurs. Ce sont des maladies comme les autres, éclatant régulièrement, avec leurs cas isolés, mais aussi leurs épidémies.

Et ces afflictions ne sont pas soignées avec des baumes et des cataplasmes, mais par l'acier et le feu. Les histoires réellement terrifiantes qui circulent ne sont pas celles sur les monstres, mais parlent plutôt de mères enfonçant elles-mêmes un pieux à travers le cœur de leur progéniture aux canines un peu trop développées, de supposés métamorphes et de prétendues sorcières exécutés sans sommation ni procès par une foule en colère, de nobliaux soupçonnés d'avoir pactisé avec des puissances obscures brûlés vifs dans l'incendie de leurs demeures.

Là aussi, l'exagération est de mise, mais il suffit de quelques minutes au voyageur de passage au contact d'une poignée d'ylériens pour s'apercevoir qu'ils composent un peuple aussi rude et dur que leur pays. Et les voyageurs sont bien moins rares qu'il n'y paraît au premier abord. Bien qu'officiellement rattaché au royaume voisin, l'Ylèdre est en pratique un territoire neutre, sur lequel personne n'a de réel contrôle, et est entouré de régions nettement plus riches. De nombreux contrebandiers et autres trafiquants en foulent régulièrement les chemins, y faisant transiter des marchandises de provenances douteuses. Certains ont même leur résidence principale au milieu des bois, là où ils se pensent à l'abri des autorités compétentes. D'où la deuxième catégorie de personnes errant sur les routes locales, et n'arrangeant guère les choses : les chasseurs de prime.

En-dehors de ces gens de mauvaise compagnie, l'Ylèdre est principalement habité par des familles rurales, qui y survivent depuis des générations, ainsi que par des réfugiés divers, principalement religieux, persuadés que les horreurs qui vivent dans les montagnes valent toujours mieux que les persécutions de leur lieu d'origine. Si certains changent d'avis assez rapidement, d'autant plus que l'accueil des locaux est rarement amical, ils arrivent que d'autres persistent, et finissent avec le temps à fusionner avec la masse xénophobe et paranoïaque des habitants du cru.

En résumé, l'Ylèdre est une terre corrompue, pervertie, désolée dans son âme sinon dans ses paysages.

Vous êtes là pour corriger cela.

Bien sûr, vous ne pensez pas changer le cœur des hommes du jour au lendemain. Mais un premier pas serait de régler le problèmes des monstres. Le pays n'est pas pauvre en lui-même, et s'il était possible d'en exploiter les ressources, notamment le bois et le minerai, sans que des créatures d'ombres jaillissent des profondeurs de la forêt pour vous dévorer, si les portes et les fenêtres n'avaient plus besoin d'être calfeutrés avant même que crépuscule ne soit achevé, si les vivants ne passaient plus leur temps à craindre les morts, alors peut-être y aurait-il un espoir.

Et pour régler ce problème particulier, vous comptez utiliser une nouvelle stratégie. Non pas combattre la conséquence mais la cause. Vos recherches démontrent clairement que le nombre de phénomènes sortant du cadre des lois naturelles est diaboliquement supérieur en Ylèdre par rapport au reste du monde, aussi bien en fréquence qu'en intensité. Vous êtes venu ici pour comprendre la raison de ce déséquilibre, et l'anéantir.

Mais alors que votre cœur s'enflamme devant cette résolution, une douleur vive traverse votre organisme, comme un puissant choc électrique qui vous tétanise. Cela ne dure qu'un instant, puis vous retrouvez le contrôle de votre corps avec juste une vague sensation de malaise. Le message est clair. Il vous est rappelé que vous devez suivre les règles du jeu. Vous n'avez pas le droit de vous impliquer trop personnellement.

Qu'à cela ne tienne. Vous serez bientôt en vue de l'auberge. Et de là, vous pourrez faire ce que vous savez faire de mieux : recruter de preux héros pour agir à votre place !

\chapter{Règles}

\section{Un protagoniste en retrait}

Dans cette aventure, vous incarnez le Vieux Sage, une mystérieuse personnalité qui erre à travers le monde pour combattre les forces du mal. Autrefois brillant et orgueilleux sorcier, bien que fondamentalement bon, vous avez écopé d'une terrible malédiction en vous frottant à trop forte partie. Dorénavant, non seulement vous ne pouvez plus utiliser vos pouvoirs, mais vous ne pouvez même pas agir directement pour tenter d'améliorer un peu les choses. Si la clé de la survie de l'humanité se trouvait à portée de main, vous ne pourriez tendre le bras sans faire une crise d'épilepsie.

Toutes vos tentatives pour rompre l'enchantement se sont révélées vaines. Toutefois, il existe un moyen de contourner ce blocage. Vous avez encore la possibilité de recruter des personnes et de leur confier des quêtes. Tant que vous restez très raisonnable dans les informations et les équipements que vous leur remettez, le maléfice reste endormi. Cela vous oblige souvent à pratiquer un subtil art de communication, à base d'énigmes, d'objectifs volontairement floutés, de cadeaux à l'intérêt peu clair, pour mettre toutes les chances du côté de votre recrue sans outrepasser les limites. Il vous arrive même de vendre des objets indispensables à des héros que vous envoyez sauver le monde.

Dans la pratique, vous commencez l'aventure dans une auberge, et vous n'en bougerez pas beaucoup, mais vous enverrez fréquemment des aventuriers accomplir des quêtes diverses aux quatre coins du pays.

\section{Phases principales}

L'aventure est divisée en une série de deux phases qui se répètent inlassablement : recrutement puis quête, recrutement puis quête, recrutement puis quête. Elles composent le cœur du jeu.

\subsection{Recrutement}

Régulièrement, vous pourrez choisir un aventurier parmi ceux disponibles dans le Livre des Héros ci-joint. C'est l'homme, ou la femme, ou l'être pas tout à fait humain, à qui vous désirez confier la prochaine quête à accomplir.

Mais, même si vous êtes particulièrement doué pour cela, trouver du personnel de qualité pour accomplir des tâches aussi diverses que variées, souvent dangereuses et complexes, n'est pas une sinécure. Si le monde ne manque pas d'aventuriers à la petite semaine, errants prêts à accepter de faire n'importe quoi pour une bouchée de pain, ils sont généralement d'une efficacité douteuse. Les meilleurs, les plus aptes à réussir, ont quant à eux presque toujours un caractère prononcé, et systématiquement des exigences salariales d'un autre ordre, en bien ou en mal.

Dans cette histoire, vous aurez donc la possibilité d'engager deux grandes catégories de protagonistes.

Tout d'abord, une flopée d'anonymes à la petite semaine. Ils sont en nombre illimité, et ne coûtent pas chers. Vous aurez simplement à dépenser 2 pièces d'or pour un envoyé médiocre, sans compétence particulière, ou 5 pièces d'or pour un champion disposant d'une Spécialité, au choix entre Combat, Diplomatie, Recherche et Roublardise. Leur degré de fiabilité étant inexistant, considérez que c'est chaque fois une nouvelle tête qui se présente à vous, avec notamment un équipement vierge.

Ensuite, sont également disponibles des aventuriers avec un (sur)nom, une histoire, une personnalité, des compétences propres. Outre le fait qu'ils ont parfois plusieurs Spécialités, chaque Quête dispose de sections réservées à certains héros seulement. Exploitant au maximum leurs talents particuliers, ces passages offrent souvent de bien meilleurs résultats qu'une voie victorieuse plus classique avec un anonyme correctement équipé. Également, dans le cas où un de ces énergumènes accompliraient plusieurs missions pour vous, il conserverait l'équipement et l'expérience acquise.

Malheureusement, le talent a un coût, et dans le cas de ces fortes têtes, il n'est pas toujours simplement en pièces d'or sonnantes et trébuchantes. Si vous désirez recruter l'un de ces personnages particuliers, vous devrez vous rendre à la section dédiée dans le Livre des Héros, où il ou elle vous posera ses conditions.

Si vous les acceptez, donnez-lui le paiement convenu si besoin est, puis passez à la phase suivante. Sinon, choisissez-en un autre ou résignez-vous à embaucher un anonyme.

Sauf indication contraire, un aventurier nommé peut être recruté autant de fois que désiré, mais ses tarifs deviennent souvent très vite prohibitifs.

De plus, au début, vous n'êtes en contact qu'avec six d'entre eux : Le \hero{Mnémonique}, l’Homme à la \hero{Main d’Acier}, le \hero{Beau Parleur}, le \hero{Brigand}, la \hero{Sorcière} et le \hero{Métamorphe}. Les autres devront être débloqués en cours d'aventure.

\subsection{Quête}

Une fois votre aventurier convaincu, vous devez l'assigner à une Quête disponible. Dans ce cas, rendez-vous simplement au paragraphe correspondant et suivez les instructions.

Avant toute mission, vous aurez la possibilité de donner un objet, et un seul, à l'aventurier choisi, pour l'aider dans sa quête. Une information cruciale (un mot de passe par exemple) compte pour un objet pour cette limite. Les aventuriers ayant tendance à s'accrocher à leurs possessions matérielles, tout objet physique que vous leur remettrez ne pourra être récupéré en aucune façon.

Exception 1 : Si l'aventurier exige un objet en guise de paiement, cela ne compte pas dans cette limite. Seules les aides volontaires que vous pouvez apporter sont limitées.

Exception 2 : Si un aventurier accomplit pour vous une seconde quête, ou plus, vous avez à nouveau la possibilité lui donner un objet pour l'aider dans sa tâche. En abusant des failles, vous pourriez ainsi vous retrouver avec un aventurier ayant quatre objets à la seconde mission : deux paiements et deux aides.

Sauf indication contraire, toute quête échouée peut être recommencée (tant que vous trouvez des aventuriers à envoyer au casse-pipe !) mais toute mission réussie devient définitivement indisponible. Une mission est considérée comme réussie si vous passez par un paragraphe comportant le terme \success ainsi calligraphié en l'accomplissant.

Une fois votre quête accomplie, ou ratée, vous devriez normalement être renvoyé sur le \textbf{50}, qui gère d'éventuels événements spéciaux se déroulant entre les phases ordinaires. Ensuite, choisissez un nouveau champion parmi ceux encore disponibles, et recommencez un tour complet.

\subsection{Et si ?}

Si d'aventure vous vous retrouviez dans une phase de recrutement où vous seriez incapable de recruter qui que ce soit, par exemple pour cause de finances désastreuses, rendez-vous immédiatement au \textbf{13}.

\section{Possessions, Informations et Codes}

\subsection{Possessions}

Dans cet aventure transiteront entre vos mains une certaine quantité d'objets plus ou moins utiles. Vous n'aurez généralement pas la possibilité de les utiliser vous-même, mais vous pourrez les confier à des aventuriers pour les aider dans leur mission.

Vous commencez l'aventure avec 20 pièces d'or et les objets suivants :
\begin{description}
\item[{Dague en argent [5]}:] Un poignard ordinaire si ce n'est que sa lame n'est pas à base de fer mais d'un précieux métal, fléau de bon nombre d'abominations.
\item[{Flûte traversière [1]}:] Bien que de qualité médiocre, cet instrument est capable de produire des sons tout à fait corrects entre les mains d'un joueur expérimenté.
\item[{Pendentif sacré [2]}:] Sur ce simple médaillon de cuivre est gravé dans une écriture archaïque le nom d'un antique divinité à moitié oubliée. Cet objet ne dispose pas de pouvoirs explicites, c'est avant tout un symbole.
\item[{Anneau de Lumière [10]}:] Cette bague enchantée rayonne dans le noir, éclairant le chemin bien plus  efficacement qu'une simple lanterne. Il n'est pas possible de l'activer ou le désactiver à volonté, mais un gant opaque suffit à dissimuler sa lumière.
\item[{Fétiche en os [5]}:] Un macabre collier à base d'os humains. Le pendentif en est un crâne taillé, recouvert d'inscriptions impies, la cordelette des phalanges attachées entre elles. Vous l'avez récupéré, suite à un long concours de circonstances, parmi les trophées d'un inquisiteur fanatique, mais vous en ignorez les effets, s'il en a.
\end{description}

Le nombre entre crochets est la valeur estimée de l'objet, en pièces d'or. Lors d'un paiement, au lieu de dépenser de l'or en espèces sonnantes et trébuchantes, vous pouvez offrir un objet de valeur égale ou supérieure.

Une valeur de [-] correspond à un objet sans valeur monétaire définie, qui ne peut être employé pour une équivalence de ce type.

\subsection{Informations}

Les Informations sont des données cruciales, qui comptent comme un objet quand vous les remettez à un aventurier. Elles ont cependant l'avantage par rapport aux Possessions d'être réutilisables à l'infini (vous ne perdez pas soudainement la mémoire après avoir révélé un secret).

\subsection{Codes}

Les Codes servent à enregistrer votre progression dans l'histoire. Ils correspondent à des événements qui se sont écoulés. Vous ne pouvez rien en faire, ce sont juste des mécanismes apparents causés par le format papier. Si cet ouvrage était un jeu vidéo, ils seraient traités en interne par le logiciel, sans jamais que vous ayez à les voir.

Certains Codes ont la double fonctionnalité d'indiquer qu'un événement a eu lieu et qu'il a été accompli par un personnage précis. Dans ce genre de cas, il vous sera demandé de l'associer au personnage l'ayant débloqué, c'est-à-dire d'indiquer à côté, entre parenthèses, le nom de celui qui a effectué la quête correspondante à son acquisition.

Si le Code en question a été acquis par un aventurier anonyme ou par le Vieux Sage lui-même, ne l'associez à personne, même si le texte vous le demande.

\section{Réputation}

Au début de cette aventure, vous n'êtes au mieux aux yeux de la population locale qu'un vieillard louche, potentiellement dangereux. Les habitants du cru seront donc peu enclins à vous apporter leur soutien, que ce soit en terme d'informations, de matériel ou simplement d'encouragements. Toutefois, au fur et à mesure que vos quêtes seront couronnées de succès, et par une habile propagande, vous pourrez petit à petit acquérir leur confiance, voire leur aide.
Cela est symbolisé par un score de Réputation, qui démarre à 0, et montera et descendra selon les conséquences de vos actes au cours de cette aventure. Sa valeur n'a pas de limite maximum, en positif comme en négatif.

Parfois, le texte vous demandera de calculer votre réputation auprès d'une fraction précise de la population, comme par exemple une organisation particulière ou une ethnie minoritaire. Dans ce cas, prenez votre score de Réputation et appliquez-lui les modificateurs indiqués dans le paragraphe en question pour estimer votre aura auprès ce groupe en particulier. Ces modificateurs ne s'appliquent que pour la durée de ce paragraphe, ne changez donc pas votre total général dans ce genre de cas.

\chapter{Aventure}

\gsection{1}

Ce n'est pas votre première visite dans cette région, mais jusqu'à présent, vous ne vous étiez occupé que des cas isolés. Jamais auparavant n'avez-vous tenté une opération de cette ampleur. Toutefois, vos précédents passages vous ont permis de nouer quelques contacts, et vous avez réussi à convaincre le patron d'une des auberges les plus passantes de vous héberger et de vous nourrir pour la durée de votre affaire, contre il est vrai, une somme si coquette qu'il reste bien peu d'or dans votre bourse.

Vous avez passé la première partie de votre quête à rassembler des renseignements, sans grand succès. En guise d'archives, vous n'avez trouvé que quelques registres paroissiaux, et les rumeurs et la tradition orale sont trop déformées pour dessiner une image claire dans votre esprit.

Votre premier objectif sera donc de rassembler des informations de meilleure qualité. Pour cela, vous avez découvert plusieurs sources qui vous paraissent intéressantes.

\quest{L'ordre des tueurs de monstres}{67}{Diplomatie ou Roublardise (\ankh, \cross)}{Facile}{
Ce groupuscule s'est donnée pour objectif depuis plusieurs générations d'exterminer tout ce qu'ils estiment contraire à l'ordre naturel, et ils emploient pour cela des méthodes d'une brutalité qui fait frémir même leurs contemporains ylèdrois.

Leur fanatisme, leur ascétisme, leur schéma de pensée unique, les rendent aussi peu désirables que ceux qu'il combattent. Mais ils conservent des archives détaillées de tous leurs affrontements, avec moult détails sur leurs ennemis, une mine d'informations que vous ne pouvez ignorer.

Cependant, convaincre des personnes qui ont amené la paranoïa au rang d'art de vivre de vous ouvrir leur porte sera difficile. Il faudra soit quelqu'un d'assez doué pour les amadouer soit quelqu'un d'assez peu scrupuleux pour contourner cette difficulté.
}

\clearpage

\quest{Le château du comte}{23}{Recherche (\ankh)}{Facile}{
Le meilleur endroit pour trouver des documents est encore la demeure de ceux qui contrôlaient jadis la région. Bien que toute la famille ait été éliminé par un chasseur de vampires et qu'une partie de la bâtisse ait brûlée, il devrait être encore possible de trouver des éléments intéressants en fouillant les décombres.

Les lieux, désertés par l'homme, sont devenus le logis de diverses vermines, mais, au moins en apparence, les créatures les plus dangereuses ne s'y sont pas installées, d'où une mission théoriquement facile.
}

\quest{Baba Yaga}{103}{Diplomatie}{Moyenne}{
Il est assez facile de trouver des sorcières en Ylèdre, mais elles ne sont pas forcément beaucoup plus au courant du vrai fonctionnement des choses que la moyenne des autres habitants. La plupart ne sont que des femmes plus ou moins vieilles qui connaissent deux ou trois sortilèges, voire simplement quelques filtres.

La Baba Yaga, c'est autre chose. C'est la sorcière parmi les sorcières. Son existence est avérée depuis des siècles. Elle a supposément été tuée un bon nombre de fois, brûlée, noyée, découpée en morceaux, mais cela ne l'a jamais empêché de réapparaître, parfois après une longue absence, parfois dès le lendemain. Si quelqu'un peut vous en apprendre plus sur cette région, c'est bien elle.

La convaincre de parler à votre émissaire plutôt que de le transformer en crapaud ne sera cependant pas une mince affaire, mais il s'agit d'un être avec lequel il est possible de négocier. Reste à trouver l'aventurier qui saura la charmer ou lui faire une offre qu'elle ne pourra refuser.
}

\quest{La traque}{3}{Combat (\caduceus)}{Moyenne}{
Comme trop souvent, un monstre terrorise la population. Laissant des traces qui évoquent un gigantesque loup, nul ne l'avait jamais vu et est revenu pour le décrire, mais il laisse derrière lui un sillage de victimes. Même si l'éliminer ne contribuera pas directement au grand œuvre que vous vous êtes fixé, personne, et surtout pas les habitants, ne vous en voudra si vous réglez ce problème.
}

\clearpage

\quest{Ceux qui vivent}{108}{Diplomatie (\cross)}{Facile}{
Votre mission n'implique pas que de fréquenter des monstres et des tueurs de monstres. Parfois, vous devez interagir avec des personnes beaucoup plus ordinaires, que ce soit pour leur demander des informations, désamorcer des tensions critiques, ou même leur soutirer quelques deniers nécessaires à la continuation de votre tâche. Comme vous n'êtes guère présentable et avenant vous-même, vous déléguez cependant ce travail nécessitant du doigté et du charme à des aventuriers plus habiles en la matière.

Spécial : Cette Quête est sans fin. Elle ne comporte pas de mention \emph{Succès !}, et peut être effectuée autant de fois que désiré, tant que vous avez assez de main d’œuvre à disposition. Un héros donné ne peut cependant l'accomplir qu'une seule et unique fois.
}

Vos objectifs à court terme maintenant définis au mieux des informations à votre disposition, ne vous reste plus qu'à désigner le premier champion qui sera chargé de s'y attaquer

S'agira-t-il du \hero{Mnémonique}, de l’Homme à la \hero{Main d’Acier}, du \hero{Beau Parleur}, du \hero{Brigand}, de la \hero{Sorcière}, du \hero{Métamorphe} ou même d'un des autres aventuriers moins hauts en couleur mais bon marché qui circulent ici ? À vous d'en décider.

\gsection{2}

La différence de puissance entre les adversaires est insurpassable. Un combat loyal se solderait forcément par la victoire de l'horreur enracinée. Dans ce genre de cas, il ne reste qu'une seule solution : tricher.

Une mélopée se fait entendre, rédigée par des êtres oubliés, mais s'échappant aujourd'hui de lèvres humaines. Un chant maudit, un appel au chaos, à la rupture des liens qui composent la réalité.

À ces mots, la sphère s'illumine, s'enflamme. Elle devient la clé. Elle devient la porte. Mille milliards de mondes existent en son sein. Mille milliards d'époques se déroulent en son cœur. Elle est l'univers et elle contient l'univers.

User d'un tel pouvoir comme un simple moyen de transport est une insulte, un blasphème, un sacrilège ! Mais c'est le maximum qu'un esprit humain puisse concevoir sans se détruire intégralement, même s'il doit déjà pour cela s'enfoncer au plus profond de la folie.

Ainsi, l'espace se contente-t-il de se fissurer, et dans les airs d'apparaître une déchirure, un passage vers un autre lieu, si loin que la lumière du soleil ne l'atteint qu'après de longues éternités. Le tourbillon avale la bête. Il avale le globe. Il épargne ce qui est à sa place en ce lieu, mais une paire d'yeux a le temps d'apercevoir ce qui se trouve au-delà, et un rire dément retentit, symbole d'un esprit à jamais brisé.

\ellipse

Vous aviez envoyé pour cette mission une personne que vous ne qualifierez pas d'ordinaire, ni même peut-être de tout à fait saine d'esprit à l'origine. Mais rien à voir avec la loque bavante, hurlante, que votre nouvel aventurier trouvera quelques jours plus tard, prostrée au milieu d'une large mare de boue rigoureusement vide. Elle avait accompli sa mission, mais cela lui avait tout coûté, sinon sa vie.

Notez le Code \emph{Yog-Sothoth}.

Rendez-vous au \glink{135}.

\gsection{3 – La traque}

En recoupant les lieux où se sont déroulés les différentes attaques, il est assez facile d'identifier une zone des bois où devrait logiquement vivre la créature. Sauf que plusieurs battues successives n'ont rien donné. Des rumeurs circulent donc sur un monstre fantôme, qui ne se matérialiserait qu'au plus noir de la nuit pour fondre sur les fous qui ne sont pas à l'abri de quatre murs solides à cette heure.

Mais pour l'heure, le soleil est au zénith, et une silhouette solitaire examine scrupuleusement le terrain à la recherche d'indices. Tâche malaisée dans l'obscurité de cette forêt épaisse, et après que de tant d'hommes et d'animaux l'aient foulé. Certains endroits démontrent toutefois du passage d'une créature imposante, suite de troncs brisés, des souches enfoncées, des buissons arrachés. Mais, et contre toute logique, ces voies royales s'interrompent rapidement, les dommages infligés à la nature cessant subitement devant un parterre de ronces, un ruisseau, quelques pierres, effectivement comme si la chose les ayant causés s'était volatilisée.

Les victimes n'ont été découvertes que plusieurs heures après leur mort, tard dans la nuit, sous la forme d'un macabre repas sans que nul ne soit témoin de l'attaque. Elles étaient toutes seules, isolées au moment des faits.

Si vous avez envoyé le \hero{Chasseur} pour cette mission, rendez-vous au \glink{71}.

Si vous avez préféré quelqu'un doté de la spécialité Roublardise, rendez-vous au \glink{107}.

Si c'est le Combat qui a eu votre préférence, rendez-vous au \glink{57}.

Dans tous les autres cas, rendez-vous au \glink{91}.

\gsection{4}

Notez le Code \emph{Purification}.

Dans l'absolu, vos alliés de l'Ordre considèrent cela comme une belle victoire, et ils vous débloquent, comme convenu, l'accès à leurs archives. Vous y découvrez un certain nombre d'éléments intéressants.

Débloquez les Quêtes suivantes :

\quest{Le marais interdit}{85}{Recherche (\ankh, \caduceus)}{Difficile}{
Les rapports démontrent une activité plus que louche aux abords du Marais Noir, une zone humide en friche où ne sont censés vivre que les moustiques. D'après les registres, ils enregistrent un nombre élevé de disparitions d'êtres humains, et une valeur toute aussi élevée d'apparitions de monstres. Assez étrangement, l'Ordre a très peu enquêté sur le sujet. Lorsque vous leur faites remarquer, ils se révèlent aussi perplexes que vous. Certains se rappellent que le sujet avait déjà été évoqué, que des mesures avaient même été prises, mais semblent incapables de se souvenir de la raison pour laquelle elles n'ont pas été appliquées. Mener votre propre enquête ne sera pas de trop.
}

\quest{Le cercle de la paix}{37}{Recherche (\caduceus)}{Moyenne}{
Le problème inverse du précédent : une zone de la carte où aucun événement sortant de l'ordinaire ne se produit. Pas d'attaque de créatures, pas de disparition, pas de magie noire, rien. Pourtant, l'endroit n'a en apparence rien de particulier, simple colline boisée comme les autres, un classique dans cette région. Un tel calme est suspect et mériterait une investigation plus poussée.
}

Augmentez également votre Réputation de 1.

Rendez-vous au \glink{50}.

\gsection{5}

\success

Depuis le recoin où il s'est dissimulé, le fieffé gredin observe l'ordre donner la chasse à l'intrus. Pour peu, il applaudirait. Sa malheureuse victime a attiré à elle nombre de soldats, et même si la plupart des gardes sont restés à leur poste avec un admirable professionnalisme, ils ne peuvent s'empêcher d'être distraits par tout ce vacarme.

Lorsque le vieillard a demandé au ruffian de s'infiltrer dans les locaux de cette organisation fanatique, il a failli refuser, tant la récompense monétaire était faible par rapport au risque. Mais il avait un vieux compte à régler avec ces champions auto-proclamés de la justice, et c'était l'occasion de joindre l'utile à l'agréable.

Après plusieurs jours passés à observer les lieux, se renseigner auprès des habitants, étudier les entrées et les sorties, il fut convaincu qu'il n'était pas possible de prendre cette place d'assaut sans ajouter une variable à l'équation.

Pour corriger cela, il recruta un baluchonneur sans ambition, officiellement pour faire le travail à sa place, officieusement pour lui offrir une splendide distraction.

Empruntant le chemin défriché par son malavisé partenaire toujours en fuite, il se faufile dans la bâtisse, progressant d'un pas sûr tout en esquivant habilement quelques pièges, alarmes et gardes encore actifs.

Car il est et reste un meilleur voleur que tous les voyous bas de gamme qui pullulent dans cette région.

\ellipse

Votre mercenaire pose devant vous une liste de notes griffonnées à la va-vite, contenant pêle-mêle des croquis de cartes, des passages recopiés en diverses langues, et ses propres remarques, avec des flèches et des cercles partout.

Vous avez ensuite une intéressante discussion avec lui. Il a remarqué plusieurs éléments d'intérêt.

Débloquez les Quêtes suivantes :

\quest{La secte du renouveau}{56}{Roublardise ou Combat (\ankh, \cross)}{Difficile}{
La secte du renouveau est un groupuscule religieux hérétique, officiellement éliminé à plusieurs reprises dans l'histoire, mais qui est toujours parvenu à se reconstituer avec le temps. Elle dispose d'une base très active en Ylèdre, et l'Ordre est récemment parvenu à découvrir la localisation de leur principale cache. Ils préparent en ce moment-même un assaut de grande envergure, et il vous faudra vous hâter si vous voulez récupérer des informations de l'autre côté de la barrière avant qu'ils n'en restent que des cendres purifiées.
}

\quest{Le marais interdit}{85}{Recherche (\ankh, \caduceus)}{Difficile}{
Les rapports démontrent une activité plus que louche aux abords du Marais Noir, une zone humide en friche où ne sont censés vivre que les moustiques. D'après les registres, ils enregistrent un nombre élevé de disparitions d'êtres humains, et une valeur toute aussi élevée d'apparitions de monstres. Assez étrangement, l'Ordre a très peu enquêté sur le sujet. Quelques soldats ont bien été envoyés, mais votre employé n'a trouvé aucune trace de leurs témoignages. Mener votre propre enquête ne serait pas de trop.
}

\quest{Le cercle de la paix}{37}{Recherche (\caduceus)}{Moyenne}{
L'exact inverse du problème précédent : une zone de la carte où aucun événement sortant de l'ordinaire ne se produit. Pas d'attaque de créatures, pas de disparition, pas de magie noire, rien. Pourtant, l'endroit n'a en apparence rien de particulier, simple colline boisée comme les autres, un classique dans cette région. Un tel calme est suspect et mériterait une investigation plus poussée.
}

Le cambrioleur a encore d'autres informations à vous transmettre. Il n' a accédé pour l'instant qu'aux rapports ordinaires de l'Ordre, mais a découvert l'existence d'une réserve, fortement gardée, où sont entreposés les manuscrits trop précieux pour être détruits et trop maléfiques pour être laissés en liberté. Une mine d'informations extraordinaire, à laquelle il pense pouvoir accéder, contre un tarif qu'il juge très raisonnable par rapport au danger de la mission : 10 pièces d'or !

Si vous avez de quoi le payer et désirez le faire, rendez-vous au \glink{109}.

Sinon, il prend congé. Rendez-vous au \glink{50}.

\gsection{6}

\success

Vous remerciez votre missionnaire pour son aide précieuse, puis vous vous plongez dans les documents qui vous ont été apportés. Il s'agit du récit de la déliquescence de la noble famille, du statut de maîtres relatifs de cette contrée à celui de monstres pourchassés par leurs propres sujets. Les textes sont de différentes plumes (le chapelain, le frère cadet, l'économe) mais sont cohérents entre eux. Apparemment, la situation a brusquement dégénéré après une visite du maître de maison au Marais Noir, un lieu particulièrement inhospitalier de l'Ylèdre. L'objectif de ce voyage est peu clair, mais à son retour, le comte n'était plus tout à fait humain.

Vous essayez de vous renseigner sur ces marais, mais vous n'en apprenez guère plus que ce que vous ne savez déjà : c'est un bourbier infâme, un piège collant où la boue peut facilement s'affaisser et avaler avec elle homme, bête ou monstre, sans discernement. Il va falloir mener l'enquête vous-même. Enfin, presque.

Débloquez la Quête suivante :

\quest{Le marais interdit}{85}{Recherche (\ankh, \caduceus)}{Difficile}{
Le marais, ou ce qu'il abrite, semble directement lié aux problèmes de prolifération des morts-vivants en l'Ylèdre. Mais impossible d'en savoir plus sans s'y rendre directement.
}

Rendez-vous au \glink{50}.

\gsection{7}

Vous connaissez maintenant tous les acteurs majeurs de l'Ylèdre, et avez réussi à gagner leur confiance indépendamment. Reste à les convaincre de travailler ensemble, qui plus est sur un projet pharaonique.

L'Ordre est le premier groupe à qui vous vous ouvrez. Après quelques tentatives malavisées d'éliminer eux-mêmes la créature des marais, ils se rangent à votre avis que, dans l'état actuel des choses, il n'est pas possible de faire mieux que de la sceller. Ils assureront la sécurité du chantier, une tâche nécessaire en cette contrée.

Les sorcières sont assez faciles à amadouer, ou plutôt à acheter. Il suffit d'y mettre le prix. Persuader les fées de faire équipe commune avec elles est une autre paire de manches. Mais le petit peuple est un peuple libre, et même si la plupart s'y refusent, d'autres se portent courageusement volontaires pour vous aider. En unissant tous leurs pouvoirs, ils produisent des charmes capables de protéger leurs porteurs de l'influence négative de la bête durant un certain temps. Les deux clans opposés de jeteurs de sort ne sont cependant pas encore assez ouverts pour s'exposer publiquement, et refusent tout contact direct avec le reste de l'équipe, préférant passer par des intermédiaires sûrs.

La force vive des travailleurs est assurée par les habitants ordinaires, auxquels se sont mêlés un grand nombre de membres du Peuple des Ombres. Pour les premiers, l'or, fourni par l'Ordre (et indirectement par la secte du renouveau), est une incitation suffisante, pour les seconds, c'est l'espoir d'une tranquillité retrouvée qui les fait agir.

Et la prison prend forme. À chaque brique posée, le monde semble un peu moins terne, un peu moins terrible. Et bientôt l'engeance impie est enfermée, ses ondes malfaisantes contraintes, son influence corruptrice affaiblie.

Le sceau n'est pas parfait, en partie à cause du terrain difficile, en partie car le maître du chantier n'a qu'une connaissance de seconde main sur sa conception. Mais maintenant, l'existence de la créature est connue de tous, ce qui diminue déjà de beaucoup son aura pernicieuse.

Toutefois, cela reste un emprisonnement, pas une solution pérenne. Peut-être qu'ainsi isolée, la bête finira par mourir de faim, mais il est plus probable qu'elle survive aux années, attendant son heure.

Cependant, en cet instant, non seulement, la racine du mal en Ylèdre est enfermée, mais vous avez qui plus est réussi à faire travailler main dans la main les différentes factions d'un pays déchiré.

Notez le Code \emph{Union} et augmentez votre Réputation de 5.

Si vous avez décidé de maintenir prisonnier le messie non-mort, rendez-vous au \glink{63}. Sinon, rendez-vous au \glink{76}.

\gsection{8}

Les pièces du puzzle se mettent en place. Un être ou une chose d'une grand puissance n'a de cesse de ranimer les morts, même ceux qui ont été éradiqués à plusieurs reprises. La secte a des racines anciennes, profondes et vivaces en ces lieux, symbolisées par leurs temples décrépis mais toujours en activité, et malgré les difficultés et les fréquents affrontements, n'en ait jamais réellement parti.

Tant d'acharnement vous étonne, aussi parcourez-vous de nouveau toutes les informations dont vous disposez à leur sujet. Et vous trouvez enfin ce que vous cherchiez depuis le début dans leurs textes sacrés. Alors que tous les autres psaumes sont dans un style archaïque, démodé, comprenant des mots ayant totalement disparus du vocabulaire, un passage est résolument plus moderne, utilisant des tournures de phrases passées mais pas antiques. Un chant à la gloire de leur messie, le non-mort originel :

\textit{Il est le premier et le plus parfait des ressuscités. Il a tout les atouts de la première vie, mais aucune de ses tares. Il marche parmi nous, et sous ses pas le chaos de la nature est remplacé par l'ordre divin.} [...] \textit{Les mécréants, craignant la vérité qu'il apportait, le jugèrent, le condamnèrent et sous la terre ils l'enfermèrent. Mais ne peut mourir ce qui est déjà mort, et dans sa prison de roches et de pierres, l'Élu attend son heure.}

Vous débloquez la Quête :

\quest{L'avatar}{95}{Combat (\ankh)}{Insensée}{
Les adeptes ne vivent peut-être pas sous terre simplement pour échapper aux persécutions. Peut-être cherchent-ils quelqu'un, quelqu'un dont la simple présence est suffisante pour modifier l'équilibre de la nature. Avec cette hypothèse en tête, un second examen de leurs caches risque d'apporter de troublantes révélations.
}

Notez \emph{Eurêka}.

Retournez au \glink{50}.

\gsection{9}

Vous faites un rêve étrange. Une créature du fond des âges, une monstruosité cyclopéenne qui ne peut être réellement décrite qu'à l'aide de termes barbares issues de langues oubliées, cherche à s'emparer de vous, à dévorer votre esprit comme votre âme. Mais une petite femme pitoyable, un épouvantail avec juste la peau sur les os, s'interpose et l'arrête, sans effort apparent.

Au réveil, vous avez déjà presque tout oublié, et les derniers restes du songe ne survivront guère longtemps à la lueur de l'aube.

Notez le Code \emph{Mémoire}.

Rendez-vous au \glink{93}.

\gsection{10}

C'est déjà un petit miracle que votre monte-en-l'air ait réussi à se sortir de la forteresse de l'Ordre en un seul morceau. Certes, après avoir écopé de nombreuses blessures légères, morsures, lacérations et hématomes, ses vêtements déchiquetés et son équipement éparpillé lors de sa fuite, mais encore capable de se déplacer sur ses deux jambes.

Une chance pour vous aussi, car l'Ordre n'a ainsi pas pu remonter jusqu'au commanditaire de l'opération. Cependant, vous avez l'impression qu'ils s'en doutent.

Effacez tout l'équipement de la personne que vous aviez envoyée. Diminuez également votre Réputation de 1.

Rendez-vous au \glink{50}.

\gsection{11}

L'histoire qui s'échappe, se découvre, parle d'un autre temps, d'un autre âge. Une époque où la corruption ne s'était pas encore installée en ces terres, où la nature était plus clémente, où une ville florissante s'étendait en lieu et place de ce marais putride. Une cité riche, cultivée, ouverte sur le monde et en abritant une des merveilles : sept obélisques, de sept roches différentes, gravés de millions de signes, contenant chacun un septième de toute la connaissance du monde. Sur celui de granite, la somme des sciences appliquées, en agriculture, en construction, en métallurgie. Sur celui de marbre, les arts écrits, la littérature, la poésie, le théâtre. Et sur celui d'obsidienne, les secrets des arts obscurs, de la sorcellerie, de ce qui est au-delà du voile.

Et à la lumière de ces nouvelles informations, la vision des lieux change. Des plate-formes rocheuses au milieu de la boue se révèlent être non des plaques naturelles, mais des sommets de bâtiments enfouis. Une métropole enterrée se dévoile sous la boue, l'eau et la végétation.

Et alors un fol espoir naît dans le cœur de l'érudit. Il se presse vers ce qui a été autrefois la grande place, maintenant une vaste étendue gluante, ni solide, ni liquide. Mais au milieu de celle-ci, dépasse encore une pointe d'obsidienne. Et l'homme, tel un possédé, se rue sur elle, et commence frénétiquement à creuser, à mains nues.

Les premiers symboles apparaissent. Tant de connaissances, tant de savoir. Un influx continu de nouvelles réponses mais aussi de nouvelles interrogations envahit l'esprit du chercheur alors qu'il ne cesse de creuser, de s'enfoncer toujours plus loin dans la vase, ignorant de l'horloge qui avance, des tremblements de froid dans son organisme trempé, ou de l'étau de plus en plus en fort de terre et d'eau mélangées. Il y a tant à découvrir, il n'a pas le temps de se soucier de ces détails. Ainsi continue-t-il à s'enfoncer, vers l'illumination.

Rendez-vous au \glink{33}.

\gsection{12}

Après avoir parcouru attentivement plusieurs fois la crypte, votre spécialiste trouve enfin ce que son esprit aiguisé s'attendait à trouver : l'indispensable passage secret que tout noble mal-aimé se doit de posséder, particulièrement s'il a outrepassé la durée de vie que la nature lui a accordé. Passage étroit, assez court, se terminant en cul-de-sac par une petite pièce carrée, ne comportant pour tout meuble qu'un sarcophage ouvert. Sur le sol repose un autre pieu, cette fois à pointe d'argent et tâché de sang.

Dans le tombeau se trouve un cercueil fracassé, et dans celui-ci des morceaux d'homme : autour d'un squelette encore intact, des organes, des muscles, des morceaux de chair, ont partiellement repoussé, reconstituant intégralement un visage aux canines proéminentes, un bras droit, un morceau de hanche, un bout de bassin, mais laissant encore à nu, bien visible entre les côtes, un cœur ne battant pas. L'ensemble évoque un écorché de biologie, en beaucoup plus malsain.

Comme s'il sentait qu'il était observé, le cadavre se met soudain à bouger, et avec une promptitude que sa décrépitude ne laissait pas suggérer, se jette sur la personne ayant dérangé son repos.

Si c'est le \hero{Mage} qui est présent, rendez-vous au \glink{53}.

Si ce n'est pas le cas, mais que vous avez choisi quelqu'un disposant d'une Spécialité en Combat, d'une arme en argent ou d'un symbole sacré, rendez-vous au \glink{65}.

Sinon, rendez-vous au \glink{31}.

\gsection{13}

Vous êtes seul dans la taverne, attendant une aide qui ne viendra pas. Votre quête vous paraît progresser, mais lentement, péniblement, laissant une pile de morts et de disparus derrière elle. Vos finances sont à sec et votre réputation à terre. Vous êtes devenu un oiseau de sinistre augure que les gens biens évitent. Plus personne ne veut faire affaire avec vous.

Bon, il ne sert à rien d'insister. Vous remontez dans votre chambre, faites vos bagages et repartez. D'ici quelques années, quand les souvenirs de vos échecs se seront tassés, vous reviendrez et recommencerez. Mais d'ici là, vous allez devoir vous racheter une réputation. Il paraît qu'ils ont une invasion de dragons dans le nord. Un problème facile à résoudre, idéal pour redorer votre blason à peu de frais. Alors en route !

\theend

\gsection{14}

L'orbe, ou quel que soit le nom qui lui est donné, est un objet de pouvoir, capable par sa seule présence de briser les barrières. Il suffit juste d'un petit coup de pouce. Cela peut être une infâme malédiction comme une fervente prière. L'important n'est pas tant le contenu que l'énergie, l'âme, la volonté qui se cache derrière.

Par les pouvoirs libérés apparaissent à travers tout le marais d'innombrables spectres. Toutes les victimes du monstre, depuis la nuit du temps jusqu'à ce jour, se dessinent, silhouettes diaphanes, sans consistance. S'y trouvent une antique civilisation que sa simple naissance a exterminée, un peuple nomade assez fou pour lui avoir rendu un culte et qu'il a dévorés jusqu'au dernier, un kaléidoscope d'ethnies, d'espèces, dont il a attisé les haines et les tensions pour se nourrir de leur rage et de leur désespoir.

Le marais est couvert d'ombres blanches, intangibles mais froides comme la mort. L'eau gèle dans le bourbier tandis que les rangs des morts s'épaississent toujours plus. Il y a des générations et des générations, toutes unies par le même meurtrier, toutes revenues pour la même vengeance.

Et soudain tout cela déferle, de tous les côtés, en un flot continu, fondant en masse sur la créature. Ses pouvoirs lui permettent de se défendre quelque peu, déchirant les âmes comme si elles étaient faites de papier, mais il y en a trop, beaucoup trop. Ses protections sont submergées, dépassées tandis que les fantômes s'enfoncent un par un dans les couches de son esprit, en emportant chacun un minuscule fragment avant de disparaître.

Le combat se faisant uniquement sur le plan du mental, la bête paraît parfaitement intact lorsqu'il se termine. Mais l'équivalent chez elle du cerveau a été complètement nettoyé, effacé, anéanti. Même si son cœur bat encore dans un mouvement réflexe, la créature est belle et bien trépassée.

\ellipse

Il y a quelque chose de magique, de féerique, dans ce bourbier purulent complètement paralysé par le froid au petit matin. Il ne dégage plus d'odeur, le bruit des insectes s'est tu, la lumière s'y réfléchit un peu partout en des milliers d'étoiles. Même l’horrible créature dans sa gangue de glace paraît moins terrifiante, comme une simple sculpture d'un artiste fou.

Mais pour l'heure, la personne que vous avez envoyée aux nouvelles s'efforce de sauver d'une grave hypothermie son prédécesseur, car le souffle glacial du royaume des morts n'a pas fait de distinction.

Rendez-vous au \glink{135}.

\gsection{15}

Vous vous réveillez pâteux et épuisé après une très mauvaise nuit, encore une fois. À nouveau, vous avez l'impression que quelque chose a pillé votre esprit. Mais cette fois, vous étiez préparé. Vous passez l'essentiel de la matinée à relire vos notes, à la recherche des éléments que vous pourriez avoir oublié. Et effectivement, vous en trouvez. Votre système, bien que primitif, semble avoir fonctionné.

Notez le Code \emph{Mémoire}.

Rendez-vous au \glink{93}.

\gsection{16}

C'est un combat de professionnels qui s'engage. Bottes, esquives, parades, les deux camps maîtrisent leur art, rendent coup pour coup. Toutefois, plus le temps passe, plus l'avantage du monstre se creuse. Il devient de plus en plus en rapide, de plus en plus précis, de plus en plus machine et de moins en moins humain, tandis que son adversaire transpire, ralentit, fatigue.

Bientôt, ce petit jeu s'achève, avec brutalité, par une décapitation en plein mouvement. Le sang du vaincu asperge le corps d'acier du vainqueur alors que sa tête roule piteusement sur le sol.

Rendez-vous au \glink{33}.

\gsection{17}

L'homme s'avance sans crainte dans les ténèbres. Il ne peut imaginer que s'y trouvent pires horreurs que celles qu'il a déjà aperçues. La perceptive d'un accident, d'une chute, ne l'effraie pas non plus. Au contraire, c'est avec soulagement qu'il accueillerait une mort aussi banale, une libération rapide des tourments de ce qu'il ne peut oublier.

Bientôt, son environnement change. Des ténèbres inconnues il passe à un aperçu du paradis, qui le laisse tout autant de marbre. Cela ne peut être qu'une illusion. Il n'y a pas de paradis. Il n'y a pas de dieux bienveillants, pas de refuges possibles. Il n'y a qu'eux, qui attendent leur heure.

Et, protégé de la folie extérieure par la sienne propre, il continue son chemin, insouciant.

Rendez-vous au \emph{44}.

\gsection{18}

\success

L'infiltration a été un succès. Votre fouine vient de revenir, et vous ramène un grand nombre d'informations que vous vous hâtez de transmettre à l'Ordre : la disposition des lieux, une estimation du nombre de personnes présentes, les principaux chemins d'accès et de sorties... Seuls quelques recoins trop surveillés sont encore inconnus, mais les précieuses données sont estimées suffisantes pour préparer l'offensive.

Vous ne vous en mêlez pas, mais en obtenez les échos. Une centaine de frères de l'Ordre se sont engouffrés dans les mines, prenant par surprise des adorateurs en nombre similaire. Ce fut cependant un combat de longue haleine, une guérilla dans le dédale des galeries, et les attaquants perdirent nombre d'hommes. Plusieurs meneurs réussirent aussi à s'enfuir, mais la plupart furent capturés ou éliminés. L'idole hérétique fut jetée à bas et brisée en morceaux, et les trésors du temple pillés, pour la cause.

Pendant ce temps, vous lisez les documents qu'a réussi à se procurer au passage votre spécialiste en infiltration : une copie des principaux textes sacrés de la secte, une série de psaumes rassemblés d'une écriture serrée sur des parchemins de récupération. Probablement compilée et oubliée par quelque novice, elle forme une bonne introduction à leur religion.

Sans surprise, son credo est la résurrection des morts, bien qu'elle soit bien plus libertaire que d'autres sur le sujet. Ainsi, les morts-vivants supérieurs, c'est-à-dire ceux dotés de conscience comme les vampires ou les liches, sont considérés comme des ressuscités à part entière, des êtres bénis à qui le seigneur a offert le don de seconde vie. L'idéal absolu des adeptes semble d'ailleurs être la seconde vie éternelle, débarrassée une fois pour toutes du risque de la mort.

Augmentez votre Réputation de 2 et notez le Code \emph{Religion}.

Rendez-vous au \glink{4}.

\gsection{19}

Votre envoyé disparaît pendant plusieurs jours, période où vous êtes sans nouvelle directe de lui mais où d'étrangers rumeurs vous parviennent. Un riche marchand aurait perdu toute sa fortune dans un pari absurde avec un inconnu. Une magnifique blonde à la chevelure luxuriante se serait réveillé le crâne rasé et surmonté d'une simulacre de perruque en cordelettes. Un notable aurait été découvert déguisé en âne dans ses écuries et se refuserait obstinément à expliquer comment il en est arrivé là.

Et puis un beau jour, l'être changeant revient. Il s'excuse de son retard, arguant de la nécessité de récupérer quelques ingrédients, et vous apporte plusieurs présents pour compenser.

Le premier est le soutien moral de Baba Yaga et ses fidèles. Si cela n'est guère concret, il vous assure que c'est un élément qui pourra vous sauver la vie plus tard.

Le second est un livre, un épais ouvrage ayant des siècles derrière lui. L'âge l'a décati, et il tient maintenant plus du paquet de feuillets emprisonnés entre deux tranches de cuir racorni que du vrai livre relié. Mais plus que l'ancienneté, il exhale le mal, le pouvoir, la connaissance interdite. Ce n'est pas un manuel à l'usage des débutants, plutôt l'encyclopédie de la sorcellerie, le livre original à partir duquel ont été recopiés les sortilèges les plus dangereux, les plus puissants.

Le troisième est un caillou grossier serti dans un médaillon d'acier. Il dispose d'un enchantement mineur qui lui permet de s'enflammer dès que la situation l'exige, produisant lumière et chaleur. Toutefois, la pierre se consume peu à peu à chaque utilisation, s'épuisant lentement et sûrement. Ce cadeau fait piètre figure par rapport au précédent, mais le métamorphe, citant la sorcière, déclare qu'avant de s'aventurer dans l'ombre, il faut s'assurer d'avoir une torche fiable.

Ajoutez le Tome Scellé [-] et la Pierre de Feu [8] à vos Possessions. Gagnez ensuite le Code \emph{Sabbat}, et associez-le au \hero{Métamorphe}.

Enfin, diminuez votre Réputation de 3. En effet, certaines personnes semblent croire que vous êtes à l'origine de la vague de chaos qui s'est récemment abattu sur le pays, et votre instinct vous dit qu'ils n'ont pas forcément tort.

Rendez-vous au \glink{50}.

\gsection{20}

\success

« Le petit peuple a entendu votre appel et accepte de vous aider. »

À ces mots, le cercle de chevaliers se transforme en une haie d'honneur, le long de laquelle s'avance la maîtresse des lieux. Vêtue d'un antique toge, un diadème d'or posé sur son ample chevelure blonde, la lumière semble passer à travers elle comme à travers un prisme, lui donnant un aspect aussi resplendissant que fragile, éphémère.

Une reine. Nul besoin de connaître le protocole pour savoir qu'il est nécessaire de mettre genou à terre et d'attendre qu'elle parle à nouveau, sans chercher à la brusquer.

« Le petit peuple n'a pas choisi de s'isoler en ces terres obscures de son plein gré. Il y est venu voilà bien longtemps, à l'époque où cet endroit foisonnait de vie et de magie. Puis la corruption est arrivée et a perverti hommes, bêtes et même certains d'entre nous. Pour nous protéger, nous avons été contraints de nous enfermer nous-mêmes derrière de nombreuses barrières, des murailles de charmes et de sortilèges. Nous ne les briserons pas pour quelques belles paroles, pas plus que je n'exposerai la vie et l'âme du moindre de mes sujets pour un si faible prix. Néanmoins, votre quête est noble, aussi avons-nous décidé de vous offrir une aide matérielle. »

Un présent enchanté est alors amené sur un plateau d'argent, porté à bout de bras par deux minuscules fées voletant dans les airs. Il est accepté avec de multiples courbettes.

« Si vous souhaitez réellement guérir cette contrée, continue la reine, vous devrez vous rendre là où tout a commencé. En sortant d'ici, suivez la direction du soleil couchant. Vous y trouverez une haute montagne solitaire, une jeune pousse de roches et de colère. Escaladez-la, puis plongez en son cœur. Là vous trouverez des réponses... Ou la folie. »

Et sur ces mots, l'entretien est terminé. Dans un tourbillon de feuilles, arène et spectateurs cessent d'exister. Ne reste qu'une silhouette humaine au milieu de la clairière déserte, ses possessions matérielles étalées à ses pieds, parmi laquelle son nouveau trésor.

Notez le Code \emph{Féerie} et associez-le au héros ou à l'héroïne qui vient de l'obtenir.

\ellipse

« Voilà toute l'histoire. »

Vous caressez votre barbe en réfléchissant. Ainsi, le peuple des lutins, des gnomes et des fées vit encore en ces lieux. Qu'ils aient en plus accepté de vous aider, même sommairement, est déjà miraculeux en soi. Vous contemplez d'ailleurs leur présent, qui vous a été transmis sans rechigner bien longtemps. Rendez-vous au \glink{61} pour savoir de quoi il s'agit.

Ils vous ont également donné une précieuse nouvelle piste. Vous débloquez la Quête suivante :

\quest{Les cavernes du début du monde}{51}{Recherche}{Difficile}{
Vous pensez avoir identifié la montagne que la royale fée a évoqué. Si vous avez bien compris ses paroles, au sommet de celle-ci devrait se trouver une grotte contenant des mystères oubliés. Vous ignorez quoi exactement, mais tous vos instincts vous préviennent que vous n'aimerez sans doute pas ce qui s'y trouve.
}

De plus, augmentez votre Réputation de 2.

Le \hero{Croisé} est dorénavant disponible.

Rendez-vous au \glink{50}.

\gsection{21}

L'Ylèdre est une région pleine de secrets. Certains ne sont connus que d'une seule personne, d'autres n'ont de secret que le nom. Parmi tous ces mystères se trouve une histoire qui ne se transmet que de sorcière à sorcière. Il existe au fond d'un certain bosquet un arbre à la silhouette torturée qui fut autrefois une voyante exceptionnellement douée. Son talent était indéniable, mais son honnêteté malheureuse. Ses prédictions fâchèrent quelqu'un qu'il ne fallait pas provoquer. Ses pieds devinrent alors racines, ses mains branches, son corps tronc. Son esprit lui resta intact, emprisonné dans le bois. Sans yeux pour voir, sans oreilles pour entendre, sa vie devint un enfer, éternellement plongée dans l'obscurité et le silence.

Toutefois, ses dons de prescience persistent, et pour qui sait communiquer avec les plantes, il est possible de lui soutirer quelques informations.

Aujourd'hui, une vieille sorcière s'efforce de l'interroger à propos des lubies d'un autre vieillard. Une éternité de souffrance, entrecoupée de quémandes, n'a cependant pas rendu pas la pythie très coopérative, et si la magie la condamne à répondre, elle ne l'oblige pas à être clair :

« La mémoire est la vie. L'oubli est la mort. Les mots écrits survivent quand les pensées s'effacent. Dans le grand livre se trouve la clé des souvenirs perdus dans l'obscurité. »

Notez le Code \emph{Sibylle}.

\ellipse

Décrypter les prophéties n'est jamais une partie de plaisir. Tant de cas de figures à envisager, d'interprétations à explorer pour au final n'acquérir aucune certitude. Cependant, vous ne pouvez vous permettre de laisser échapper la moindre piste, aussi ténue soit-elle.

Vous vous mettez donc au travail, analysant chaque mot, chaque tournure, à travers de le prisme de ce que vous avez déjà découvert sur cette région désolée.

Si vous possédez le Tome Scellé, rendez-vous au \glink{125}.

Sinon, votre bonne volonté n'est malheureusement pas couronnée de succès. Après avoir perdu beaucoup trop de temps en vaines spéculations, vous vous décidez enfin à mettre de côté les phrases énigmatiques. Pour le moment.

Rendez-vous au \glink{50}.

\gsection{22}

Si vous possédez le Tome Scellé et désirez le consulter, rendez-vous au \glink{28}.

Sinon, vous pouvez faire jouer quelques contacts fort savants sur ce genre de sujets, mais vivants loin d'ici, ce qui va prendre un peu de temps. Quand vous aurez accompli trois nouvelles phases de recrutement, et les quêtes correspondantes, vous pourrez vous rendre au \glink{99} pour y parcourir les informations recueillies.

Une fois vos recherches terminées, rendez-vous au \glink{93}.

\gsection{23 – Le château du comte}

La forteresse se dressait au sommet d'une petite montagne escarpée, un flanc face au vide. Une position stratégique imbattable, si quelqu'un avait voulu l'attaquer de front avec une armée. Mais les flammes n'en ont eu cure. Maintenant il n'en reste que des ruines noircies, qu'hommes et bêtes évitent. Mais aujourd'hui, une silhouette armée d'une pelle et d'un grand sac entame l'ascension au petit matin. À dix heures, après un premier tour d'exploration infructueux, elle commence à creuser. En début d'après-midi, elle met à jour un passage ayant échappé aux flammes.

À l'exception des blattes et des araignées, le souterrain est désert. Fort étendu, il englobe les restes d'une cave à vin, quelques réserves de venaison, une salle débordant de documents poussiéreux, et bien sûr la crypte. Celle-ci porte trace des violents combats qu'elle a abrités. Des morceaux de squelettes sont éparpillés un peu partout, la surface de certains murs a partiellement brûlé ou fondu, comme attaquée par un incendie localisé ou un acide, le sol est strié de griffures d'une bête ayant des crocs assez solides pour marquer de la pierre de taille, et un pieu est encore profondément enfoncé dans la paroi intérieure d'un cercueil, là où se serait trouvé le cœur de son occupant s'il y en avait encore un.

Si vous avez envoyé en mission quelqu'un disposant d'une Spécialité en Recherche, rendez-vous au \glink{12}.

Sinon, rendez-vous au \glink{81}.

\gsection{24}

Une chance surnaturelle semble protéger votre bandit. Les rondes des gardes paraissent naturellement l'éviter, un des guetteurs s'absente pour un besoin naturel juste au bon moment, la serrure se révèle triviale à forcer… Le destin conspire pour l'épauler.

Rendez-vous au \glink{127}.

\gsection{25}

Vous n'êtes pas aussi discret que vous voudriez l'être. Les gens murmurent en vous regardant, s'interroge sur le défilé de personnages louches que vous côtoyez. Vous vous efforcez de paraître le plus inoffensif possible, voire un peu gâteux, pour prévenir toute initiative malheureuse à base de fourches et de torches.
Cependant, à toute chose malheur à son bon, car votre petite réputation attire aussi des gens de meilleure compagnie.

La \hero{Princesse} est dès maintenant disponible.

Si vous en êtes aussi à votre septième recrutement, rendez-vous au \glink{52}.

Sinon, il est temps pour un nouveau cycle Recrutement/Quête de commencer.

\gsection{26}

Une sensation de déjà-vu. L'impression notable qu'une scène passée est en train de se produire. Besoin de se concentrer. De voir au-delà des apparences. Pas soleil. Nuit. Pas grotte. Marais. Créature modifie pensées. Résister. Se rappeler.

Dans une tentative désespérée de reprendre le contrôle de ses sens, l'être solitaire, agité de soubresauts dans la caverne vide, se mord violemment le poignet. Un acte stupide, irréfléchi, mais efficace.

Des bribes de souvenirs commencent à lui revenir, lui permettant de recomposer le fil des événements. L'illusion percée à jour perd alors tout impact et se désagrège d'elle-même.

Le piège était le même que dans le marais, mais en beaucoup plus fort, en beaucoup plus subtil. Mais le chausse-trappe mental du bourbier a comme unique avantage d'avoir fourni un bon entraînement mental, permettant de triompher de celui-ci aussi.

Rendez-vous au \glink{44}.

\gsection{27}

La Baba Yaga renifle la personne qui se trouve en face d'elle. Elle sent les effluves des noires sorcelleries qu'elle n'a pas hésité à employer lorsque le besoin s'en est fait sentir, elle sent la graine du mal. C'est un être à la morale flexible, qui ne demande qu'à être assouplie encore un peu plus. Alors l'antique sorcière décide de voir jusqu'à quel point. Elle extrait de son inventaire un objet de grand pouvoir qu'elle est prête à céder en échange d'un service ignominieux. Elle énonce ses termes, et, après une courte négociation, ils sont acceptés, à sa plus grande surprise.

Le sabbat est convoqué dès le lendemain. Seules les sorcières dignes de ce nom répondent à l'appel, les autres ayant bien trop peur de la Baba Yaga. Treize à former le cercle elles sont quand le sacrifice leur est amené, comme convenu, encore vivant. L'énergie issue de la tendre chair lui rend des années de vie, et c'est guillerette qu'elle offre sa sombre bénédiction à l'ombre sous forme humaine qui lui a fourni.

Notez le Code \emph{Sabbat}, et associez-le à celui ou à celle qui vient de le débloquer. Puis diminuez votre Réputation de 2.

\ellipse

C'est d'une voix sans émotion, contenue, morne, que vous recevez un rapport étrangement élusif. Vous préférez ne pas creuser la question. Le soutien de la Baba Yaga vous est maintenant acquis, ou du moins elle ne s'opposera pas directement à vous, et c'est déjà un exploit non négligeable.

Le présent qu'elle vous a indirectement offert, tant sa cible initiale semble heureuse de vous le transmettre, est un épais ouvrage ayant des siècles derrière lui. L'âge l'a décati, et il tient maintenant plus du paquet de feuillets emprisonnés entre deux tranches de cuir racorni que du vrai livre relié.

Mais plus que l'ancienneté, il exhale le mal, le pouvoir, la connaissance interdite. Il est maintenu fermé par des cordelettes entremêlées de fils d'argent, qui ont été récemment dénouées et renouées. Vous avez une bonne idée de l'identité du coupable, et n'êtes guère rassuré par le fait qu'il veuille s'en débarrasser aussi vite après l'avoir ouvert.

Ajoutez un exemplaire du Tome Scellé [-] à vos Possessions.

La sorcière du fond des âges a aussi parlé de trouver ses doubles inversés, ses contemporains des contes, ceux qui vivent par la lumière mais se cachent dans l'ombre.

Débloquez la Quête suivante :

\quest{Le cercle de la paix}{37}{Recherche (\caduceus)}{Moyenne}{
La Baba Yaga a désigné cette zone de la forêt comme abritant des êtres qui pourraient vous aider. Elle semble tout à la fois les mépriser et les respecter, mais fait beaucoup de mystères sur leur compte. À vous d'aller voir ce qu'il en est.
}

Et surtout, l'antique sorcière est remontée dans ses souvenirs pour vous offrir un aperçu du passé.

Rendez-vous au \glink{72}.

\gsection{28}

Il y a bel et bien un chapitre consacré à des créatures similaires à celle aperçue dans le marais. Son contenu n'est cependant pas de très bon augure. Elles sont nommées les enfants des dieux, les graines des rêves, les bourgeons de l'horreur. Ce sont des êtres se nourrissant du chaos, de la peur, de la haine. Elles altèrent naturellement leur environnement pour le conformer à leurs besoins, et ce sur tous les plans à la fois. Plus le temps passe, plus elles grandissent tandis que le monde où elles poussent sombre dans la folie et la destruction.

Une seule peut suffire à ravager une planète entière. Mais le pire, c'est que, comme leur nom l'indique, ce ne sont que des bébés. Et après une éternité à baigner dans les remous du chaos, l’œuf éclot. Et là, ce sont des univers qui peuvent mourir.

L'ouvrage n'indique aucune méthode pour déraciner le mal avant qu'il ne soit trop tard, ne dit même pas si c'est possible. Mais il décrit avec une joie morbide les conséquences néfastes de la simple croissance de ces créatures.

Vous dormez bien mal cette nuit-là.

Rendez-vous au \glink{93}.

\gsection{29}

Suivant un rituel bien huilé, tous les rideaux se soulèvent soudain, déversant un flot de fidèles encapuchonnés... Et armés !

Un homme que tout indique comme le grand prêtre de cette sinistre messe prend alors la parole pour sommer l'intrus de se rendre.

S'il s'adresse au \hero{Croisé}, rendez-vous au \glink{46}.

S'il parle au \hero{Chasseur}, à l'homme à la \hero{Main d'Acier}, au \hero{Mage}, à la \hero{Légende} ou à un quelconque  combattant revêtu de la Cape de Crocs, rendez-vous au \glink{105}.

Sinon, rendez-vous au \glink{36}.

\gsection{30}

\success

Les histoires ne cessent pas, toujours aussi aguicheuses, toujours mélange subtil de vérité et de mensonge. Tout est promis, tout est possible. Mais le piège ne fonctionne plus, et une volonté d'acier se fraye un chemin vers ce dont on a voulu la détourner.

Le cœur du marais se révèle alors. Une gigantesque masse vivante, informe, palpitante. Un amas boursouflé, pustuleux, suintant une substance noirâtre qui se mélange à l'eau. L'être est immobile, profondément enfoncé dans le sol, mais sa simple présence physique est suffisante pour mener aux abords de la syncope, à la différence de son existence psychique, plus insidieuse, plus subtile, qui n'attend qu'un instant de relâchement pour s'emparer des sentiments de sa cible et les retourner contre elle.

\ellipse

Vous êtes mis au courant de la réalité de cette créature avec le retour de votre spécialiste, qui n'a pas commis la folie de l'attaquer sur le champ, vous permettant ainsi de mieux préparer le futur. Débloquez la Quête suivante :

\quest{Au cœur du marais}{60}{Combat (\caduceus)}{Insensée}{
Vous avez découvert ce qui se cache au milieu des marais, pour votre plus grand malheur. Reste à vous en débarrassez, ce qui s'annonce, sans surprise, fort difficile.
}

Vous apprenez également quel danger attende votre futur candidat avant même de croiser la bête. Notez que vous et la personne venant de le parcourir disposaient dorénavant de l'Information \emph{Le Chemin du marais}.

Si vous avez le Code \emph{Vengeance}, rendez-vous au \glink{89}.

Sinon, rendez-vous au \glink{50}.

\gsection{31}

L'élément de surprise n'aura pas été déterminant. Dans ce genre de circonstances, n'importe qui s'attend à ce qu'un cadavre ne reste pas immobile très longtemps. Mais même en étant préparé, affronter au corps-à-corps un mort-vivant, un être inépuisable, ne ressentant pas la douleur, qui plus est doté d'une force phénoménale, n'est pas une chose facile.

Le bras de chair du  vampire repousse aisément celui de sa victime, tandis que sa main d'os se place dans son dos, et pousse un cou tendre à portée de ses canines. Et un instant plus tard, la prise du monstre est définitivement assuré par son imparable morsure. Malgré quelques derniers efforts désespérés, le sang de la vie passe sans discontinuer d'un organisme à l'autre.

Rendez-vous au \glink{33}.

\gsection{32}

Ce combat est un mêlée étrange. Votre mercenaire focalise l'attention de la créature, qu'il ne cesse d'aiguillonner. Équipement supérieur, expérience ou habileté naturelle lui donnent un ascendant qui empêche la bête de se détourner de lui pour s'occuper des villageois qui la harcèlent. Il n'est pas clair à quel point la transformation affecte ses capacités cognitives, mais elle se laisse manipuler comme un taureau dans l'arène. Elle s'effondre finalement rapidement sous les assauts combinés de la population en furie sans que les différents protagonistes n'aient à souffrir plus que quelques belles balafres.

Ajoutez 1 point à votre Réputation.

Rendez-vous au \glink{115}.

\gsection{33}

Toujours aucune nouvelle. Il faut se rendre à l'évidence, soit vous avez simplement été allégé de vos possessions par une personne peu désireuse d'en faire plus pour vous, soit votre confiance était bien placée, mais un malheur est arrivé.

Vous avez cessé d'être sentimental à propos des accidents qui arrivent à vos protégés depuis longtemps. Trop de pertes, trop fréquemment. Aussi, est-ce ce avec la force de l'habitude que vous cherchez déjà un remplaçant.

Si vous aviez envoyé un héros nommé, il devient définitivement indisponible. Il ne peut plus être recruté d'aucune façon et son éventuel équipement est perdu à jamais.

Rendez-vous au \glink{50}.

\gsection{34}

Ce n'est pas une histoire précise qui s'échappe des abysses du temps, mais un kaléidoscope d'événements passés, une déferlante de souvenirs de tous les âges. À chaque nouvelle vision, l'apparence du marais change légèrement, pour que l'accord soit parfait, que le présent semble une conséquence logique de ce qui fut. Mais aucune de ces innombrables fables n'accroche l'esprit de l'entité qui continue d'avancer tranquillement, sans dévier d'un pouce de son chemin.

L'attaque redouble de violence, les légendes se mélangent en une bouillie confuse de sons et d'images, promesses de trésors, de secrets, de pouvoirs, mais tout cela se révèle vain face à une psyché qui ne réagit pas ou plus selon les normes de l'humanité.

Rendez-vous au \glink{30}.

\gsection{35}

Vous êtes sur le point d'aller vous coucher quand une pulsion irrationnelle, un instinct inexplicable, vous pousse à vous coiffer du cercle végétal. Vous vous alitez ensuite, la couronne de fleurs posée sur votre crâne, et passez une excellente nuit, fraîche, calme et reposante.

C'est en pleine forme que vous reprenez votre travail dès le lendemain. Rendez-vous au \glink{93}.

\gsection{36}

À cent contre un, l'intrus est rapidement maîtrisé, capturé et déposé sur l'autel. La cérémonie commence alors, une sombre invocation aux puissances de la non-mort. Et lorsque les chants religieux atteignent leur apogée, le grand prêtre plonge le poignard rituel dans le cœur du sacrifice.

\ellipse

Des spectres ont tourné autour de l'auberge où vous logez toute la nuit dernière. Les pâles silhouettes blanches, intangibles mais audibles, ont hurlé des malédictions à votre encontre sans discontinuer pendant des heures, en une sarabande infernale.

Si les fantômes ont disparu au lever du jour, cet incident vous a fait passer une nuit blanche et vous a quelque peu démoralisé. Et il a bien sûr engendré un certain nombre de rumeurs, pas forcément tendres à votre égard.

Diminuez votre Réputation de 2.

Rendez-vous au \glink{33}.

\gsection{37 – Le cercle de la paix}

Il existe une portion de la forêt que les gens et les animaux évitent naturellement. Elle n'est ni dangereuse, ni insalubre, ni dévastée, ni même de mauvais augure. Simplement, quand la possibilité se présente de s'y rendre ou d'aller n'importe où ailleurs, une force pousse toujours à choisir le ailleurs.

Mais le charme n'est pas assez fort pour détourner de son but quelqu'un désirant explicitement y pénétrer, en toute connaissance de cause. En ce cas, il découvrirait rapidement que les bois sont bien plus peuplés qu'ils ne veulent le paraître. Des créatures s'enfuiraient devant lui, se réfugiant dans la végétation avant qu'ils ne puissent tout à fait distinguer ce dont il s'agit. La nature se révélerait plus fertile et plus nourricière qu'elle ne devrait, les arbres arborant encore de beaux fruits, dont certains en partie mangés.

Et s'il s'obstinait dans sa quête, il ne tarderait pas à voir son chemin bloqué par un noble chevalier à l'armure argentée qui le sommerait de s'en aller.

Si vous avez désigné la \hero{Sorcière} pour cette tâche, rendez-vous au \glink{134}.

Si vous avez envoyé quelqu'un disposant de la Spécialité Diplomatie, rendez-vous au \glink{102}.

Si ce n'est pas le cas, mais que le Combat est plus son domaine, rendez-vous au \glink{80}.

Sinon, rendez-vous au \glink{54}.

\gsection{38}

L'anneau du leprechaun ne réagit pas. Ce n'est qu'un détail, mais le genre de détails qui une fois remarqué ne peut plus être ignoré. Un élément discordant qui rompt l'illusion, brise la transe, met fin à la magie. En un instant, il n'y a plus de civilisation disparue, plus d'or. Il n'y a qu'un homme seul au fond d'un trou, dont les coups de pelles rageurs ont mené au bord l'effondrement les parois du tombeau qu'il s'est lui même creusé.

Dès que son esprit réalise de quelle manipulation il a été victime, il se transforme en être furieux, à qui la rage donne la force de triompher de ses muscles endolories pour s'extraire de la terre peu avant qu'elle ne l'enferme.

Notez le Code \emph{Vengeance}.

Rendez-vous au \glink{30}.

\gsection{39}

Vous vous réveillez une demi-seconde avant que la griffe du monstre ne vous arrache le visage, et vous jetez sur le côté juste à temps pour qu'il ne vous fasse qu'une large entaille au cuir chevelu et à l'oreille. Pas beau à voir, mais toujours mieux que de perdre ses deux yeux.

La goule qui vous a attaqué est encore fraîche. Et pour cause, même si sa peau est si pâle qu'il ne doit ne plus rester une goutte de sang dans son corps et que ses ongles et ses dents ont retrouvé leurs racines préhistoriques, il n'y a pas à s'y tromper :  c'est la personne que vous aviez tout récemment envoyé enquêter sur le messie des nécromanciens.

C'est dans ce genre de situations que vous regrettez le plus de ne plus disposer de vos pouvoirs. Une bonne boule de feu aurait réglé le problème en deux temps trois mouvements. Mais dans votre état actuel, vous n'avez cependant d'autre choix que de fuir en hurlant à pleins poumons.

Vous ne savez s'il faut rire ou pleurer du chaos qui s'ensuit. Un voisin alerté par le bruit a le malheur de sortir de sa chambre au mauvais moment. Le propriétaire, dérangé dans son sommeil, vient lui aussi vous houspiller, avant de prendre ses jambes à son cou en comprenant la situation. Le problème est finalement réglé en quelques coups de hache par un de ces mercenaires à la petite semaine qui vous proposent régulièrement leurs services, qui demande bien sûr paiement pour son travail.

Pas la meilleure soirée de votre vie concluez-vous alors qu'un boucher local vous recoud comme il peut. Au moins êtes-vous encore en état de continuer votre quête. Si vous aviez été amoché juste assez pour être mis hors d'état de poursuivre mais pas pour succomber, ce qui n'a rien d'impossible tant il est notoirement difficile de vous achever, la frustration d'échouer si près du but vous aurait probablement rendu fou.

Diminuez votre Réputation de 1 et votre or de 5. Effacez ensuite le Code \emph{Corruption}.

Enfin, retournez au \glink{50}.

\gsection{40}

Alors qu'il s'avance vers la masse chaotique, l'homme aux mille visages se demande ce qui lui a pris de venir ici. Cette histoire était amusante un moment, mais là, il s'attaque à un beaucoup trop gros poisson. Qu'importe le sang qui coule dans ses veines, il n'est pas de taille contre cette créature. C'est un dévoreur de mondes, une bête qui ronge les racines de la création jusqu'à qu'elle s'effondre sur elle-même. Lui est à peine plus qu'un simple humain, une frêle créature à la vie éphémère et fragile.

Son père, lui, aurait accouru devant un tel défi, ne serait-ce que pour le plaisir de démontrer sa supériorité aux autres en réalisant l'impossible devant eux. Mais il n'est pas son père, si tant qu'il s'agisse bien de son père, car rien ne lui prouve qu'il ne s'illusionne pas lui-même, qu'il n'ait pas défini cette personnalité précise comme principale simplement sur les divagations d'une vieille gâteuse.

Toutefois, il ne peut empêcher les flammes de l'excitation de le pousser de l'avant. Il a un plan. Un plan incroyablement peu fiable, qui l'amènera dans ces derniers retranchements, mais qui a une chance d'aboutir. Et il sait qu'il ne pourra pas repartir d'ici avant de l'avoir essayé. Héréditaire ou pas, cette folie est ancrée en lui.

L'objet qu'il tient entre ses mains est la clé. Il ne soucie pas de son intérêt pratique. Ce qui est important, c'est ce qu'il symbolise, le lieu et le temps qui l'ont vu naître. Une époque révolue, où la limite entre l'humain et le divin était floue. C'est un catalyseur pour atteindre une personnalité particulière, une de celles qui sont tellement puissantes qu'il doit les ensevelir tout au fond de lui pour ne pas être submergé par elles.

Disposer d'un support matériel n'est que la première étape. Une concentration extrême constitue la seconde, tandis que son esprit actuel bat en retraire pour faire place à celui qui peut vaincre son ennemi. La troisième, celle qui l'ancrera dans ce corps, est encore une fois un nom, car les noms ont des pouvoirs.

Et alors que les syllabes franchissent ces lèvres, il se transforme comme jamais il ne s'était transformé.

Rendez-vous au \glink{62} si le \hero{Métamorphe} porte le Coffret des Malheurs et au \glink{129} s'il tient plutôt la Branche de l'Arbre-Vie.

\gsection{41}

« À genoux ! »

L'être répète son ordre, impérieusement cette fois-ci, puis encore, et encore, rageusement, désespérément. Comprenant que ses pouvoirs sont sans effet, il tente de s'approcher, mais il n'a fait un pas qu'il se heurte violemment à la barrière invisible.

Voilà l'incarnation de la tombe. Voici celui qui est revenu d'entre les morts pour hanter les vivants. Un être pitoyable, confiné dans quelques mètres carrés.

Cependant, il retrouve bien vite sa contenance, et sourit même de toutes ces dents.

« Voilà bien trop longtemps que je n'avais pas croisé quelqu'un capable de me résister ne serait-ce qu'un peu. Les choses redeviendraient-elles enfin intéressantes ? Si vous venez pour me tuer, je dois cependant vous avertir que vous perdez votre temps. Croyez-moi, cela fait des millénaires que des personnes bien plus dangereuses que vous essayent, mais aucune n'a réussi. Quoi que vous fassiez, je continuerai à vivre. »

Commence alors une longue conversation. Le non-mort se montre serviable, répondant aux questions sans langue de bois. Il se peut que sa simple présence permette aux autres morts-vivants de s'affranchir de certaines règles, mais qu'y peut-il ? Ce n'est pas lui qui a choisi d'être scellé en ces terres.

« Pour être exact, je n'étais même pas enfermé ici à l'origine. Auparavant, j'étais dans un autre lieu, mais le puissant roi-mage qui le dirigeait en eut un jour accès des conséquences de ma présence sur son pays et usa de ses pouvoirs pour m'envoyer, moi et toute cette montagne qui m'emprisonne au diable Vauvert. Je pense qu'il avait choisi le fond d'un océan ou le cœur d'un désert comme destination, mais son sortilège ne se passa pas comme prévu et j'ai atterri ici. Mon hypothèse est que le dieu enfoui des marais a interféré pour m'attirer. Je pense que j'ai des qualités qui l'intéressent, notamment de pouvoir rompre l'ordre normal des choses. »

Et de conclure :

« C'est lui votre véritable ennemi, vous savez. Moi-même, je ne suis qu'une conséquence de ses caprices et ne demanderais pas mieux qu'à partir loin d'ici. Le fait que nous puissions tenir cette conversation sans que vous ne rampiez à mes pieds m'indique que vous n'êtes pas ordinaire. Peut-être que nous pourrions nous aider mutuellement. Voyez-vous, je connais peut-être moyen de vous débarrasser de votre problème. Je me trouve d'ailleurs en ce moment-même au centre de ma solution.

En effet, le sortilège qui a été utilisé pour m'enfermer ici est terriblement puissant. J'irais jusqu'à dire que mes anciens ennemis m'ont probablement un peur surestimé. C'est une prison du genre qu'utilisent les nouveaux dieux pour enfermer leurs prédécesseurs. Elle fonctionne dans les deux sens. Ce qu'elle emprisonne, moi en l’occurrence, ne peut la quitter, mais personne d'autre ne peut non plus y entrer. La force d'attraction est infiniment supérieur à celle de répulsion, mais si vous vous amusiez à semer des petits cailloux, vous les verriez se déplacer lentement mais sûrement vers la sortie. Cela a l'avantage de m'éviter de faire les poussières, mais c'est aussi pour cela que je suis incapable de conserver des vêtements bien longtemps.

Cependant, de très longues années où je n'avais rien de mieux à faire que de l'étudier m'ont permis d'en comprendre le fonctionnement dans les moindres détails. Et je pense pouvoir le reproduire. Je ne veux pas paraître prétentieux, mais je suis sûr qu'un sceau d'une telle ampleur pourrait neutraliser n'importe qui, même la créature des marais. »

Plusieurs pièges sont évidents dans le discours charmeur du non-homme, et son interrogatoire se fait féroce.

« Quel intérêt a ce sceau si l'influence de ce qu'il contient peut s'étendre à l'extérieur ? Je vous rassure, ce n'était pas censé être le cas lorsqu'ils l'ont conçu. Cependant, je suis parvenu à trafiquer quelques runes lors de sa réalisation, dans l'espoir de me ménager une porte de sortie. Je n'ai cependant réussi qu'à me percer ce petit trou d'aération. N'ayant aucune envie que ce monstre continue à corrompre mes frères et sœurs, je ne lui laisserai pas l'opportunité de me faire un coup semblable.

Ce que j'y gagne ? Et bien voyez-vous, ce type de sceau a une petite particularité. Rien de grave, simplement il est tellement puissant qu'il ne peut en exister deux en simultané dans une zone d'espace aussi réduite. Sinon ils interférent l'un avec l'autre et se neutralisent mutuellement. Le seul moyen d'en créer un nouveau dans cette région sera donc de faire disparaître celui-ci au fur et à mesure que l'autre naîtra. Pour chaque cercle de protection qui sera établi autour du monstre, un des miens, devenu inopérant, devra être effacé. Ainsi, lorsque sa prison sera complète, la mienne sera détruite. Le seul prix que je demande pour mes services est donc la liberté. »

Rendez-vous au \glink{118}.

\gsection{42}

La descente vers les profondeurs se prolonge, infiniment. Le monde se limite à une minuscule sphère autour de la source de lumière magique, faible, pâle, mais constante, rassurante.

Et bientôt, la nervosité s'atténue au profit de la routine. Les ténèbres devant et derrière ne cachent que des marches, pas des horreurs indicibles. Les squelettes desséchés d'autres aventuriers effondrés çà-et-là ne sont plus qu'un élément de décor répétitif.

Et lorsque la peur disparaît, l'escalier se finit.

Et au-delà de cet enfer se trouve le paradis. Non pas un monde parfait idyllique de fleurs, de plumes, de nuages et d'arcs-en-ciel, juste le même que celui d'en haut avec quelques petites différences. Ceux qui n'auraient pas dû mourir mais que la fatalité a rattrapé sont en vie. Ceux qui ont souffert ou souffraient encore d'un acharnement immérité du destin ont maintenant une existence meilleure. Le méchant est toujours puni et le juste toujours récompensé, qu'il soit riche ou pauvre, grand ou petit.

Si la personne présente ici a déjà accompli pour vous, avec succès, la Quête \textbf{Le marais interdit}, rendez-vous au \glink{26}.

Sinon, mais que vous lui avez appris la \emph{Formule du Souvenir}, rendez-vous au \glink{98}.

Dans tous les autres cas, rendez-vous au \glink{33}.

\gsection{43}

Les mots seuls ne semblent pas suffisants pour convaincre l'assemblée, et bientôt crachats et sifflets fusent. Mais alors qu'il reprend son souffle, le réceptacle de toute cette vindicte populaire sent de le poids de l'instrument dans la poche intérieure de son vêtement, oublié là – volontairement ou non – par ses geôliers et juges. Alors pris d'une inspiration, il s'en saisit et joue.

Des souvenirs longtemps oubliés remontent à la surface. Des traces d'un lointain passé, bien avant qu'il ne soit jeté sur les routes, à errer à travers le monde. Poussé par la magie du lieu, de l'instant, il transcende sa maigre technique pour produire une mélodie nouvelle, met à nu son âme et son ambition dans le déferlement de la musique. Le récital est aussi court qu'intense, et lorsque la dernière note se meurt, le silence est complet.

Rendez-vous au \glink{20}.

\gsection{44}

\success

Tout ce dédale, toute cette marche pour terminer dans un minuscule cul de sac rocheux, un réduit qui nécessite de se plier en deux pour en atteindre l'extrémité. Et au fin fond de cette ode à la claustrophobie se trouve l'ombre d'une femme. Repliée en position fœtale, elle sert contre elle de toutes ses forces une cassette ornementée. Tremblante, maigre à faire peur, en haillons et couverte de crasse, ses yeux désespérés sont posés sur l'intrus qui la contemple de haut. De ses lèvres craquelées s'échappe une voix rauque, et elle raconte une histoire. Les termes qu'elle emploie sont archaïques, issus d'une langue depuis longtemps morte, mais le sens se transmet tout de même.

Le conte parle d'une divinité tombée amoureuse d'une mortelle. Mais celle-ci avait le malheur d'en aimer un autre. Pour se venger, le dieu lui donna alors un magnifique coffret en lui ordonnant de ne jamais l'ouvrir. Ce qu'elle fit dans un premier temps, jusqu'à ce que le dieu, à bout de patience, monte tout une machination pour l'y contraindre. Alors de la boîte s'échappèrent tous les maux du monde, les pires maléfices qui soient, et ils se répandirent sur ce qui deviendrait plus tard l'Ylèdre. Elle la referma aussi vite qu'elle le put, réussissant à maintenir quelques horreurs supplémentaires enfermées. Pour la punir de lui avoir désobéi, le dieu l'enferma alors ensuite ici pour toute l'éternité, avec l'objet de sa honte et de sa déchéance.

Une fois l'histoire terminée, elle tend la cassette, toujours fermée. Lorsque celle-ci est attrapée, elle murmure un dernier avertissement avant de s'effacer comme le brouillard sous le vent. L'instant suivant, il ne reste qu'une seule personne perplexe dans la caverne, un coffret à la main.

\ellipse

C'est un corps gravement épuisé par le trajet qu'il a dû accomplir pour revenir qui vous fait face, n'ayant jamais osé s'arrêter pour dormir ne serait-ce qu'un moment sur le chemin du retour, de peur de ne jamais se réveiller. Mais c'est un esprit encore actif, circonspect, qui vous narre l'histoire. Qui était la femme ? Une nouvelle illusion ? Un spectre ? Un souvenir ? Une incarnation ? Difficile à dire. Et de même que penser de sa dernière phrase ?

« Il est parti, mais il a laissé son fils – notre fils – derrière lui sur le lieu de notre première rencontre, au croisement de l'ourse et du saumon. »

Après vérification, il s'avère que ces deux termes sont encore utilisés, dans leur forme ancienne, pour désigner une certaine rivière et une certaine montagne. Le croisement en question correspondrait aujourd'hui au Marais Noir, un lieu sordide même selon les standards de cette région.

Débloquez la Quête suivante :

\quest{Le marais interdit}{85}{Recherche (\ankh, \caduceus)}{Difficile}{
Si vous avez correctement déchiffré les paroles de la femme fantôme, ce marais doit contenir le fils en question. Nul doute qu'une investigation du lieu pourrait vous en apprendre beaucoup, mais vous ne prenez pas les avertissements divins à la légère, aussi feriez-vous mieux de mettre toutes les cartes de votre côté avant de vous y risquer.
}

Reste la boîte. Elle vous a été remise fermée, et la prudence la plus élémentaire voudrait qu'elle le reste. Mais cette décision n'appartient qu'à vous.

Ajoutez le Coffret des Malheurs [-] à vos Possessions. 

Rendez-vous au \glink{50}.

\gsection{45}

C'est une forme d'hypnose puissante qui est utilisée, propre à faire plier la plupart des êtres humains. Toutefois, elle n'est pas employée contre un humain normal, mais contre une personne qui a déjà affronté bien pire. Ses défenses mentales surentraînées se dressent d'elles-mêmes, et la voix n'a pas d'autre effet que de transmettre du son.

Rendez-vous au \glink{41}.

\gsection{46}

\success

La confiance du saint soldat vacille alors que son corps faiblit peu à peu sous les coups des innombrables infidèles. Même avec sa foi, même avec l'expérience des batailles passées, il ne peut tenir le rythme et il le sait. Son âge se fait sentir alors qu'il tente inlassablement de repousser l'ennemi, cette bande de petites frappes que sa faiblesse galvanise.

Le grand responsable de cette mascarade est si sûr de la victoire de ses troupes qu'il prépare déjà les ustensiles pour le sacrifice. Il nettoie le kriss rituel, et dépose sur l'autel, sur un coussin de velours, une sphère d'un noir profond, une perle d'obscurité irradiant de pouvoir.

Quand le croisé l'aperçoit, quand il voit ce faux prêtre souiller de ses doigts cette sainte relique, une fureur sans pareille s'empare de lui. Il rugit un cri de guerre ancestral, et charge le mécréant. Les sectateurs tentent de l'arrêter, mais ne peuvent rien contre sa rage sans pareille et le ralentissent à peine. Un instant plus tard sa lame vengeresse transperce le gourou.

L'objet sacré dans une main, son épée dans l'autre, il se retourne face à la meute. Une litanie s'échappe de ses lèvres, une longue prière continue, alors qu'il protège la relique de son corps, la serrant contre son cœur sans cesser de se battre.

Quelque chose se produit alors, un événement qui sera plus tard ajouté à liste officielle des miracles de son église. La sphère se met à rayonner, et des esprits se matérialisent. Pas des spectres vengeurs, mais les fantômes des martyrs de la foi, des saints qui sont morts pour leurs convictions. Chevaliers du passé, prêtres antiques, simples croyants transcendés apparaissent, auréolés de lumière, formant une muraille autour du porteur de la larme.

Une vague de terreur s'empare alors des adeptes de la non-vie lorsqu'ils comprennent que cette fois, les morts revenus sont dans le camp d'en face. C'est alors la débandade générale. Ils s'enfuient sans demander leur reste, sans regarder derrière eux. Un bon nombre se jetteront droit dans les filets de l'Ordre.

Et rapidement, alors que les saints commencent déjà à s'effacer, il ne reste plus dans la pièce que le croisé, blessé, épuisé, mais bien vivant.

\ellipse

Le soldat de dieu n'est pas au meilleur de sa forme quand il revient vous voir, et vous devez même faire mander un médecin en urgence, mais il est dans un état de béatitude tel qu'il ne ressent plus la douleur. Sa main est crispé autour d'un objet sphérique qu'il se refuse à lâcher, et vous abandonnez vite de lui l'idée d'obtenir un récit cohérent avant que quelques heures de sommeil ne lui ait rendu sa raison.

Le lendemain matin, son esprit est plus clair, mais son histoire reste déformée par une ferveur mystique toute fraîche qui rend certains passages difficilement compréhensibles. Vous assumez cependant que la mission a été un succès, la secte ayant été dispersée.

Il est aussi fou de joie et acceptera d'effectuer pour vous une ultime mission, en remerciement, avant de rentrer déposer la relique en son sanctuaire. Vous pourrez lui confier une dernière Quête en ignorant ses conditions habituelles d'embauche, après quoi il deviendra pour de bon définitivement indisponible.

Notez le Code \emph{Mystique} et augmentez votre Réputation de 2.

Si vous avez le Code \emph{Loi}, rendez-vous au \glink{4}. Sinon, rendez-vous au \glink{50}.

\gsection{47}

En théorie, ce combat devrait être joué d'avance. D'un côté, une créature taillée pour l'escrime, disposant d'une arme capable de percer toute armure. De l'autre, un être de chair et de sang, avec ses faiblesses, ses erreurs, ses manquements.

Mais la vérité est toute autre. Car la bête de guerre souffre d'hésitations. Ses mouvements sont saccadés, inachevés, s'interrompant brutalement avant de reprendre un peu plus tard. Et ces errements sont en train de lui faire perdre le duel.

La raison de ce comportement suicidaire est transparente : le pantin dispose encore de quelques restes de conscience, de volonté. Et plus la lame perd du terrain, plus son porteur subit de coups, qui ne font pourtant qu'érafler son nouveau corps d'acier, plus le contrôle lui échappe.

Mais ce qui se cache sous cette armure ne peut lâcher l'épée pour autant. C'est dorénavant un morceau de son organisme, profondément entrelacé avec ses systèmes vitaux. Seule une violence calculée peut les séparer, une amputation radicale.

Une opération chirurgicale à laquelle son adversaire recourt, sans anesthésie, en sectionnant net le poignet lors d'un habile assaut. Avec un réflexe animal, la créature d'acier recule d'un bond, compressant son moignon de son autre main pour ralentir l'hémorragie. L'autre camp reste perplexe. À ses pieds gît l'objet de son désir, accessible simplement en tendant le bras, sans personne pour l'empêcher de s'en saisir. Mais l'effort intense du combat a quelque peu dissipé les sirènes trompeuses de la tentation. Tous ses instincts sont maintenant pleinement réveillés, et lui hurlent de ne pas toucher à la maudite épée.

\ellipse

Les années ont beau passer, vous restez toujours étonné devant la résilience extraordinaire dont font preuve certains de vos élus. La personne devant vous n'a certes pas l'air fraîche, entre son bras bandé, sa peau anémiée, pâle et craquelée d'où se détachent peu à peu des éclats argentés ou les tremblements convulsifs symptomatiques de son sevrage brutal. Mais elle est bien vivante, et, si vous pouviez recoudre sa main, serait probablement prête à repartir à l'assaut après seulement quelques jours supplémentaires de repos. Son adversaire d'un court moment, qui est aussi la bonne âme l'ayant ramenée ici, vous a déjà fait son rapport. Et avait également apporté dans sa besace, à l'aide de pinces improvisées en branchages pour éviter tout contact direct, la source de ce problème précis.

Vous n'avez pas réussi à l'identifier précisément. Le monde ne manque pas d'épées maudites. Qu'elles aient été forgées dès l'origine par des personnes peu recommandables dans des buts inavouables, qu'elle soient devenues après coup le réceptacle de l'âme d'un nécromant coriace ou qu'elles aient juste l'apparence d'une épée mais soient en réalité une toute autre créature, il y a l'embarras du choix.

En discutant avec sa victime, en analysant ses souvenirs, ses sentiments, vous ne parvenez toujours pas à trouver son nom, mais cela confirme certains de vos soupçons par rapport à sa nature. C'est une arme antique, datant d'une époque où l'homme n'était pas encore l'espèce dominante. Elle a été créée pour des êtres d'un tout autre niveau de puissance, et une partie de leurs forces coulent en elle. Elle dispose de sa propre conscience, son propre orgueil, et se refuse à être manipulé par une espèce inférieure telle que la vôtre.

Toutefois, l'expérience a prouvé qu'il était possible, sinon de la contrôler, du moins de résister a minima à son influence. Un aventurier spécialement adapté et prévenu à l'avance des conséquences que l'empoigner auraient sur lui pourrait peut-être en tirer quelque chose. Un tel pouvoir est en mesure de faire basculer le destin en votre faveur. Mais il a aussi de grandes chances de vous brûler les doigts.

Ajoutez la Lame Maudite à vos Possessions [-].

Rendez-vous au \glink{30} pour entendre la fin du rapport de votre missionnaire d'origine.

\gsection{48}

L'histoire parle d'un dieu oublié, d'un dieu dont le culte fut interdit, les fidèles pourchassés, et le nom même banni des archives. Elle parle d'un temple reculé, caché au cœur de l'endroit le plus inhospitalier qui soit, loin de ceux qui pourraient vouloir le détruire. Elle parle du secret de ce dieu, de la raison pour laquelle il fut si impitoyablement banni, du don qu'il offre à ceux qui le vénèrent.

Les pas de l'aventurier l'ont naturellement conduit devant l'autel, au cœur d'une antique chapelle en ruines. Malgré son âge et son isolement, elle montre les traces de passage non pas fréquents, mais non pas rares non plus. Le couteau et le calice sont en place sur la table de pierre, rutilants, prêts à servir. Il lui suffirait d'accomplir les gestes vus dans le souvenir, de prononcer les paroles entendus, et la magie s'opérerait : la mort relâcherait son emprise sur la personne aimée, et une seconde vie lui serait accordée.

Cet affranchissement des règles de la nature a un coût, en sang, à payer dès maintenant et à jamais, mais cela repousse à peine la tentation. Une véritable résurrection, la possibilité de communiquer à nouveau à travers le voile de l'au-delà, cela ne mérite-il pas de payer n'importe quel prix ?

L'homme reste indécis, immobile dans la brume froide des marais, pendant une éternité. Ce n'est qu'à la timide apparition de l'aube qu'il sort de sa transe. Ses doigts se referment sur la coupe d'or... Et il la fracasse sur le rebord avec une rage si noire qu'elle ferait reculer le diable en personne. Et l'instant d'après, il éclate en sanglots.

Notez les Codes \emph{Vengeance} et \emph{Rituel}.

Rendez-vous au \glink{30}.

\gsection{49}

Vous vous jurez de rédiger chaque jour un rapport détaillé de ce vous avez appris, et de le poser sur votre table de chevet chaque soir, pour être sûr de ne pas oublier de le relire chaque matin.

Notez le Code \emph{Réminiscence}.

Rendez-vous au \glink{93}.

\gsection{50}

Avant toute chose, si vous avez le Coffret des Malheurs, vous pouvez décider de l'ouvrir. Rendez-vous dans ce cas au \glink{66}.

Si votre inventaire contient à la fois le Tome Scellé et la Lame Maudite mais pas le Code \emph{Sceau}, rendez-vous au \glink{113}.

Si vous disposez du Code \emph{Corruption}, rendez-vous au \glink{39}.

Si le Code \emph{Sibylle} et le Tome Scellé sont tous deux en votre possession, rendez-vous au \glink{125}.

Si vous avez à la fois les Codes \emph{Religion}, \emph{Rituel} et \emph{Éternel}, mais pas \emph{Eurêka}, rendez-vous au \glink{8}.

Si vous ne possédez pas le Code \emph{Mémoire}, mais que vous disposez à la fois du Code \emph{Réminiscence} et du Tome Scellé, rendez-vous au \glink{58}.

Si vous avez débloqué la Quête \textbf{Le marais interdit}, qu'elle est encore d'actualité et que vous n'avez pas le Code \emph{Mémoire}, rendez-vous au \glink{79}.

Si vous avez débloqué la Quête \textbf{Au cœur du marais}, et que vous voulez faire quelques recherches à son sujet, rendez-vous au \glink{22}.

Sinon, rendez-vous au \glink{93}.

\gsection{51 – Les cavernes du début du monde}

La première partie du périple est une épreuve pour le corps. Il faut escalader un rocher acéré, un empilement de parois verticales, parfois à l'aide d'un sentier de chèvre oublié, parfois simplement à la force des bras. Nul humain ordinaire ne peut arriver au sommet sans être fourbu à l'extrême, les membres douloureux, rouge comme une pivoine et trempé de sueur, s'il y arrive.

Une fois cette épreuve franchie, il pourra découvrir l'excavation, creusée dans un éboulis de rochers. Ceux aux sens les plus aiguisés ou aux savoirs les plus étendus pourront reconnaître dans ces débris les traces de ce qui fut jadis un portail ouvragé, maintenant ravagé par les éléments. Mais l'escalier a lui survécu au temps, et passé les premières marches érodées, il se révèle aussi rigoureux et rude qu'à son origine.

Personne n'a jamais réussi à tenir le compte exacte de ses marches. Assez pour que toute lampe, toute torche, finisse par s'éteindre avant d'en voir le bout. Assez pour que personne ne puisse les remonter à temps pour échapper à la soif, la faim ou la folie lorsqu'il s'apercevra que ses réserves d'eau, d'aliments, de lumière ou de courage ne seront pas suffisantes.

C'est la deuxième, et bien plus mortelle épreuve, celle de la volonté.

Si vous avez envoyé le \hero{Mnémonique}, rendez-vous au \glink{17}.

Si vous avez désigné la \hero{Bête}, rendez-vous au \glink{127}.

Si vous avez fourni un Anneau de Lumière ou une Pierre de Feu à la personne que vous avez envoyée dans cet enfer, ou qu'il s'agit du \hero{Mage}, rendez-vous au \glink{42}.

Dans les autres cas, rendez-vous au \glink{33}.

\gsection{52}

Les chemins du destin sont capricieux et se recoupent parfois. Un des aventuriers dont vous avez déjà loué les services semble désireux de vous soutenir encore une fois. Vous pouvez choisir un bonus parmi les suivants :

\begin{itemize}
\item Le \hero{Métamorphe} redevient disponible (quelles que soient les circonstances qui ont amené à cette indisponibilité).
\item La \hero{Princesse} redevient disponible (uniquement possible si elle a été rendue indisponible par son propre effet ou que vous possédez le Code \emph{Héritage})
\item Le \textbf{Beau Parleur}, le \textbf{Brigand}, la \textbf{Sorcière} ou le \textbf{Chasseur} (un au choix) retourne à son tarif de base, c'est-à-dire qu'au lieu de payer le coût « Embauches ultérieurs » la prochaine fois que vous ferez appel à elle ou à lui, vous paierez seulement le tarif « Première embauche ».
\item Le coût du prochain aventurier anonyme auquel vous ferez appel est réduit à 0.
\end{itemize}

Et ce coup de chance tombe à point, car il est justement l'heure pour vous de recruter quelqu'un pour une nouvelle quête.

\gsection{53}

Le vampire n'est même pas complètement sorti de son sarcophage qu'il s'immobilise soudain en plein mouvement, paralysé, sur un simple claquement de doigt du vieux magicien. Le sorcier l'examine ensuite sous toutes les coutures, poussant quelques grognements d'approbation. Après cela, il fait un geste de la main, et les mâchoires se remettent en fonctionnement, mais pas le reste de son corps. S'ensuit alors une longue discussion, ou plutôt un interrogatoire. Quand le vivant est enfin satisfait, il claque sa langue, et mort s'effondre en morceaux, puis redevient poussière. Seuls quelques os isolés échappent au processus.

\ellipse

C'est un nécromant guilleret que vous retrouvez à la taverne. Il vous raconte sommairement son aventure, puis s'attarde sur son étude du vampire, en vous montrant le crâne qu'il a récupéré comme exemple.

« Ce qui est fascinant, c'est que le vampire aurait dû être mort. Tout avait été fait dans les règles. Le chasseur de vampires était un professionnel, il n'a pas oublié la pièce secrète, il a utilisé de l'argent pour le coup final, il a effacé tout ce qui ressemblait à un glyphe, et il a même mis le feu en partant au cas où.

Et pourtant, le vampire s'est régénéré.

Ou plutôt, je pense que quelque chose l'a régénéré. Il n'y a plus une seule once de pouvoir dans cet os, et pourtant, je parierai une décoction de malemort que si tu le gardes à l'ombre suffisamment longtemps, tu verras de la chair repousser dessus. Et je mets une deuxième fiole en jeu que si tu t'éloignes suffisamment de cette région oubliée, le processus s'arrêtera de lui-même. »

Vous soupesez les paroles de votre vieux compère.

« En résumé, tu penses que quelqu'un, ou quelque chose, permet aux morts-vivants d'échapper encore plus que d'habitude aux règles de la nature, les ramenant des limbes même quand ils auraient dû enfin trouver la paix éternelle ? »

Il vous répond avec un sourire entendu.

« Exactement. Il semble bien difficile de mourir définitivement ici. »

Voilà qui confirme vos suspicions. Notez le Code \emph{Éternel} et ajoutez le Crâne du Vampire [2] à vos Possessions.

Votre allié a également extrait des archives quelques documents qu'il vous suppose utiles.

Rendez-vous au \glink{6}.

\gsection{54}

Le chevalier n'est pas humain, cela est clair. Est-ce un spectre, un golem, un souvenir, une machine ? Difficile à dire. Son rôle est celui de rempart, de gardien, et il l'accomplit avec une grande efficacité. Aucune parole ne semble l'émouvoir, et il bloque tous les chemins qui pourraient permettre de progresser avec une surnaturelle célérité. Jusqu'à présent son épée est restée au fourreau, mais il ne fait pas de doute que sa grâce s'étend aussi au maniement des armes.

Savoir quand abandonner est une qualité rare, mais vous avez choisi quelqu'un la possédant. Cela lui permet de rentrer en vie pour vous relater son échec, et ces informations devraient vous aider à mieux préparer votre prochaine action.

Rendez-vous au \glink{50}.

\gsection{55}

C'est la voix du premier vampire. C'est la voix du premier tentateur. C'est une voix chargée de pouvoir par le dieu de la corruption lui-même. Elle ordonne, et les êtres obéissent.

Mais pas cette fois. Les choses ont changé. Les gens ont changé. Ce pays a changé. Et tout cela ne sera pas réduit à néant car un petit pisseur qui se croit supérieur en a décidé ainsi.

Il n'y a pas de miracle. Si l'hypnose se révèle faible, contestable, ce n'est probablement pas uniquement grâce à une force de volonté hors du commun. Au cours des quêtes successives, la source du pouvoir de l'être a dû être affectée, diminuée.

Mais qu'importe ! Le résultat est là. Dans un sursaut d'orgueil, la domination mentale est brisée, éclatée. Le démon millénaire est repoussé aussi aisément qu'un charlatan de bas étage.

Rendez-vous au \glink{41}.

\gsection{56 – La secte du renouveau}

Une vieille mine, allant sur sa fin, dont le filon presque épuisé est encore exploité par une poignée d'irréductibles. Un petit village à flanc de collines, pour les loger eux et leurs familles. Le tout desservi par une maigre route, bien tracée par le passage régulier de carrioles.

Voilà le décor de cette mission. La partie immergée tout du moins. Car la montagne est truffée de galeries, de tunnels, de passages, un labyrinthe de salles et de couloirs.

Les maigres informations que vous avez pu rassemblées indiquent que c'est dans ce dédale que la secte cache son sanctuaire, mais n'apportent aucune précision sur la manière d'y accéder, et interroger un des mineurs a été jugé trop peu discret, de peur que tous ne plient bagage en découvrant sa disparition.

Reste la bonne vieille méthode de la lanterne, de la corde et de l'explorateur hardi. Nombreux sont ceux qui se perdraient définitivement dans cet enchaînement sans fin de chemins apparemment identiques, mais pour un aventurier expérimenté, habitué à errer au fond de donjons biscornus, c'est un jeu d'enfant.

Bientôt la silhouette solitaire découvre la grande salle. Une pièce étroite, toute en longueur, s'étirant à perte de vue. Des rangées et des rangées de sièges s'alignent, régulièrement desservis par des ouvertures masquées de rideaux. Tout au bout se découpe un majestueux autel, face à un mur qui a été creusé en haut-relief pour représenter une divinité duale aux multiples paires de bras.

Sa moitié gauche est un squelette, tenant un sablier, un kriss, une faux, un linceul, sa partie droite un être humain hermaphrodite en pleine santé, portant une grappe de raisins, une plume, une coupe, la dernière main ouverte et tendue vers l'auditoire. Les proportions ont volontairement été sacrifiées au profit de l'impact de l’œuvre, impressionnante avec ses membres gigantesques qui sortent du mur jusqu'à effleurer les premiers sièges de l'assistance.

Si vous avez envoyé une personne dotée de la spécialité Roublardise, rendez-vous au \glink{82}.

Sinon, rendez-vous au \glink{29}.

\gsection{57}

Cela prend plusieurs jours. Plusieurs jours pendant lesquels votre mercenaire patrouille seul dans les bois la nuit, personne n'ayant été assez fou pour l'accompagner. Plusieurs jours à empêcher les attaques par seule présence. Et puis finalement, l'assaillant n'y tient plus, et se décide à éliminer la cause de ses ennuis. Au plus noir de la nuit, il passe à l'assaut. C'est une silhouette humaine qui se glisse souplement entre les arbres, mais c'est une centaine de kilos de muscles, de poils, de crocs et de griffes qui se jette sur sa cible d'un bond puissant. 

S'il attaque en ce moment le \hero{Chasseur}, le \hero{Mage}, l'homme à la \hero{Main d'Acier}, la \hero{Légende}, ou simplement un combattant équipé d'une lame en argent, rendez-vous au \glink{101}.

Sinon, rendez-vous au \glink{33}.

\gsection{58}

L'ouvrage maudit contient sans doute une sorcellerie capable de vous aider à retrouver vos souvenirs perdus, mais vous savez qu'il y aura un prix à payer. Vous respirez profondément, vous détendez au maximum, et enfin vous l'ouvrez.

La mémoire occupe tout un chapitre, qui détaille à quel point il est facile de la falsifier, aussi bien pour une volonté extérieure que par son propriétaire, souvent inconsciemment, pour se protéger des horreurs qu'il a connu mais voudrait oublier. Cependant, l'esprit humain est ainsi qu'il n'oublie jamais vraiment, au mieux range-t-il cela dans un coin secret. Mais la connaissance est toujours là.

Vous trouvez alors ce que vous cherchez. Une courte incantation, qui a le pouvoir de rompre les digues, de libérer ce qui avait été emprisonné. Toutefois la magie ne fait pas de distinction et libère d'un coup toutes les mémoires enfouies, même celles propres à faire perdre toute raison.

Vous savez que cet avertissement n'est pas à prendre à la légère, mais vous estimez que c'est un risque qui vaut la peine d'être pris. Aussi prononcez-vous les paroles maudites.

Des millions de souvenirs submergent aussitôt votre esprit. Des petites choses, insignifiantes, qui avaient été reléguées dans les abysses de votre cerveau. Elles reviennent toutes à votre conscience, désireuses d'exister à nouveau. Vous vous concentrez de toutes vos forces pour les renvoyer d'où elles viennent, cherchant les événements réellement importants au milieu du chaos.

Et là vous trouvez. Vous revoyez tous vos alliés, vous racontant avec fougue leurs réussites, ou parfois leurs échecs. Certains de  ces événements sont vieux, très vieux, parlent de menaces depuis longtemps écartées. Mais au milieu de tout cela, vous découvrez un élément que vous n'aviez aucune raison d'oublier, une piste effacée avec le plus grand soin. Vous vous y accrochez comme un naufragé au dernier débris flottant encore.

Vous puisez dans toutes vos ressources mentales pour extraire l'information de la masse et l'afficher au grand jour. Vous finissez pantelant, chancelant, mais satisfait. Vous savez ce qu'on a voulu vous cacher.

La Quête \textbf{Le marais interdit} (85) est de nouveau disponible. Notez également le Code \emph{Mémoire} et ajoutez la \emph{Formule du Souvenir} à vos Informations.

Rendez-vous au \glink{93}.

\gsection{59}

Vous passez une nuit agitée. Vous rêvez, au plutôt, vous cauchemardez. Dans vos songes, vous n'êtes qu'un pantin, manipulé par les appendices d'êtres que les emphases les plus exagérées ne suffisent pas à décrire. Ils jouent avec votre vie, avec votre quête, avec votre esprit, comme des bambins s'amuseraient avec une poupée.

Vous vous réveillez épuisé, las de corps et d'esprit. De plus, vous semblez avoir oublié quelque chose d'important, mais quoi ?

La Quête \textbf{Le marais interdit} est inaccessible jusqu'à nouvel ordre. Comportez-vous comme si vous ne l'aviez jamais débloquée. Si la réussite d'une autre quête devait vous la redonner, ajoutez-la normalement.

Notez le Code \emph{Oubli}.

Rendez-vous au \glink{93}.

\gsection{60 – Au cœur du marais}

Si la personne que vous avez envoyée pour cette mission dispose de l'Information \emph{Le chemin du marais} ou s'il s'agit soit de la \hero{Légende}, soit du \hero{Métamorphe}, rendez-vous au \glink{86}.

Sinon, rendez-vous au \glink{33}.

\gsection{61}

L'objet que vous manipulez a une aura étrange, roublarde, qui n'incite pas à le conserver. Vous ne sauriez dire si c'est une volonté explicite  de ses créateurs ou simplement leur façon d'être.

Mais de quoi s'agit-il exactement ? Choisissez celui que vous désirez parmi les trois suivants :

\begin{description}
\item[La Couronne des Fées] Tressée de fleurs qui jamais ne fanent, elle protège son porteur de toute magie qui lui voudrait du mal. [15]
\item[L'Anneau du Leprechaun] Ceux bénis par les Leprechaun ne connaissent plus la pauvreté financière. L'anneau les guidera naturellement vers l'or, l'argent, les gemmes. Tant que vous possédez (vous spécifiquement, et pas un aventurier quelconque) cet objet, augmentez automatiquement votre pécule d'une pièce d'or au début de chaque phase de recrutement. [100]
\item[La Branche de l'Arbre-Vie] Bien qu'elle soit détachée de son arbre nourricier, la branche continue de vivre et de bourgeonner. Greffée sur une plante mourante, elle lui rendra force et vigueur. Elle dégage une impression de vitalité extrême, de sacré primordial. [1]
\end{description}

Retournez ensuite au \glink{20} si vous avez le Code Féerie et rendez-vous au \glink{50} sinon.

\gsection{62}

Quel que soit le contexte, la créature est impressionnante. Elle semble apte à dévorer son corps d'une bouchée, mais aussi à attraper son âme dans ses filets mentaux, pour ensuite la digérer lentement durant toute l'éternité. S'il venait à périr maintenant, même son ascendance particulière ne lui permettrait de s'en tirer sans dommage. Pour la première fois depuis des siècles, il est confronté directement au risque de la mort, de l'oubli éternel.

Cela l'excite terriblement.

À vaincre sans péril, on triomphe sans gloire, et depuis bien trop longtemps, le péril est pour lui inexistant. Mais aujourd'hui les choses sont différentes, et il sent son sang bouillir dans ses veines alors que la perspective du combat se fait plus précise d'instant en instant. Un monstre gargantuesque, terrible, en apparence invulnérable, se dresse devant lui. Il ne dispose pour le battre que d'un équipement dépassé, de sa force et de sa ruse. Comme au bon vieux temps.

Poussant un cri de guerre sauvage, il dégaine sa vieille lame et se rue sur la bête. En réaction, une nuée de tentacules fondent sur lui. Il se jette à terre pour les éviter, finissant son mouvement par une roulade qui le couvre de boue. Ce faisant, il sent sur sa langue le goût de la corruption qui imprègne la terre même du lieu quand quelques gouttes franchissent ses lèvres.

Il se relève d'un bond en crachant de dégoût, esquivant au même moment un deuxième assaut. D'un revers de sa lame, il entaille un des lourds appendices, qui se met à dégouliner d'un fluide nauséabond qui doit être du sang pour cette chose. Sa main libre se referme sur le rebord de la plaie, et d'un mouvement spectaculaire, il tire de toutes ses forces, déchirant le tentacule sur la moitié de sa longueur. Des entrailles du monstre s'élève alors un bruit de tonnerre, qui pourrait être un cri de douleur, de rage, ou même une tentative de communication, d'injure.

Mais pour un membre mutilé, il en reste cent autres. Il se surprend toutefois à sourire en s'apercevant que la bête n'a pas les particularités les plus insupportables qu'il ait déjà rencontré : ni protection surnaturelle quelconque la rendant vulnérable seulement à une stratégie particulière, ni régénération extraordinaire où  deux pseudopodes repoussent quand un a été abîmé. Par certains aspects, ce ne sera donc pas le combat le plus irritant qu'il ait jamais effectué. Mais ce sera sans doute le plus long et le plus difficile.

Si un sculpteur, un peintre, un poète ou un autre artiste était présent, il ne manquerait pas de scènes à immortaliser. Le héros encerclé, tranchant nette en une seule attaque plusieurs piliers de chair pour se libérer avant que ceux-ci ne l'écrasent. Ce même personnage, couvert de boue et de fluides, son arme perdue dans la lutte, continuant à se battre uniquement à la force de ses poings, au milieu d'une arrivée sans fin de bras visqueux cherchant l'écrabouiller. Ou même cet instant saugrenu, où il s'arrange pour que l'ennemi se frappe lui-même en esquivant au dernier moment une attaque croisée.

Après une très longue durée, pendant laquelle un être humain normal exposé à un tel effort se serait déjà effondré de fatigue une bonne dizaine de fois, le dernier attaquant est repoussé. Au milieu du champ de bataille soudain étrangement calme, une silhouette humaine, haletante, sale, blessée, contemple la créature du fond des âges. Celle-ci s'efforce de modifier son physique, de se métamorphoser. De sa surface molle, des embryons de nouveaux membres, couverts d'écailles, sans doute plus résistants, plus adaptés à l'affrontement, commencent à apparaître. Jusqu'à présent, elle n'avait probablement jamais pensé pouvoir perdre sur le plan d'un combat purement physique, tant les différences de taille, d'endurance, de force, jouaient en sa faveur. Maintenant que l'expérience lui a donné tort, elle cherche à déjà à combler ce manque.

L'idée saugrenue de lui laisser finir son évolution traverse l'esprit du héros légendaire, et s'il était le seul impliqué dans cette affaire, il lui aurait sans doute laissé cette chance. Mais ici et maintenant, il a une mission à accomplir, et ne peut malheureusement pas prendre le risque de l'échouer pour satisfaire un caprice personnel.

Réunissant tout ce qui lui reste d'énergie, il se jette sur le corps du monstre, écartant sans mal ses dernières défenses. Il se creuse un chemin dans l'horrible intérieur de la créature à l'aide de ses ongles, de ses doigts, à la recherche d'une partie sensible, centre nerveux, cœur, dont la destruction lui assurerait la victoire.

Et il trouve. Dans un ultime éruption de mucus, il transperce un organe vital. Privée d'énergie, la bête s'effondre sur lui, se brisant en une pluie de matières organiques.

\ellipse

En vous réveillant ce matin, vous saviez déjà que votre protégé avait réussi sa mission. La différence dans l'atmosphère était sensible, sans qu'il soit réellement possible de la décrire par des mots. C'est un peu comme si un bruit pénible mais constant auquel vous aviez fini par vous habituer s'était enfin tu : maintenant qu'il n'est plus là, vous percevez la différence positive.

Bien des questions restent cependant sans réponse. Le danger étant écarté, vous vous rendez vous-même sur place quelques jours plus tard, mais n'y découvrez que les innombrables restes éparpillés et déjà en train d'être recyclés par la nature d'une créature dont la taille est encore supérieure à ce que vous imaginiez à partir des descriptions que vous en aviez eu. Impossible toutefois de reconstituer le puzzle pour savoir à quoi elle pouvait bien ressembler.

De votre héros, vous trouvez pour toute trace un simple caillou, sur lequel a été hâtivement gravé un triangle penché, suivi de trois bâtonnets. Vous apprendrez plus tard que cela correspond à un chiffre, dans une écriture qui n'est plus utilisée depuis longtemps : 13.

Rendez-vous au \glink{135}.

\gsection{63}

Ce n'est pas un vieux singe que l'on apprend à faire des grimaces, et dans le cas présent, vous affrontiez un singe d'un âge plus que vénérable.

Non seulement, le non-mort a réussi à s'échapper, trompant votre vigilance et celle de l'Ordre, mais il n'est pas parti sans vous laisser quelques surprises. Lorsque les gardes chargés de le surveiller ont déferlé sur l'auberge où vous logiez sous la forme de zombies, vous saviez que l'histoire avait mal tourné.

Que des attaques similaires aient eu lieu au même moment sur les principaux chefs-lieux des différentes factions n'a pas arrangé les choses. D'autant que votre nouvel adversaire s'est amusé à respecter la mixité sociale dans les cadavres choisis. Vous avez ainsi découvert l'existence de loups-garous zombis, de fées vampires, de liches sorcières.

Le bilan de ces assauts, effectués lors de l'euphorie de la complétion du sceau, a été lourd, et chaque camp rejette la faute sur les autres, mais surtout sur vous. Du véritable responsable, pas de trace.

Diminuez votre Réputation de 10 pour cet ultime coup du sort.

Rendez-vous au \glink{135}.

\gsection{64}

Les langues se délient peu à peu. Les habitants se montrent toujours réservés, distants, mais communiquent parfois sur quelques points intéressants. Deux rumeurs intéressantes remontent ainsi jusqu'à vous :

\quest{La secte du renouveau}{56}{Roublardise ou Combat (\ankh, \cross)}{Difficile}{
Des adorateurs d'une noire divinité se terreraient dans la montagne, cachés dans d'anciennes galeries de mines. Quelqu'un devrait y aller et les en chasser !
}

\quest{Le cercle de la paix}{37}{Recherche (\caduceus)}{Moyenne}{
Il existe une partie de la forêt que l'on dit depuis toujours magique, terre de lutins, de gnomes et de fées. Bien qu'il n'y ait aucun témoin direct, seulement des « on dit », cela peut valoir la peine d'être vérifié.
}

Rendez-vous au \glink{50}.

\gsection{65}

Le vampire rate sa première attaque. A-t-il manqué de célérité ou son adversaire a-t-il eu des réflexes de félin, difficile à dire. Mais il a maintenant perdu l'initiative, et il lui faut quelques secondes pour retrouver son équilibre après ce fougueux assaut. Un temps suffisant pour ramasser le pieu à pointe d'argent. Deux cris bestiaux se croisent tandis que vivant et mort se chargent mutuellement.

Et c'est la vie qui l'emporte d'un cheveu, la lance improvisée transperçant de nouveau le cœur de la bête avant que ses crocs ne se referment. Et le trois fois mort redevient encore une fois poussière.

\ellipse

Vous examinez l'ossement qui vous a été ramené, mais il n'a rien de particulier. C'est un simple crâne humain, sans le maxillaire inférieur, aux canines certes plus longues que la normale, mais qui ne dégage aucune aura particulière. Vous ne mettez cependant pas en doute le récit qui dit qu'il était il y a encore peu un élément d'un vampire bien coriace.

Ajoutez le Crâne du Vampire [2] à vos Possessions.

En sus de ce surplus de calcium, vous avez reçu toute une pile de documents potentiellement intéressants, extraits des archives ayant échappées aux flammes.

Rendez-vous au \glink{6}.

\gsection{66}

Vous ouvrez la boîte et... Rien. Elle est parfaitement vide. Vous ne ressentez aucune différence entre l'avant et l'après. Ah si, de le déverouiller semble avoir rappelé au coffret son âge : il tombe en morceaux. Retirez-le de vos Possessions.

Mais bien que la différence ne soit pas encore perceptible, de nouvelles forces se sont mises en mouvement. La \hero{Légende} est dorénavant disponible.

Notez le Code \emph{Espérance}.

Retournez au 50.

\gsection{67 – L'ordre des tueurs de monstres}

L'Ordre n'est pas directement basé en Ylèdre, mais juste de l'autre côté de la frontière, dans une ville nettement plus tranquille. Son quartier général n'est pas difficile à trouver : c'est une petite forteresse isolée aux abords de la bourgade, pauvre en fenêtres mais entourée d'une imposante enceinte. Un bâtiment dépouillé, spartiate, dont l'unique entrée est défendue par des gardes à son image, ascétiques et durs.

Si vous avez déjà accompli la quête \textbf{La secte du renouveau}, rendez-vous immédiatement au \glink{128}.

Si vous possédez le Code \emph{Sabbat} et qu'il est associé à la personne présente, rendez-vous sur le champ au \glink{73}.

Si vous avez choisi quelqu'un disposant de la Spécialité Diplomatie ou le \hero{Croisé}, et lui avez de plus confié le Crâne du Vampire, rendez-vous au \glink{90}.

Si c'est plutôt la Roublardise qui fait sa force, rendez-vous au \glink{123}.

Sinon, vous allez devoir calculer votre réputation auprès de cette organisation. Prenez votre score de Réputation traditionnel, et appliquez-lui les modificateurs suivants :

\begin{itemize}
\item Si vous avez envoyé le \hero{Croisé} ou que la personne que vous avez choisie porte le Pendentif sacré, ajoutez 2.
\item Ajoutez 1 vous avez désigné quelqu'un maîtrisant la Diplomatie.
\item Retirez 1 pour chacun des Codes suivants : Féerie, Sabbat, Ombre.
\end{itemize}

Si le total ainsi obtenu est supérieur ou égal à 6, rendez-vous au \glink{78}.

S'il est inférieur, rendez-vous au \glink{97}.

\gsection{68}

Et la Baba Yaga écoute le discours qui lui est servi. Elle comprend rapidement que son aide est demandée, indirectement, par un gueux qui n'a même pas pris la peine de se déplacer lui-même, et surtout sans contrepartie valable. Et si la Baba Yaga est connue pour bien des choses, ce n'est ni pour sa générosité ni pour sa patience et encore moins pour son bon caractère.

La porte de la chaumière se referme d'elle-même. L'insecte parlant comprend l'échec de sa mission, mais pense encore pouvoir sauver sa vie. Erreur, erreur. Son arme misérable se brise comme un fétu de paille sur les os de la vieille femme dont la bouche s'ouvre, grand, si grand...

Rendez-vous au \glink{33}.

\gsection{69}

Nombre de héros se sont par le passé opposé aux forces des ténèbres en ces lieux. Chasseurs de vampires, traqueurs de loup-garous, exorcistes, tueurs de monstres ont combattu pied à pied les horreurs cachées dans les bois, les caves, les chaumières.

Parmi eux se trouvait un légendaire chevalier, aussi charitable et doux avec les innocents que terrible et destructeur face aux créatures du mal. Déjà mythique de son vivant, sa disparition mystérieuse ne fit que raffermir son aura.

Et voici que la clé de cette énigme remonte des profondeurs. Elle parle d'un long affrontement contre un démon, d'un combat acharné se déroulant sur plusieurs années, l'adversaire fuyant régulièrement aussi loin que ses pouvoirs ne lui permettaient, mais sans jamais réellement semer le champion de la justice. Enfin le duel final eut-il lieu ici même, sur cette terre maudite, là où l'être des enfers était au sommet de sa puissance.

De cet affrontement ne devait sortir aucun vainqueur. Car si le chevalier l'emporta, ses blessures étaient telles qu'il ne pouvait espérer survivre. Avec ses dernières forces, il saisit son épée, la Sainte Lame, une arme aussi extraordinaire que son porteur, et la planta dans un rocher. De la pierre il l'avait en son temps arrachée, et de nouveau ne pourrait l'extraire que de celui ou celle qui en serait digne.

Les années ont passé, le marais a changé, la végétation a poussé, mais l'épée est toujours là. Son pommeau, crasseux mais intact, sort à peine du sol, invisible aux yeux de ceux qui ne la cherchent pas.

Aujourd'hui toutefois deux pupilles sont braquées sur elle, et bientôt deux mains se referment sur son manche. Et jaillit sans effort une merveille de forge, aussi légère qu'une plume, aussi lumineuse que le soleil, aussi tranchante qu'un rasoir. De la puissance brute s'écoule de l'objet inerte à l'être de chair et de sang qui le tient, la force d'anéantir des démons immortels, des malédictions éternelles. Ce raz-de-marée de pouvoir est sans pitié pour l'organisme qu'il traverse, effaçant le faible esprit humain qu'il rencontre pour le remplacer par une volonté plus efficace.

Notez le Code \emph{Folie}.

Rendez-vous au \glink{33}.

\gsection{70}

Qu'est-ce qui pousse la plupart des vivants à rechercher à tout prix le pouvoir ? Qu'es-ce que qui les fait s'enfoncer dans les profondeurs de temples oubliés à la recherche de fragments d'antiques artefacts ? Qu'est-ce qui les incite à conclure des pactes qu'ils savent vérolés avec des êtres qui les dépassent ? Qu'est-ce qui les convainc de se battre jour après jour pour une minuscule fragment, une infime poussière, de l'infinie énergie que contient l'univers ?

Pour certains le pouvoir lui-même est une drogue. Pour d'autres une obligation pour accomplir leur véritable objectif. Mais il en existe aussi pour qui cela n'est qu'un jeu. Le but final n'a pas d'importance. Leur seul but est de passer à l'étape de suivante, de toujours progresser. Minimiser ses désavantages, maximiser ses atouts, trouver l'ordre idéal pour abattre les difficultés une par une, utiliser les crocs de l'ennemi vaincu un instant plus tôt pour triompher du suivant, découvrir le chemin optimal vers toujours plus de grandeur, mais aussi toujours plus de challenge.

Et aujourd'hui, une nouvelle frontière s'ouvre à l'un de ces aventuriers fous. Aujourd'hui, il doit tuer un dieu. Un embryon de dieu, un être réduit à une fraction de ce qu'il pourrait être, mais un dieu tout de même. Et aux hormones de la terreur se mélangent celles de l'effort, de la colère et du plaisir.
C'est une forme humaine qui est entrée dans la clairière. C'est une bête sauvage qui se rue sur son adversaire. Armurée d'écailles ou de poils, armée de griffes ou de poison, une chimère d'humain et d'animal.

L'affrontement est violent, sanglant. Un combat entre deux êtres métamorphes, constamment changeant, décidés à se battre jusqu'à leur dernier souffle. Il n'implique nul spectacle extraordinaire, nulle pyrotechnie, juste deux créatures vivantes qui se frappent, se blessent.

Enfin, l'une des deux tombe, ses dernières forces épuisées. Et alors le silence se fait, seulement entrecoupé par les halètements de celle qui a survécu.

\ellipse

C'est la Guerre elle-même qui rentre dans l'auberge. Couverte de boue, de sang, de morceaux de chair, traînant derrière elle une masse purulente qui pourrait être le cœur de la bête qui hante vos pires cauchemars. Les autres clients reculent précipitamment, une terreur religieuse dans leurs yeux. Vous-même, qui avez tout vu ou presque, vous faites des efforts pour vous contenir. Et le plus terrifiant dans tout cela, c'est que sous cette couche de crasse et de mort transparaît un sourire satisfait. Un sourire humain.

Notez le Code \emph{Évolution}.

Rendez-vous au \glink{135}.

\gsection{71}

Le claquement sec d'une corde d'arbalète retentit. En un éclair, un carreau a été projeté vers un buisson apparemment innocent, s'enfonçant dans la verdure pour ne pas en ressortir. Le tireur recharge déjà son arme quand une créature en jaillit soudain, courant ventre à terre vers les profondeurs des bois, le sang s'écoulant de sa fraîche blessure.

Il serait logique de penser que la bête quadrupède, taillée pour la forêt, serait plus rapide que l'homme dans son milieu naturel. Mais pour le chasseur, cette traque est son propre milieu naturel. Suivant l'odeur du sang comme un chien, il se glisse telle une ombre entre les arbres, ses pieds touchant à peine le sol dans un mouvement fluide et continu. Ce n'est qu'une tâche noir parmi les autres, un éclair d'obscurité qui gagne rapidement du terrain sur sa proie, esquivant les obstacles avec aisance. Là où une imperfection du sol ralentit l'animal quelques instants, il bondit par-dessus et l'ignore. Là où un arbre oblige à un détour, il se balance de branche en branche. C'est un spectacle terrifiant que de voir un être aussi expert, et qui semble prendre un tel plaisir, à rattraper, acculer, neutraliser.

Après une intense course-poursuite, il arrive à ses fins. Une lame d'argent déchire le flanc de la bête alors que celle-ci pensait avoir enfin semé son poursuivant. Elle réussit toutefois à de se dégager avant le deuxième coup, droit à la gorge, et entreprend une complexe métamorphose alors qu'elle s'éloigne. Le poil régresse, la mâchoire, le museaux, les oreilles, se rétractent, le torse, les membres s'épaississent. En une poignée de secondes un loup, un loup ordinaire, pas plus grand ou plus monstrueux que la normale, est devenu un homme. Une main comprimant sa large entaille dans l'espoir de maintenir le sang à l'intérieur, il tend l'autre vers son adversaire en un geste implorant.

« Pitié ! Ce n'est pas moi qui ai tué tous ces gens ! Je cherche moi aussi à découvrir le coupable ! »

Sans une once de compassion sur son visage, le chasseur relève l'homme à la force d'un seul bras, et plaque son couteau sur sa gorge.

« Vous m'espionniez. » 

C'est une constatation, un fait exprimé avec une froideur, un détachement émotionnel, bien plus effrayant qu'une saine colère. Le lapin dans son collet ne s'y méprend pas et continue à parler sans s'arrêter, conscient que la mort l'attend dussé-je sa langue fourcher un instant.

« Je cherchais à déterminer dans quel camp vous étiez ! Je ne vous voulais pas de mal. Je vous le jure, je ne suis pas responsable de ce qui s'est passé ici. Au contraire, je veux arrêter ces horribles meurtres. Comme vous j'en suis sûr. Je m'excuse d'avoir pu osé douter de vous. Il est évident que vous êtes dans le camp des gentils. »

Un sourire révélant une dentition de carnassier s'affiche sur le visage du chasseur. Sa dague danse, dessinant une légère coupure.

« Je vois que vous désirez un peu plus d'explications. Cela risque d'être un peu long, mais je suppose qu'il est inutile de proposer que nous nous asseyions... » 

Une bonne heure plus tard, l'homme qui est aussi un loup panse ses blessures au milieu du bois, abandonné sans remords par son agresseur. Il est toutefois encore en vie, preuve s'il en est qu'il a convaincu ce fou dangereux de sa bonne foi. Mais pour cela, il a dû lui en dire beaucoup plus qu'il ne le voulait.

Notez le Code \emph{Révélation}.

Rendez-vous au \glink{57}.

\gsection{72}

\success

« En ce temps-là, le monde était jeune, sauvage, changeant. Le pouvoir était partout, dans l'air, dans l'eau, dans la terre. C'était un temps de miracles, où l'impossible n'existait pas, où les dieux marchaient parmi les hommes.

À cette époque, je n'avais ni mortier, ni pilon, ni chaumière, et je voyageais à travers le monde prélevant mon tribu de chair fraîche dans chaque tribu que je visitais.

Un jour que j'errais, je sentis une puissante perturbation dans l'ordre des choses. Une explosion de malveillance, de malice, de perversion, de haine, un maelström de tout ce qu'il y a de pire chez les êtres pensants. Évidemment, je me rendis sur le champ à sa source. Et je découvris cette contrée non pas telle que je l'avais connue auparavant, banale, sans attrait, mais telle que vous la voyez aujourd'hui, peuplée de tous les fantasmes les plus plus malsains de l'humanité.

Et lorsque l'Âge d'Or arriva à sa fin, que le pouvoir commença à se raréfier, que le monde devenait place d'ordre, de logique et d'ennui, ces terres restèrent les mêmes, si riches, si intéressantes. Je vins alors m'installer ici, et je cherchais alors à comprendre le pourquoi et le comment. Et je compris, oh oui, je compris.

Montez au sommet de la plus haute montagne, vous y trouverez un trou. Entrez dans ce trou, vous y verrez un escalier. Descendez cet escalier, jusqu'au bout, marche après marche, jusqu'à ce que vos jambes soient dures comme la pierre, votre souffle rauque, votre torche consumée. Et alors vous trouverez toutes les réponses que vous cherchez, même celles que vous auriez préféré ignorer. »

Telle est l'histoire qui vous a été rapportée, le discours de la Baba Yaga. Il soulève plus d'interrogations qu'il n'amène de solutions, mais au moins disposez-vous d'une nouvelle piste :

\clearpage

\quest{Les cavernes du début du monde}{51}{Recherche}{Difficile}{
Malgré des indications plus que floues, vous avez réussi à identifier la fameuse grotte qui pourrait contenir les réponses à vos questions. Ne reste plus qu'à l'explorer. Une mission simple en apparence, et pourtant vous avez des sueurs froides rien que d'y penser. Bien que vous n'ayez aucune base solide pour le supposer, vous êtes persuadé que ce qui cache au fond de ces cavernes est dangereux. Extrêmement dangereux même.
}

Le \hero{Mage} est dorénavant disponible.

Si vous avez le Code \emph{Mille}, rendez-vous au \glink{133}.

Sinon, rendez-vous au \glink{50}.

\gsection{73}

L'Ordre commet parfois des erreurs de jugement et en est conscient, mais parfois l'odeur du soufre est trop forte pour être simplement une coïncidence. Vous avez envoyé pour cette mission quelqu'un ayant pactisé avec des puissances obscures, et ils l'ont bien compris.

Le nombre exact de membres de cette organisation est assez dur à estimer, car elle emploie des auxiliaires, des informateurs. Toutefois, ce sont deux douzaines des leurs qui tombent comme la foudre sur votre émissaire.

Ce qui se passe ensuite n'est pas beau à voir. Disons simplement que l'Ordre a un protocole très strict pour ce genre de cas, qui implique de mystérieuses chambres souterraines insonorisées.

Après ce coup d'éclat, vous découvrirez rapidement que votre nom été inscrit dans leur livre noir, ce qui a peu de chance d'être à votre avantage.

Celui ou celle que vous aviez envoyé ici devient définitivement indisponible. De même cette Quête n'est plus accessible. Enfin, diminuez votre Réputation de 3 et notez le Code \emph{Paria}.

Rendez-vous au \glink{50}.

\gsection{74}

Les cauchemars continuent, toujours plus sombres, toujours plus féroces. Vous vous réveillez au milieu de la nuit, en nage, avec l'impression que des morceaux de votre esprit ont été arrachés, déchiquetés.

Retirez à nouveau la Quête \textbf{Le marais interdit} des quêtes disponibles.

Vous êtes persuadé d'avoir encore oublié quelque chose, quelque chose d'important. Et cette fois, vous n'attribuez pas ce trou de mémoire à votre grand âge. Quelqu'un s'efforce de faire disparaître des informations cruciales, allant jusqu'à les effacer dans l'esprit des témoins.

Si vous possédez le Tome Scellé, rendez-vous au \glink{58}. Sinon, rendez-vous au \glink{49}.

\gsection{75}

La mission se déroule sans événement particulier, sans rien sortant de l'ordinaire. Seuls des habitants normaux, ou du moins, qui ont l'apparence de la normalité, sont rencontrés. Ils ont tous leurs petits tracas mais rien de réellement marquant ou important.

Si la personne qui accomplit cette mission a la Spécialité Diplomatie, augmentez tout de même votre Réputation de 1 car ses habiles discours contribuent à votre bonne image.

Si au contraire, vous avez envoyé le Mage, l'Homme à la Main d'Acier, le Mnémonique, le Chasseur ou le Métamorphe, diminuez-la de 1. L'aura singulière, pour ne pas dire effrayante, de votre missionnaire ne contribue en effet pas vraiment à rassurer les locaux rencontrés.

Dans le cas où vous obéissez aux deux conditions contraires à la fois, leurs conséquences s'annulent mutuellement et votre réputation reste inchangée.

Si votre Réputation est maintenant de 7 ou plus, rendez-vous au \glink{96}.

Si elle est comprise entre 3 et 6 inclus, rendez-vous au \glink{64}.

Si elle est inférieure, rendez-vous au \glink{110}.

\gsection{76}

Vous êtes allé en personne effacer le dernier symbole retenant le non-mort. Le fait que votre malédiction vous ait laissé le faire semble indiquer que ce geste est purement symbolique, et que l'être aurait pu s'enfuir sans votre aide. Vous lui avez également apporté des vêtements, qu'il enfile avec un plaisir manifeste. Il vous observe ensuite avec attention, en souriant :

« Amusant, j'étais persuadé que vous essaieriez de me planter une poignard en argent dans le dos dès que vous n'auriez plus besoin de moi. »

« Amusant, j'étais persuadé que vous chercheriez à boire mon sang dès qu'aucune barrière ne vous en empêcherait plus. »

« Bonne réponse. Apparemment, nous sommes entre personnes raisonnables. Puisque vous avez été honnête avec moi, je ne vais pas faire d'esclandre et me contenter de m'en aller gentiment. Cependant, je ne me fais pas d'illusions. Nous sommes dans des camps opposés, et si les circonstances nous ont alliés, la prochaine fois que nous croiserons, cela risque d'être sur le champ de bataille. Par pions interposés bien sûr, car nous sommes tous deux de la race des stratèges, pas des premières lignes. »

Et sur ces paroles, il s'en va, la tête haute. Comme si vous aviez besoin de vous créer de nouveaux ennemis !

Notez le Code \emph{Némésis}.

Rendez-vous au \glink{135}.

\gsection{77}

La valeur d'une ressource naturelle est directement liée à sa rareté. Dans le désert brûlant, l'eau vaut tous les trésors du monde, mais dans une région aussi humide que celle-ci, elle n'a pas plus de valeur que l'air qui s'y respire. Pour la civilisation qui vivait en ces lieux voilà bien des siècles, c'était l'or la denrée commune, banale, vulgaire. Tous les monuments, les statues, même les murailles, en étaient tapissés. Ils vénéraient le soleil, et leur cité-état reflétait sa splendeur dans tous les sens du terme.

Puis les fléaux arrivèrent. La terre trembla, ravageant leur pays. Une vapeur noire, une force novice, s'échappa des fissures, transportant avec elle la peste, le choléra, la tuberculose, la malaria. Le climat devint ensuite fou. Les derniers survivants s'enfuirent lors du trentième jour de tempête, alors que la ville déchirée par les vents et noyée sous la pluie s'affaissait dans un tourbillon de boue.

Cette capitale se dressait ici même. Et sous la masse stagnante, sous la terre humide, se trouve encore l'or.

Une telle réflexion mérite une réponse immédiate, et le sol cède peu à peu sous les vigoureux coups de pelle. L'endroit est choisi en se basant sur les bribes de souvenirs lointains qui continuent d'affluer, et ceux-ci se révèlent exacts quand après plusieurs heures d'effort, un fragment du métal jaune apparaît.
Il ne s'agit pas d'un marais. Il s'agit d'une mine d'or à ciel ouvert ! L'explorateur perd soudain ses derniers zestes de raison, et se met à attaquer furieusement les alentours à la recherche du précieux métal. Plus, il lui en faut toujours plus.

Si votre envoyé porte l'Anneau du Leprechaun, rendez-vous au \glink{38}.

Sinon, rendez-vous au \glink{33}.

\gsection{78}

Les méthodes de l'Ordre sont fort discutables. Leur fermeture d'esprit, leur obsession, leur hantise de tout ce qui est différent, n'aident pas à les rendre sympathiques. Mais leur but final est assez proche du vôtre, et ils commencent à s'en rendre compte. Aussi, même si ce n'est pas de gaieté de cœur, ils se sont décidés à faire une entorse à leurs infranchissables protocoles de sécurité pour permettre à votre émissaire de rencontrer directement un de leurs dirigeants, en tête-à-tête.

Ce qui n'empêche pas le prudent chef ainsi désigné de conserver à portée d'une main durant toute la discussion une clochette dont le tintement suffirait sans doute à rameuter toute la garde en un instant, et d'avoir l'autre sur le manche de son arme. Toutefois, il parle, et c'est là l'essentiel

Rendez-vous au \glink{131}.

\gsection{79}

Si vous avez le Code \emph{Espérance}, rendez-vous au \glink{9}.

Sinon, mais que vous possédez la Couronne des Fées, rendez-vous au \glink{35}.

Si ce n'est pas le cas, mais que vous disposez du Code \emph{Réminiscence}, rendez-vous au \glink{15}.

Si vous avez juste le Code \emph{Oubli}, rendez-vous au \glink{74}.

Dans tous les autres cas, rendez-vous au \glink{59}.

\gsection{80}

\success

C'est un duel dans les règles qui s'amorce. Le chevalier est aussi vif que silencieux, aussi leste que gracieux. Son armure dévie avec la même aisance l'acier que la sorcellerie. Mais son adversaire n'est pas en reste. C'est un enchaînement de passes, d'attaques, d'esquives, de parades, de feintes. Certaines sont subtiles, d'autres grossières. Des techniques exemplaires directement issues des manuels côtoient de pures improvisations à l'efficacité douteuse.

En tout honneur, c'est un combat remarquable, et il est suivi avec passion par de multiples spectateurs, cachés dans les buissons, les feuillages, les terriers.

Longtemps l'argenté semble dominer, supérieur en endurance, en calme et en maîtrise. Et pourtant, en un instant, en un assaut risqué, fou, son adversaire transperce ses défenses et le jette à terre. Sitôt vaincu, il se disperse comme la brume au petit matin.

Un silence parfait, oppressant, s'installe alors, comme si toute la forêt retenait son souffle. Et onze nouveaux guerriers apparaissent alors, surgissant du néant comme portés par le vent. Ils forment un arc de cercle, bloquant toute progression. La clé de voûte de la formation tient un objet dans ses mains ouvertes, mais ne fait pas mine de vouloir débloquer le passage pour autant. Après un nouveau silence pesant, le trophée est accepté et un demi-tour effectué.

Notez le Code \emph{Ante}.

\ellipse

Cette visite dans la forêt a amené plus de questions que de réponses. Au moins y avez-vous reçu une forme d'aide, sous l'apparence de ce mystérieux objet offert en récompense à votre missionnaire, qui vous l'a transmis avec presque du soulagement.

Rendez-vous au \glink{61} pour savoir de quoi il s'agit.

\gsection{81}

Une fouille en profondeur n'ayant rien révélé d'autre que de la crasse et de la vermine, ce sont les archives précédemment découvertes qui sont mises à sac. Bientôt, le sac est rempli de documents et entreprend la descente de la colline en direction de son commanditaire.

Rendez-vous au \glink{6}.

\gsection{82}

L'ombre remonte silencieusement l'allée, attentive au moindre bruit qui pourrait la trahir. Maintenant qu'elle est arrivée jusqu'ici, elle hésite. Trop de galeries, trop de lumières, trop d'angles morts, pas de cachettes évidentes...

Une longue indécision étant le plus grand des dangers, elle finit par tenter une opération absurdement audacieuse, et se dissimule derrière la statue, dans l'étroit espace entre le relief et le mur.

Et chance sourit aux fous. Bientôt, deux acolytes encapuchonnés viennent nettoyer et préparer l'autel. Quand ils repartent une fois leur travail accompli, ils sont suivis.

Grâce à ces guides involontaires, la carte de l'endroit commence à prendre forme. L'emplacement des dortoirs, des communs, de la cuisine se fixe. Bien sûr, ces lieux de vie sont fort fréquentés, et le duo est abandonné avant de les avoir atteints, mais le bruit et l'odeur sont suffisants pour les reconnaître même à distance.

Le dernier élément manquant à l'appel est bientôt découvert. Des appartements isolés, gardés, et nettement plus riches. C'est dans ce coin que doivent vivre les gros bonnets, les têtes pensantes.

La tentation est grande d'aller y jeter un œil, mais la corde de la chance est déjà bien effilochée. 

Si vous avez le Code \emph{Loi}, rendez-vous au \glink{18}.

Sinon, rendez-vous au \glink{87}.

\gsection{83}

C'est un beau discours qui retentit dans l'arène végétale. Il parle de changement, d'un nouvel avenir possible. Il parle de rompre avec les malédictions du passé, il parle d'union contre l'adversité. Lorsqu'il s'interrompt, les gradins se remplissent de murmures, de bruissements, de débats passionnés. Le silence n'est rétabli que quand la voix impériale l'exige.

Rendez-vous au \glink{20}.

\gsection{84}

Quand l'être ordonne, il est obéi. Un automate dénué de pensées s'avance sans que rien ne l'en empêche, et présente son cou dégagé à la créature. Celle-ci s'empresse de satisfaire ses appétits, et bientôt il y a deux morts-vivants dans la cellule, l'un esclave de l'autre.

Notez le Code \emph{Corruption}.

Rendez-vous au \glink{33}.

\gsection{85 – Le marais interdit}

Au fond d'une vallée encaissée où l'eau de pluie s'accumule facilement se trouvent un ensemble d'étendues d'eaux putrides. Cet endroit sombre et nauséabond porte le doux nom de Marais Noir. Terre d'abondance pour les moustiques, il est ignoré des hommes et des animaux sensés. Supposément ne s'y rendent que les herboristes les plus courageux, pour s'y procurer quelque plante rare.

Mais parfois une silhouette solitaire s'y engouffre à la recherche de quelque chose dont la nature même lui est inconnue. Commence alors une longue et pénible errance, dans la boue, le froid, la brume. Et lorsque la fatigue commence à se faire ressentir, que l'explorateur s'efforce de retrouver son chemin pour être au sec avant que la nuit noire ne soit là, c'est alors que se manifestent les visions. Des flashs soudains, de souvenirs d'événements d'un passé lointain qui remontent et s'imposent à l'esprit du marcheur, des secrets longtemps enfouis qui éclatent en atteignant la surface, des fragments d'histoire enterrées qui s'échappent de leur prison d'oubli.

Si vous avez envoyé le \hero{Mnémonique}, le \hero{Mage}, la \hero{Sorcière} ou un anonyme spécialisé en Recherche, rendez-vous au \glink{11}.

S'il s'agit plutôt de la \hero{Princesse}, du \hero{Chasseur}, du \hero{Croisé} ou d'un anonyme expert en Combat, rendez-vous au \glink{130} si vous disposez du Code \emph{Folie}, et au \glink{69} sinon.

Si c'est le \hero{Brigand}, l'Homme à la \hero{Main d'Acier} ou un anonyme ne rentrant pas dans les cas précédents, rendez-vous au \glink{77}.

Si votre choix s'est porté sur la \hero{Bête}, le \hero{Métamorphe} ou la \hero{Légende}, rendez-vous au \glink{34}.

Enfin, si votre champion a été le \hero{Beau Parleur}, rendez-vous au \glink{48}.

\gsection{86}

Une armée de chimères, d'illusions et de faux-semblants protègent la créature, mais un esprit fort et préparé peut les percer à jour et venir jusqu'à elle. Mais l'affrontement avec la bête elle-même est d'un tout autre niveau.

Une fois entré dans sa clairière, s'il faut nommer ainsi cette esplanade corrompue dépourvue de toute autre forme de vie, qu'elle soit animale ou végétale, s'approcher ne serait-ce que d'un pas supplémentaire demande un effort surhumain. En partie à cause de l'influence pernicieuse de l'être, mais aussi car il est nécessaire d'étouffer tous ses instincts de survie les plus élémentaires. Et cela bien longtemps avant d'être à portée physique de la chose.

Qui avez-vous choisi pour cette mission périlleuse ?

Si c'est la \hero{Légende}, rendez-vous au \glink{62}.

S'il s'agit du \hero{Mnémonique} ou du \hero{Mage}, et qu'il dispose à la fois de l'Orbe Transiente et du Tome Scellé, rendez-vous au \glink{2}.

Si votre choix s'est plutôt porté soit sur la \hero{Bête} munie du Kriss du Dragon ou de la Cape des Crocs, soit sur le \hero{Chasseur} revêtu de cette même Cape des Crocs, rendez-vous au \glink{70}.

Si le \hero{Croisé} accomplit cette tâche et que vous avez le Code \emph{Mystique} ou que la Sorcière s'y est attelée, qu'elle est associée au Code \emph{Sabbat} et amène l'Orbe Transiente, rendez-vous au \glink{14}.

Si vous avez désigné la \hero{Princesse}, que vous avez le Code \emph{Héritage}, et qu'elle emporte avec elle une Branche de l'Arbre-Vie, rendez-vous au \glink{129}.

Si vous avez confié la Lame Maudite à l'homme à la \hero{Main d'Acier}, ou qu'un anonyme doté d'une Spécialité en Combat en dispose et que vous avez également le Code Sceau, rendez-vous au \glink{112}.

Si c'est le \hero{Métamorphe}, que vous avez le Code Mille et qu'il transporte soit le Coffret des Malheurs soit la Branche de l'Arbre-Vie, rendez-vous au \glink{40}.

Dans tous les autres cas, rendez-vous au \glink{104}.

\gsection{87}

La tentation l'emporte sur la prudence. Mais pour réussir un coup aussi risqué, il faudra des compétences extraordinaires... Ou une chance tout aussi exceptionnelle !

Si vous avez envoyé le Métamorphe et que vous possédez le code \emph{Mille}, rendez-vous au \glink{126}.

Si la personne qui accomplit cette mission porte la Cape des Plumes, rendez-vous au \glink{100}.

Sinon, plusieurs possibilités :
\begin{itemize}
\item C'est la \hero{Légende} qui accomplit cette mission.
\item Vous avez le Code \emph{Sabbat}, et c'est la personne associé à ce Code que vous avez envoyé pour cette mission.
\item Vous avez le Code \emph{Féerie}, et vous avez également choisi celui ou celle qui vous a permis de l'obtenir pour cette opération.
\end{itemize}

Si vous êtes dans un de ces cas, rendez-vous au \glink{24}.

Sinon, rendez-vous au \glink{33}.

\gsection{88}

La nuit s'écoule. Des marchandages effrénés ont lieu, sur des thèmes diaboliques dont la simple évocation dans un lieu saint suffirait à se faire excommunier. Mais quand le soleil se lève, si quelqu'un devra dorénavant vivre en sachant des choses qu'il aurait préféré ne jamais savoir, la Baba Yaga est elle satisfaite. Elle commence alors une histoire.

Rendez-vous au \glink{72}.

\gsection{89}

C'est un esprit meurtri par les manipulations mentales qu'il a subi qui est revenu du marais. Mais c'est également une âme vindicative, prête à y retourner sur le champ pour affronter la source de ses tourments.

Sa furie est telle qu'à la prochaine phase de recrutement, vous pourrez l'assigner, gratuitement et en ignorant les restrictions habituelles, à la mission de votre choix. Après cela, sa colère retombera, et vous devrez de nouveau payer le tarif normal si vous désirez l'utiliser.

Rendez-vous au \glink{50}.

\gsection{90}

L'Ordre se méfie de tout et de tous. Tout ce qui paraît n'être pas complètement humain est évidemment rejeté, de même que tout ceux qui ont touché directement ou indirectement au surnaturel. Même les gens qui ont simplement le malheur de sortir du lot grâce à leur esprit, que ce soit par leur charisme ou leur intelligence, sont refoulés, car plus aptes à jouer double jeu.

Mais cette organisation a aussi un besoin vital en informations, et quand quelqu'un, d'apparemment normal, se présente à eux avec les restes d'un vampire que l'Ordre n'aurait pas tout à fait achevé, ils sont bien obligés de la questionner dans les règles. Même si son histoire est douteuse, des cas similaires se sont malheureusement déjà produits, et ils ne prendront le risque de laisser un mort-vivant en liberté.

Leurs spécialistes écoutent donc sa version avec attention, puis examinent la relique, qu'ils mettent sous clé peu après. Un traqueur est même dépêché pour réexaminer le contenu de la pièce secrète.

Ce premier contact n'est cependant guère instructif pour la personne interrogée. Mais avant de repartir, à sa grande surprise, un ponte de l'Ordre la prend à part.

Rendez-vous au \glink{131}.

\gsection{91}

Vous n'avez probablement pas choisi la personne la plus adaptée pour cette quête. Après plusieurs jours d'enquête infructueuse, une nouvelle attaque a lieu sans que le début de l'ombre d'une piste soit en vue. Non seulement vous ne progressez pas, mais vous êtes obligé de faire battre en retraite votre détective alors que les soupçons commencent à se porter sur lui. 

Diminuez votre Réputation de 1.

Rendez-vous au \glink{50}.

\gsection{92}

\success

Votre pie voleuse se débrouille avec brio, et parvient  non seulement à s'introduire dans les appartements privés du gourou au nez et à la barbe des gardes, mais également à ouvrir son coffre privé. En plus d'une petite fortune en or et pierres précieuses, ce dernier contient une petite boîte ouvragé, présentant un étrange entrelacs de runes pour toute serrure. Son butin en poche, l'oiseau de nuit s'envole.

\ellipse

À défaut d'informations directement exploitables, votre ombre vous a ramené deux objets forts intéressants de son expédition.

Le premier est une copie des principaux textes sacrés de la secte, une série de psaumes rassemblés d'une écriture serrée sur des parchemins de récupération. Probablement compilée et oubliée par quelque novice, elle forme une bonne introduction à leur religion.

Sans surprise, son credo est la résurrection des morts, bien qu'elle soit bien plus libertaire que d'autres sur le sujet. Ainsi, les morts-vivants supérieurs, c'est-à-dire ceux dotés de conscience comme les vampires ou les liches, sont considérés comme des ressuscités à part entière, des êtres bénis à qui le seigneur a offert le don de seconde vie. L'idéal absolu des adeptes semble d'ailleurs être la seconde vie éternelle, débarrassée une fois pour toutes du risque de la mort.

Notez le Code \emph{Religion}.

Le deuxième objet est un petit coffret. Il est fermé par magie, mais le sceau est un travail d'amateur, et vous n'avez besoin que de gratter quelques runes avec la pointe d'un couteau pour le faire sauter. À l'intérieur se trouve une relique qui aurait mérité bien meilleur écrin. Pour un profane, c'est un poignard sacrificiel frustre, un croc à peine taillé. Mais quiconque a un jour pratiqué les arts obscurs reconnaîtra sans mal l'animal dont il est extrait : c'est un élément de la dentition d'un dragon, et même, comme l'indiquent les strates, d'un dragon millénaire.

Ce couteau rarissime est très recherché par les mages, et les tueurs de mages, pour sa totale imperméabilité à la sorcellerie, qui lui permet de s'affranchir totalement de toutes les barrières surnaturelles. C'est typiquement une des rares choses qui pourraient vous tuer. À manier avec précaution donc.

Ajoutez le Kriss du Dragon [-] à vos Possessions.

Rendez-vous au \glink{50}.

\gsection{93}

Si votre Réputation est de 3 ou plus et que vous n'avez encore jamais croisé la \hero{Princesse}, rendez-vous au \glink{25}.

Sinon, mais que vous venez d'accomplir votre septième phase de recrutement, qu'importe que les Quêtes correspondantes aient réussi ou échoué, rendez-vous au \glink{52}.

Dans tous les autres cas, rien de particulier ne se passe, et un nouveau cycle Recrutement/Quête démarre comme à l'ordinaire.

\gsection{94}

Malgré tout son talent, malgré toute sa fougue, la personne en qui vous aviez placé vos espoirs ne parvient pas à convaincre l'assemblée. Ses paroles ne rencontrent nulle approbation, nul enthousiasme. Quand elle abandonne enfin la partie, seul le silence lui répond. Dans un violent tourbillon, une tornade furieuse, l'arène disparaît et elle se retrouve à l'orée des bois.

\ellipse

La mission a été un échec, mais au moins quelqu'un est revenu pour vous le dire explicitement, ce qui n'arrive malheureusement pas assez souvent. À vous d'en tirer les enseignements pour une prochaine tentative.

Rendez-vous au \glink{50}.

\gsection{95 – L'avatar}

Les grottes ont été dévastées, pillées. L'Ordre est passé par ici, a arraché tout ce qui était transportable et brisé le reste. Néanmoins, trop sûrs d'eux, ils n'ont pas fouillé aussi bien qu'ils auraient dû. Il existe une porte secrète, menant à de nouvelles galeries, plus frustres, moins bien étançonnées.

Tout au bout de ce nouveau labyrinthe de cul-de-sacs, de boucles, d'errements, se trouve une grande caverne. Son plafond, son plafond, ses murs, sont intégralement couverts de vifs symboles, aussi éclatants que s'ils venaient d'être dessinés. Près de son unique entrée s'accumule un capharnaüm d'objets, outils, pots de peinture, tabourets, livres, entassés pêle-mêle comme s'ils avaient été jetés là sans ménagement.

Au centre de celle-ci, au milieu de cercles concentriques de complexes inscriptions, se tient un homme, nu comme un ver, négligemment assis. Il semble humain au premier coup d’œil, mais un examen prolongé se révèle perturbant sans que l'observateur ne puisse expliquer pourquoi. Le fait est qu'il manque une multitude de micro-mouvements que les êtres de chair et de sang accomplissent naturellement. Pas de déglutition, pas de légère vibration dans les artères du cou, pas de mouvement respiratoire même minime.

Les sectateurs ont trouvé leur messie, leur non-mort parfait. Ils l'ont peut-être même déjà découvert depuis des siècles. Mais s'il est encore ici, sous terre, c'est qu'ils n'ont jamais réussi à briser les dernières barrières, les dernières sceaux érigés par ceux qui l'ont enfermé ici.
Avec un regard de dédain, d'ennui, l'être se lève, et ouvre la bouche.

« À genoux. »

La voix est un ordre. La voix est un commandement. La voix est absolu.

S'il s'adresse à la \hero{Légende} ou à quelqu'un ayant déjà accompli avec succès les Quêtes \textbf{Le marais interdit} et \textbf{Les cavernes du début du monde}, rendez-vous au \glink{45}.

Si vous avez confié la Couronne des Fées à votre émissaire, rendez-vous au au \glink{106}.

Si ce n'est pas le cas, mais que vous avez le Code \emph{Vie}, rendez-vous au \glink{55}.

Sinon, rendez-vous au \glink{84}.

\gsection{96}

Il serait exagéré de dire que le soleil brille, que les oiseaux chantent et que les gens sont tous heureux, mais l'ambiance dans les villages et les hameaux est bien meilleure. Grâce à vos efforts, la région semble reprendre vie petit à petit. Votre émissaire est d'ailleurs bien accueilli là où hier les cailloux et les injures auraient été ses seuls cadeaux. La méfiance commence à se dissiper, et c'est là la plus précieuse des informations que vous récoltez.

Notez le Code \emph{Vie}. Si c'est la première fois que vous l'obtenez, augmentez votre Réputation de 1 point supplémentaire.

Rendez-vous au \glink{50}.

\gsection{97}

Négocier avec un juge des enfers est une sinécure à côté d'une discussion avec un membre de l'Ordre ! Les gardes grognent dès que quelqu'un essaye de leur parler, et deviennent même rapidement agressifs s'il a le malheur d'insister un peu trop longtemps. De multiples stratégies d'approche, et même des essais sur des jours différents, avec des gardes différents, n'y font rien : impossible de ne serait-ce qu'entamer la conversation. Il s'avère même nécessaire de battre en retraite avant que tout cela ne dégénère.

\ellipse

Ce n'est pas de gaieté de cœur que votre émissaire vous annonce son échec. Au moins l'Ordre est-il toujours là, et vous pourrez toujours réessayer plus tard, avec un autre missionnaire.

Rendez-vous au \glink{50}.

\gsection{98}

Quelque part au fin fond d'un subconscient, une petite lumière s'allume. Une minuscule lueur, chancelante, étouffée dans le flot anesthésiant de la paix et du bonheur. Le rayon devient son, mélopée, et quelques syllabes malhabiles s'échappent d'elles-mêmes de lèvres qui auraient voulu rester scellées.

Le choc est comme un plongeon dans l'eau glacée. Les mémoires occultées déferlent, emportant par leur nombre et leur clarté les maigres illusions du présent. Le monde cesse d'être un paradis et redevient une grotte obscure sur le sol de laquelle est prostrée une silhouette solitaire, qui lentement se relève, puis reprend sa route vers l'inconnu.

Rendez-vous au \glink{44}.

\gsection{99}

Quelques temps après avoir lancé vos hameçons, vous finissez par recevoir un courrier. Il porte la signature d'un de vos amis, un historien possédant une impressionnante collection d'ouvrages anciens. C'est un fanatique de la version originale, cherchant toujours à se procurer le texte de départ plutôt que ses multiples traductions ultérieures, pour s'assurer que le sens n'a pas été altéré, volontairement ou non, avec le temps.

Il a trouvé un extrait, qui d'après son âge et son origine géographique, pourrait correspondre à votre problème. Il vous fournit, par principe, une copie conforme de l'original, rédigé dans un système d'écriture que vous ne connaissez pas, ainsi que sa propre traduction :

Arrignidák, la perle du couchant, la ville aux sept obélisques, régnait sur le monde. Dans leur orgueil, ils allèrent là où aucun être humain n'aurait jamais dû aller, et réveillèrent ce qui y dormait. Alors en une nuit,  une seule longue et terrible nuit, les remparts centenaires, les magnifiques jardins, toute la connaissance et la vie de ses habitants, tout fut dévoré. Car dans leur folie, ils avaient appelé celui qui vit entre les mondes, le juge qui jamais n'innocente, le héraut de la destruction.

Cependant, il n'a trouvé aucune évocation d'un moyen de vaincre l'être dans les autres légendes de cette civilisation. Il est en fait rarement cité, et toujours sous la forme d'une insulte ou d'un châtiment, être dévoré par lui étant considéré comme la pire façon de mourir, car non seulement le corps est détruit, mais il déchiquette également l'âme de sa victime.

Retournez au \glink{50}.

\gsection{100}

La cape frémit. Les plumes se lissent, s'unifient comme si elles appartenaient à une véritable aile et n'étaient pas simplement attachées ensemble. Ses couleurs se mélangent jusqu'à former ce noir pas tout à fait parfait de la véritable obscurité. La peau de l'humain qu'elle enveloppe s'altère de même, jusqu'à ce qu'il ne soit plus qu'une ombre parmi les ombres, une chouette à l'affût dans la nuit.

De son nouveau pas, léger, sans trace, l'être à mi-chemin entre deux mondes se glissent silencieusement vers sa proie.

Rendez-vous au \glink{92}.

\gsection{101}

\success

La créature fait près de deux fois la taille de l'humanoïde qu'elle attaque. Sa supériorité physique est évidente. Toutefois, elle n'a jusqu'ici affronté que des combattants médiocres, ne disposant pas de techniques ou d'équipement adaptés pour gérer sa résistance naturelle, ses capacités de régénération, sa force brute. Et cela change tout.

La situation bascule très vite en sa défaveur, et elle s'effondre bientôt de tout son poids sur un vieil arbre qui se brise sous le choc. Dans la mort, elle retrouve des dimensions et des traits humains.

\ellipse

Il a été assez délicat de convaincre la population que votre émissaire avait bien éradiqué le monstre qui les tourmentait et pas simplement assassiné le premier passant venu avant de l'accuser d'être un lycanthrope. Heureusement, vous aviez les traces indubitables de la présence de la créature à l'endroit du combat pour soutenir votre version des faits, et l'absence de nouvelles attaques depuis cet affrontement joue aussi en votre faveur. Problème réglé donc.

Augmentez votre Réputation de 1.

Si vous avez le Code \emph{Révélation}, rendez-vous au \glink{116}. Sinon, rendez-vous au \glink{50}.

\gsection{102}

Autant négocier avec un pot de chambre ! Le gardien reste imperturbable dans sa rutilante armure. Il n'est même pas sûr qu'il comprenne le flot de paroles qui se déverse sur lui. Après une éternité de monologue, changement de stratégie. L'intrus en ces bois enchantés dépose ses armes à terre et s'assoit confortablement dans l'herbe. Une longue attente commence alors.

Le soleil a le temps de se coucher et de se relever que la situation n'évolue pas. Le sommeil finit par gagner, et de lourdes paupières se ferment quelques instants. Quand elles se rouvrent, le lieu a changé. Il s'agit maintenant d'une vaste clairière encaissée, comme un théâtre ou une arène. Ce n'est plus un mais douze chevaliers qui sont présents, formant un cercle empêchant toute sortie. Armes et équipement ont disparu.

Un vaste public s'entend, bruyant, tapageur, mais reste invisible. Une voix royale retentit, étouffant tout autre son :

« Le petit peuple vous écoute. »

Si vous obéissez à au moins trois des conditions suivantes, rendez-vous au \glink{83} :
\begin{itemize}
\item Vous possédez le Code \emph{Ombre}.
\item Vous disposez du Code \emph{Sabbat}.
\item Vous avez le Code \emph{Entente}.
\item Votre émissaire est \emph{naturellement} doué en Diplomatie.
\item Il ou elle porte la Cape des Plumes.
\end{itemize}

Sinon, mais que vous avez envoyé le \hero{Beau Parleur}, la \hero{Princesse} ou le \hero{Métamorphe}, et que vous lui avez confié un instrument de musique quelconque, rendez-vous au \glink{43}.

Dans tous les autres cas, rendez-vous au \glink{94}.

\gsection{103 – Baba Yaga}

Tout au fond des bois se trouve une petite chaumière qui marche. Perchée sur deux solides pattes de poulet, aussi larges et hautes que de robustes chênes, elle se déplace régulièrement, transportant son unique occupante là où elle le désire.

Celle qui vit ici est bien sûr une sorcière, mais pas n'importe quelle sorcière : il s'agit de la Baba Yaga, l'ogresse aux dents d'acier, l'immortelle vilaine de nombreux contes et légendes. Elle était déjà vieille quand le monde était jeune, déjà crainte quand l'humanité savait à peine parler.

Bien peu sont ceux qui lui rendent visite, et encore plus rares ceux qui le font de leur plein gré. Mais aujourd'hui, quelqu'un est là pour demander une audience. Et si la Baba Yaga est anthropophage et toujours affamée, elle est également curieuse, aussi retient-elle son appétit pour laisser monter cette téméraire personne.

Si vous avez choisi le \hero{Métamorphe} pour cette mission, rendez-vous au \glink{132}.

Si vous avez désigné le \hero{Brigand}, l'Homme à la \hero{Main d'Acier} ou la \hero{Sorcière}, rendez-vous au \glink{27}.

Si c'est le \hero{Beau Parleur} qui est présent ou que vous avez envoyé quelqu'un de spécialisé en Diplomatie et disposant d'un Fétiche en Os, rendez-vous au \glink{88}.

Dans tous les autres cas, rendez-vous au \glink{68}.

\gsection{104}

Quel espoir peut avoir un simple humain contre une créature dont la simple existence tord son esprit, dévore sa chair ? Un courage incroyable, une chance insolente, une volonté d'airain, lui permettront peut-être de tenir quelques instants de plus, mais ses efforts dérisoires ne sont rien face à la toute puissance de son adversaire. C'est un affrontement perdu d'avance qui se déroule dans cette clairière, un combat ordinaire entre l'éléphant et la souris, sans que le miracle de la fable ne s'accomplisse.

Rendez-vous au \glink{33}.

\gsection{105}

\success

Les adeptes du dieu de la non-mort ont souvent eu d'autres religions auparavant, avant de se convertir à la vraie foi. Certains d'entre eux appartenaient à une secte encore plus barbare, vénérant les deux jumeaux du sang, frère et sœur ennemis, avatars de la destruction, de la violence, du carnage.

Et cette nuit-là, leurs anciennes croyances leur reviennent avec une vigueur décuplée. Le sacrifice était seul, isolé, ils étaient une centaine. Pourtant, tout du long, le combat fut à un seul sens. Les fidèles attaquaient, attaquaient mais leur cible se refusait à s'effondrer, et à chaque assaut leurs rangs s'éclaircissaient.

Était-ce que leur victime profitait de leur inexpérience du combat de masse pour retourner leur nombre contre eux, les sectateurs se gênant mutuellement ? Disposait-elle de puissants pouvoirs thaumaturgiques lui donnant la force de cent hommes ? Avait-elle simplement une hargne, une prestance telle que leurs réflexes, leurs réactions étaient émoussés par la peur ?

Toujours est-il qu'au cœur du temple, au cœur de l'arène, l'incarnation de la guerre dansait, et les hommes tombaient.

Alors le grand prêtre retira le voile couvrant l’œil du dieu, et invoqua son aide dans une prière désespérée, reprise en chœur par les survivants. Et il lui répondit.

La réalité se tordit, les spectres vengeurs, les âmes des morts damnés, traversaient les barrières entre les mondes pour libérer leurs courroux sur les vivants, sans discrimination.

À partir de là, ce fut le chaos. Le chaos pur, indescriptible. Le flot des événements ne peut être retracé, seuls leurs conséquences sont concevables. Le temple s'effondra, ainsi que la plupart des galeries. Une dizaine d'adorateurs réussirent à s'échapper, courant comme s'ils avaient tous les démons de l'enfer aux trousses. Leur désarroi était tel que la plupart se jetèrent droit dans les griffes de l'Ordre. L'intrus s'en sortit également. Quant à ce qui avait été libéré au fond de la montagne, nul ne put donner une estimation valide sur leur nombre, leur nature et, surtout, combien avaient survécu.

\ellipse

L'origine du mal échappa au drame. Votre redoutable émissaire ne se rappelle pas l'avoir emportée dans la furie de la bataille, mais elle était dans ses affaires quand son esprit s'est éclairci. L’œil du dieu. Une sphère noire comme le péché, froide et lisse comme du verre au toucher. Il est difficile de le regarder trop longtemps sans se sentir vaciller, et vous préférez le conserver emballé dans un tissu.

L'objet était utilisé par la secte pour lier ce plan à d'autres, dans le but de communiquer directement avec leur dieu. Rien ne dit qu'ils y aient réellement réussi. Votre avis est que ces fous ont eu accès par un hasard à un pouvoir qui les dépassait largement, un artefact capable d'affaiblir ou de briser certaines lois élémentaires de l'univers, et qu'ils l'utilisaient sans en comprendre l'origine ou la portée. Vous espérez que vous serez un peu plus sage qu'eux.

Ajoutez l'Orbe Transiente [-] à la vos Possessions. Augmentez également votre Réputation de 1.

Si vous avez le Code \emph{Loi}, rendez-vous au \glink{4}. Sinon, rendez-vous au \glink{50}.

\gsection{106}

Le commandement se heurte à la barrière d'une force naturelle sur laquelle l'être n'a aucun contrôle, une énergie vitale qu'il ne peut manipuler, altérer ou corrompre. Et tous ses pouvoirs, toute sa malice, se trouvent alors annihilés.

Rendez-vous au \glink{41}.

\gsection{107}

La clé du mystère, ce n'est pas la bête, c'est l'homme. Que faisaient ces gens, dehors, seuls, à la tombée de la nuit, dans les bois ? Ce n'étaient pas des touristes, ils savaient depuis leurs naissances le danger que représente un tel comportement. Qu'est-ce qui les a attirés en ces lieux ? Ou, s'ils n'y sont pas venus de leur plein gré, qu'est-ce qui les y a amenés ? Probablement quelque chose de plus discret qu'une bestiole démesurée devant déraciner des arbres pour se creuser un chemin.

En utilisant ce nouvel angle d'approche, il est possible de reprendre l'enquête à zéro, en se traînant non dans la forêt, mais dans les villages où vivaient les victimes. Des débris d'une corde près d'une literie évoque un enlèvement dans un cas, un reste de message suggère un rendez-vous galant dans un autre, l'interrogatoire d'un ami révèle qu'un des morts lui avait dit être sur un gros coup financier peu avant sa fin tragique... Rapidement se dessine un profil, puis un suspect. Au jeu de la filouterie, du mensonge et de la tromperie, la balance penche en votre faveur.

Au final, celui vers qui toutes les pistes convergent est sommé de s'expliquer par votre émissaire, escorté de quelques gros bras recrutés parmi les habitants du cru disposant des armes de leurs professions : hache, faux, marteau. Deux faits sont alors clairement établis. Le premier est sa culpabilité. Le second est qu'il puisse se transformer à volonté, même de jour.

Sa nouvelle apparence n'a que du loup que la forme générale du visage. C'est plutôt une masse de muscles hypertrophiés couverte de poils, affectant une vague forme bipède. Un gorille avec des crocs et des griffes, qui malgré son infériorité numérique criante se jette dans la mêlée avec la volonté de tout détruire.

Si vous avez choisi quelqu'un disposant d'une arme en argent, rendez-vous au \glink{32}.

Si ce n'est pas le cas, rendez-vous au \glink{114}.

\gsection{108 – Ceux qui vivent}

L'Ylèdre n'est pas excessivement peuplé, mais les habitants sont grandement dispersés, avec de multiples hameaux isolés pour quelques grandes villes seulement. Acquérir des informations sur l'arrière-pays nécessite donc tout autant des jambes solides et des chaussures robustes pour gravir les collines qu'une langue agile et un charisme naturel pour faire parler les solitaires les plus renfrognés.

Si votre émissaire possède l'Information \emph{Là où se cachent les ombres} et que vous n'avez pas déjà le Code \emph{Ombre}, rendez-vous au \glink{117}.

S'il dispose plutôt de \emph{Du loup à l'homme} mais que le Code \emph{Révélation} vous est inconnu, rendez-vous au \glink{121}.

Si vous avez envoyé la \hero{Princesse}, rendez-vous au \glink{111}.

Si vous avez désigné le \hero{Croisé}, rendez-vous au \glink{124}.

Si votre choix s'est porté sur la \hero{Sorcière}, rendez-vous au \glink{21}.

Si vous n'êtes dans aucun de ces cas particuliers, rendez-vous au \glink{75}.

\gsection{109}

Plusieurs jours passent sans nouvelle, et vous commencez à penser que vous investissez bien mal vos richesses, quand le voleur revient au crépuscule. Bien qu'il soit physiquement intact, il a perdu beaucoup de sa superbe et de sa morgue, et c'est avec soulagement qu'il vous remet un épais paquet avant de disparaître dans la nuit naissante.

Vous déballez son présent et découvrez un antique manuscrit à la reliure d'acier. Il est scellé par un cadenas, mais la clé est à proximité. Vous faites jouer la serrure, et vous rendez vite compte qu'il ne vous a pas menti. L'ouvrage est un recueil de connaissances ésotériques, d'informations maudites et taboues. Vous essayez de lire toutes les pages qui pourraient concerner votre quête, mais vous abandonnez très vite, vous sentant déjà mal, confus, perturbé. C'est un livre à consommer à très petite dose, ou il vous coûtera votre santé mentale.

Si vous avez une question précise dont vous désirez la réponse, il vous apportera sans doute ce que vous cherchez, au prix d'un peu de folie. Mais sinon, mieux vaut le garder sous clé.

Ajoutez le Tome Scellé [-] à vos Possessions.

Rendez-vous au \glink{50}.

\gsection{110}

C'est à peine si les habitants ne lapident pas à mort votre porte-parole. Vous n'obtenez rien de plus, pas une information, pas même un encouragement.

Rendez-vous au \glink{50}.

\gsection{111}

S'il n'y a plus de noblesse officielle en Ylèdre, cela ne veut pas dire qu'il n'y subsiste pas une haute bourgeoisie, et que parmi ses nombreux paysans et ouvriers pauvres ne se trouvent pas aussi quelques riches marchands. Ils ont simplement l'intelligence de se faire discrets. Toutefois, pour qui a les bonnes connexions, il est facile de les rencontrer.

Un certain nombre d'heures ont été nécessaires à deux servantes désespérées pour rendre présentable le jeune fille en robe de soirée qui s'avance dans la salle de bal. Malgré cela, elle se meut comme un poisson dans l'eau, saluant les autres invités avec une étiquette irréprochable, passant de l'un à l'autre en échangeant les banalités requises.

Assez rapidement cependant, elle s'ennuie mortellement. Aucune des personnes présentes n'a d'attaches plus que financières avec ce pays. Au contraire, elles font même des efforts pour ne pas creuser sous la surface, ne pas voir ce qui se cache dans ses profondeurs.

Alors qu'elle médite à ce sujet seule sur le balcon, elle est abordée par une vieille connaissance. Risque évident, qu'elle savait prendre en se rendant à une soirée de ce type. En l’occurrence, il s'agit d'un de ses nombreux oncles, parmi les plus jeunes, ceux n'ayant aucun espoir d'hériter de la charge de son grand-père et condamnés à trouver d'autres moyens de subsistance.

Celui-ci a choisi le commerce. Mais le blason vert et argent, ce grand arbre sur fond de lune, référence directe à leur légendaire ancêtre commun, qu'il arbore fièrement sur son torse alors qu'elle l'a rejeté, ne laisse pas de doute sur ses origines aristocratiques. Certains diraient même théocratiques.

Après les effusions d'usage, et les habituelles tentatives pour la ramener dans le bon chemin en lui rappelant ses devoirs, le parent change d'angle d'approche :

« Je ne crois pas que vous acharner à refuser d'assumer la place que votre naissance vous impose soit la bonne solution. Oublions un instant toutes ces histoires d'obligation morale. Pour réussir dans la vie, il faut savoir utiliser toutes les opportunités à sa disposition. Vous vous êtes bien servie de votre titre pour vous faire inviter ici, pourquoi vous refuser à recevoir ce qui vous revient de droit ? J'admets que certains éléments que cela implique sont embarrassants, voire paralysants, mais cela vous donnera aussi accès à bien des avantages que vous ne pourrez obtenir en vous acharnant à vous comporter comme si vous n'étiez qu'une aventurière à la petite semaine de basse extraction. Personne n'est dupe, et vous vous faites du mal à vous-même en vous forçant à jouer la comédie. »

« Vous savez pertinemment que si je retourne là-bas, ils ne me laisseront pas partir. »

« Ils ne voudront pas vous laisser partir. Avec votre caractère de cochon et votre obstination de mulet, je doute qu'ils y arrivent bien longtemps. Non, je pense plutôt que la vraie raison pour laquelle vous refusez de retourner là-bas, c'est que vous avez plus peur du savon que vont vous passer mon frère et sa femme que des monstres qui rodent dans ces bois. Ne serait-il pas temps de nous démontrer que vous avez réellement le courage de vos convictions et pas une simple lubie de jeunesse ? Si vous voulez vraiment continuer dans cette vie, allez leur dire en face au lieu de continuer à les fuir. »

\ellipse

La princesse a deux nouvelles pour vous. La première, c'est qu'elle a effectué une collecte entre votre faveur auprès des bourgeois locaux, et même si elle peste énormément sur leur avarice, elle a réussi à faire se délier quelques cordons de bourse. Ajoutez 50 pièces d'or à celles dont vous disposez actuellement.

La deuxième, c'est qu'elle va devoir s'absenter quelques temps, pour raisons familiales. Elle ne sait quand elle pourra revenir, si elle le peut. Mais si le destin en décide ainsi, peut-être vos chemins se recroiseront-ils.

Vous la voyez repartir à cheval en compagnie d'un homme plus âgé, avec lequel avec elle partage une vague ressemblance, son visage transpirant la joie du taureau qui va à l'abattoir.

La \hero{Princesse} est dorénavant indisponible.

Notez le Code \emph{Héritage}.

Rendez-vous au \glink{50}.

\gsection{112}

La créature est gigantesque, ne ressemble à rien de connu, mais dégage une infinie énergie destructrice. Avancer vers elle est déjà une épreuve en soi. Approcher assez près pour simplement la toucher serait déjà un exploit, l'abattre un miracle.

La lame entre ses mains est peut-être dotée de grands pouvoirs, mais cela reste un simple mètre d'acier contre des dimensions bien supérieures. Toutefois, lorsque le premier tentacule se détend pour l'abattre, son arme s'interpose machinalement, effort dérisoire s'il en est, car équivalent à bloquer un tronc d'arbre avec un brin de paille.

Mais lorsque le contact se fait, c'est l'appendice démesuré qui se replie brutalement, une profonde marque de brûlure à l'endroit de l'impact.

Il est impressionnant de noter à quelle vitesse un être humain peut passer de la résignation à la confiance, du désespoir à l'espoir. L'épée est efficace. Pour une raison ou pour une autre, elle est extrêmement nocive pour la créature cauchemardesque. Cela signifie qu'il doit être possible de l'abattre en s'en servant.

Le monstre tente bien d'arrêter la silhouette pitoyable, minuscule, qui fonce vers lui en faisant des moulinets. Mais il lui suffit de l'effleurer pour que ses membres se contractent en irradiant d'une sensation qu'il n'avait jamais connu jusque là : la douleur. Un assaut un peu plus courageux que les autres se solde même par la perte de l'un d'eux, coupé en deux, complètement calciné au niveau de la plaie.

Ce n'est que lorsque la pointe de l'arme s'enfonce dans son corps principal, que celui-ci se met à se convulser, à cloquer, à se déchirer de l'intérieur, qu'il comprend la véritable nature de ce qu'il affronte. Ce n'est pas une arme forgée par cette insignifiante humanité. C'est un être similaire à lui, peut-être même plus ancien, qui se développe et croît en dévorant ses semblables. Un dieu prédateur, limité sur bien des aspects, qui a besoin des stupides bipèdes pour des choses aussi simples que se déplacer, mais pour qui détruire une engeance comme la sienne est la chose la plus naturelle du monde.

L'essence du monstre est avalé par la fausse lame, digérée, emmagasinée. Le processus ne se fait pas en douceur. Le pantin persuadé de tenir les fils a le réflexe salvateur de jeter son arme le plus loin possible alors que celle-ci commence à se surcharger.

Puis elle explose. La lumière est aveuglante, le souffle renversant, mais aucune matière n'est projetée. L'épée est devenue une masse chaotique de particules, saturées de réactions chimiques violentes, se gonflant et se dégonflant aléatoirement, mais parvient à conserver son intégrité. Bientôt, c'est une lame stable, en apparence similaire à celle qu'elle était encore quelques minutes plus tôt qui tombe sur le sol, au milieu de restes craquelés, brûlés, fondus de sa dernière victime.

\ellipse

Vous ne pouvez pas en vouloir à votre déjà si téméraire épéiste de ne pas avoir osé ramasser son arme. C'était certainement la décision la plus intelligente à prendre devant un tel phénomène. Mais vous ne pouvez vous empêcher de vous demander ce qu'elle est devenue.
En effet, lorsque vous avez renvoyé des gens faire le ménage là-bas, ils n'en ont trouvé nulle trace. Les conséquences de son passage étaient clairement visibles, mais sa destinée après l'affrontement reste un mystère. Vous vous demandez si en résolvant un problème, vous n'en avez pas créé un plus grand.

Notez le Code \emph{Tempête}.

Rendez-vous au \glink{135}.

\gsection{113}

Une lame impie appelle un livre de la même nature. L'ouvrage ne manque pas d'informations sur un tel sujet, mais plutôt que découvrir l'identité exacte de la vôtre, vous êtes noyé sous des possibilités effrayantes auxquelles vous n'aviez pas songées auparavant. Il existe bien des créatures au-delà du voile de la normalité qui ont l'apparence d'une épée mais n'en sont pas. Si cette arme est en réalité un de ces monstres, la jeter au fin fond d'une fosse océanique est le seul comportement rationnel à adopter.

Comme vous n'êtes cependant pas sûr que ce soit le cas, vous faites taire vos instincts et prenez le risque de la conserver.

Votre lecture finit tout de même par vous apporter un élément directement utile. Un sortilège pour amoindrir l'influence d'une telle arme sur l'esprit de son porteur. Comme tout ce qui se trouve en ces pages, il est à double tranchant. La colère de l'arme est en effet redirigée vers le reste du monde, épargnant celui qui la tient, mais augmentant de façon conséquente les chances qu'elle entre dans un état de rage frénétique, où sa pointe volerait d'elle-même vers le cœur des autres présents, amis comme ennemis, jusqu'à qu'elle et son bretteur soient les seuls êtres vivants au milieu d'un océan de cadavres.

Si vous souhaitez prononcer ce rituel malgré cela, notez le Code \emph{Sceau}.

Dans tous les cas, retournez ensuite au \glink{50}.

\gsection{114}

L'affrontement est une boucherie. L'avantage du nombre est contrebalancé par l'absence d'expérience des combattants, et la différence criante dans leurs capacités physiques. Le sang gicle de tous les côtés, sous le tranchant des armes improvisées comme des griffes. La mêlée est infâme, confuse, destructrice. Plusieurs villageois succombent, d'autres sont mis hors-jeu par de graves blessures. Enfin la bête, lacérée de tous les côtés, cesse de bouger. Dans la furie du combat, les coups continuent cependant de pleuvoir pendant encore plusieurs minutes. C'est une victoire, mais elle a été chèrement payée. 
La personne à qui vous aviez confié cette quête s'en est sortie, mais pas indemne. Elle sera indisponible pour les deux prochaines phases de recrutement, le temps de se refaire une santé.

Rendez-vous au \glink{115}.

\gsection{115}

\success

Les échos de la mort de la créature sont arrivés à vous avant que votre détective ne soit revenu faire son rapport, mais lorsqu'il vous présente celui-ci, il y ajoute quelques informations intéressantes inédites. En fouinant pour découvrir la vérité sur ce monstre précis, il est rapidement apparu que d'autres villageois avaient eux-mêmes des secrets similaires, mais moins sanglants. Plusieurs habitations dans lesquelles ne sont pas censées vivre des animaux disposent d'entrées secrètes adaptées pour des quadrupèdes de la taille d'un beau chien. Les décoctions de mandragore et autres potions étranges sont étonnamment présentes dans la pharmacopée de certains. Lors de l'affrontement final, une main pas tout à fait humaine, plus une serre bordée de plumes noires, a été un instant entrevue dans le chaos de la mêlée, et ce plumage n'est pas sans rappeler celui d'un mystérieux oiseau qui a espionné toute la une partie de l'enquête.

Ce curieux aspect des choses mériterait effectivement d'être creusé. Une nouvelle visite, diplomatique et prudente, s'impose peut-être, pour voir ce qui peut-être tiré de ce nouvel éclairage.

Vous disposez dorénavant de l'Information \emph{Du loup à l'homme}. La personne que vous aviez envoyé dans cette mission la connaît bien évidemment aussi. Augmentez également votre Réputation de 1.

La \hero{Bête} est maintenant disponible.

Rendez-vous au \glink{50}.

\gsection{116}

Le chasseur a des informations très intéressantes à vous communiquer. Il a rencontré un autre métamorphe, bien moins impressionnant, au cours de sa quête et lui a fait avouer qu'il n'était pas le seul dans son cas. Il semblerait qu'il existe toute une confrérie d'humains altérés en ces terres. Qu'ils aient la possibilité de transformer, à volonté ou sous l'influence des astres ou qu'ils disposent de caractéristiques physiques animales permanentes, ce sont avant tout des exilés qui ont immigré en Ylèdre à la recherche d'une certaine forme de tranquillité, dans l'isolement naturel du pays, et qui s'entraident, solidaires dans leur différence. 

Pour préserver leur mode de vie, ils se chargent souvent d'éliminer les éléments incontrôlables eux-mêmes, avant qu'ils n'attirent l'attention. Ils ont cependant échoué sur l'affaire présente. 

Ceci apporte un éclairage nouveau à bien des événements qui se sont déroulés ici. De plus, votre efficace émissaire a réussi à arracher encore quelques anecdotes dignes d'intérêt. Débloquez la Quête suivante : 

\quest{Le cercle de la paix}{37}{Recherche (\caduceus)}{Moyenne}{
Si ce peuple des ombres consiste principalement en des lycanthropes et autres changeurs de forme parvenant à contrôler leurs transformations, quelques êtres encore plus anormaux vivent au cœur des bois, loin des regards. Ces créatures, beaucoup plus proches des racines mystiques du lieu, pourraient être une source d'informations précieuses.
}

Notez que vous disposez dorénavant de l'Information \emph{Là où se cachent les ombres}, et que le \hero{Chasseur} la connaît également.

La \hero{Bête} est dorénavant disponible.

Rendez-vous au \glink{50}.

\gsection{117}

Les informations que vous avez accumulées vous ont permis d'identifier un village particulièrement isolé qui hébergerait plusieurs presque humains incapables de cacher les disparités de leur apparence, mais cependant civilisés et pacifistes.

Lorsque votre émissaire y pénètre, l'ambiance est lourde. Les habitants, normaux en apparence, sont suspicieux devant cette visite de courtoisie. Ils restent renfermés, maussades, répondant aux questions par des monosyllabes inintelligibles. 

Avant de continuer, vous allez devoir estimer votre réputation auprès ce groupe particulier. Pour cela, prenez votre Réputation normale, et appliquez-lui les modificateurs suivants :
\begin{itemize}
\item Ajoutez 1 si la personne présente dispose de la Spécialité Diplomatie.
\item Ajoutez 3 s'il s'agit de la \hero{Bête}.
\item Ajoutez 3 si vous disposez du Code \emph{Féerie}.
\item Retranchez 2 si vous possédez le Code \emph{Loi}.
\end{itemize}

Si le résultat final est de 7 ou plus, rendez-vous au \glink{119}. Sinon, rendez-vous au \glink{120}.

\gsection{118}

Vous détestez prendre des décisions de ce type. Ce mort-vivant spécial est clairement maléfique, dangereux, mais c'est peut-être votre seul espoir de venir à bout de la créature qui vie dans les marais. Vous pensez qu'il vous a dit, à quelques omissions près, la vérité. Son idée a des chances de fonctionner, et il est plus probablement le seul être encore en vie (plus ou moins) à disposer des connaissances suffisantes pour établir un tel sceau.

De l'autre, ce n'est pas quelqu'un que vous avez envie de relâcher dans la nature comme cela. Le remède pourrait être pire que le mal. De deux maux, faut-il réellement choisir le moindre, ou y a-t-il encore une solution alternative ?

Mais commençons par le commencement. Disposez-vous déjà des ressources nécessaires pour réaliser un tel sceau ?

En effet, ramené aux dimensions du monstre que vous voulez enfermer, et en comptant sur le fait qu'il ne soit pas possible de l'approcher de trop près, cela nécessite de réaliser des cercles d'un diamètre gigantesque. Le fond d'un marais n'étant pas un terrain très propice, il sera nécessaire d'assécher, de construire une chaussée de briques sur laquelle il soit possible de graver et de peindre. Cela implique donc des ouvriers, à protéger contre les insinuations mentales de la créature, donc des charmes ou des protections, donc un soutien magique...

Si vous voulez persévérer dans cette voie, il vous faudra un nombre non négligeable d'alliés. Pour être exact, vous devez disposer des codes \emph{Vie}, \emph{Purification}, \emph{Ombre}, \emph{Sabbat} et \emph{Féerie}.

Si vous ne les avez pas, rendez-vous au \glink{50}.

Si vous avez réellement réuni les cinq pointes de l'étoile, et que vous souhaitez appliquer ce plan, vous devez décider dès maintenant à quel point vous comptez respecter votre part du marché. Vous pouvez bien sûr être honnête avec le non-mort et le laisser partir comme il vous le demande, mais vous pouvez aussi prévenir l'Ordre de sa existence au moment opportun pour qu'ils le cueillent à la sortie.

Notez votre décision, puis rendez-vous au \glink{7}.

\gsection{119}

Alors que la situation semblait destinée à rester bloquée éternellement, un nouvel arrivant fait son entrée, à la surprise générale. Pour toutes les personnes présentes sauf une, cette surprise est finalement modérée. C'est le fait de le voir ici et maintenant qui les surprend, pas son apparence. Et pourtant, il y aurait des choses à dire.

Haut de plus de deux mètres, couvert d'une épaisse toison bouclée noir corbeau débordant de ses vêtements, il pourrait passer pour un humain à très forte pilosité si son visage n'abritait pas deux yeux flamboyants, crépitants, un regard que la nature ne peut avoir créé. Malgré des efforts de propreté évidents, il dégage aussi une odeur animale forte, un puissant musc bestial.

En contradiction avec son physique, sa voix est étrangement basse, sans intonation, comme contenue, maîtrisée. Encore plus bizarre est le fait qu'elle propose aimablement de discuter autour d'un bon repas au lieu d'évoquer des images de sang, de violence et de destruction.

La soirée qui s'ensuit se déroule finalement sans accroc. Le géant y raconte les circonstances dans lesquelles il est devenu ce qu'il est aujourd'hui, et sa difficulté à retrouver un foyer après la mort du fou l'ayant transformé. Avec quelques survivants du même laboratoire alchimique, plus aptes à se fondre dans la société que lui-même, mais ayant cependant leurs propres altérations, il s'est installé ici, loin de tout.

Les années ont passé, leur communauté en a croisé d'autres avec des problèmes similaires. Certains se sont joints à eux, d'autres ont continué de leur côté, et souvent, la rencontre s'est mal terminée. Ces gens n'aiment en effet pas beaucoup plus les garous et autres hommes-bêtes destructeurs, sans contrôle, que l'ylèdrois moyen, et réagissent à peu près de la même manière quand ils les rencontrent. Ils les désignent d'ailleurs sous les termes de déchus, corrompus.

Les générations passant et les enfants naissant, parfois en conservant les tares de leurs parents, ils sont maintenant un nombre non négligeable. S'ils ont réussi à éviter d'être anéantis jusqu'au dernier par des zélotes, c'est avant tout en appliquant des règles extrêmement strictes de confidentialité et de secret, couplées à des sentences très lourdes en cas de transgression. L'individu se montre très intimidant quand il explique avoir exécuté de ses mains quelques imbéciles trop bavards.

Toutefois, vos actions semblent l'avoir favorablement impressionné, et il a donc décidé qu'il était peut-être temps de faire une exception. Il ne va pas non plus aller gambader sous le nez des fenêtres des tueurs de monstres, mais si les circonstances s'y prêtaient, il pourrait peut-être vous soutenir.

Pour l'heure son assistance prendra la forme d'un précieux objet, infusé du corps et de l'âme de ses précieux compagnons. Ajoutez au choix à vos Possessions l'un des objets suivants :
\begin{description}
\item[La Cape des Crocs] Un patchwork de multiples peaux animales, un rectangle de poils de multiples couleurs. Assez rêche, elle dégage une odeur tenace, rance, agressive. [10]
\item[La Cape des Plumes] Similaire à la précédente, mais composée cette fois de différentes plumes attachés. En l'absence d'espèces tropicales, sa coloration varie simplement du blanc au noir, avec quelques nuances de bleu. Elle est aussi légère et élégante que l'autre est épaisse et grossière. [10]
\end{description}

Quel que soit le vêtement sur lequel votre choix se portera, il vous assurera qu'elle est, selon ses propres paroles, « de réveiller la bête en l'homme quand les circonstances le demandent ».

Notez également le Code \emph{Ombre} et augmentez votre Réputation de 2.

Rendez-vous au \glink{50}.

\gsection{120}

La défiance règne en maître. Rapidement, même les grognements cessent au profit d'un silence lourd de sens et de regards haineux appuyés. Il est nécessaire de fuir l'endroit avant que cela ne tourne au lynchage public.

Rendez-vous au \glink{50}.

\gsection{121}

Discuter avec les gens, aller de porte en porte, de village en village, permet de découvrir beaucoup de choses, surtout si l'on a une vague idée de ce que l'on recherche. Au fur et à mesure de son parcours, votre émissaire va croiser un certain nombre d'indices, aussi bien physiques que des témoignages. À savoir maintenant s'il sera assez malin pour les comprendre et les relier.

S'il dispose de la Spécialité Roublardise et maîtrise en sus la Diplomatie ou la Recherche, rendez-vous au \glink{122}.

Sinon, rendez-vous au \glink{75}.

\gsection{122}

Un peu trop d'alcool à la bonne personne, quelques questions incisives, et le début d'une piste apparaît. Le lendemain matin, l'intéressé se rendra sans doute compte qu'il en a trop dit, et s'il est doué d'un rien de jugeote, fuira vers des terres plus civilisées. Mais pour l'heure, il évoque, vaguement, à demi-mots, que oui, ce n'est pas pour rien que sa maison possède une discrète sortie vers la forêt avec un loquet maniable par une forme quadrupède de petite taille.

Encore une rasade, et il confirme qu'il pourrait en connaître d'autres comme lui, mais qu'il ne dira rien, parce qu'il a juré et tout. Quelques gouttes de plus, et il s'épanche sur ce que Zieux rouges lui ferait s'il avait le malheur de parler. Avant de rouler sous la table, il a tout de même le temps de révéler presque explicitement où habite ce mystérieux personnage et ses compagnons.

Pour éviter des remords potentiellement dangereux, ou le passage d'amis désireux de protéger leurs secrets même en payant le prix fort, comme ce mystérieux Yeux Rouges, la personne que vous avez choisie préfère rentrer au plus vite, dès l'aube et avant le réveil de son hôte, pour vous transmettre ce qu'elle a appris. Vous ne lui donnez pas tord.

Enregistrez l'Information \emph{Là où se cachent les ombres} pour vous et celui ou celle qui l'a découverte. Notez également le Code \emph{Révélation}.

Rendez-vous au \glink{50}.

\gsection{123}

À la faveur de la nuit et d'un court interstice entre deux tours de garde, une silhouette encapuchonnée escalade la muraille, puis se laisse retomber souplement dans la cour intérieure. De là elle se glisse dans les ombres jusqu'à atteindre le bâtiment principal, se coule jusqu'à une poterne excentrée, dont elle crochète la serrure avec dextérité et célérité.

La porte s'ouvre. Et l'enfer se déchaîne.

Dans bien des endroits du monde, en arriver à piéger jusqu'à l'entrée des cuisines avec une multitude d'alarmes, physiques ou magiques, contre les intrusions ordinaires ou surnaturelles, et y adjoindre en sus une poignée de pièges à la dangerosité mortelle, serait sans doute perçu comme le symptôme d'une psychose avancée.

Pour une organisation telle que l'Ordre, c'est un critère de recrutement.

Fosses, filets, collets, molosses déchaînés, au propre comme au figuré, cloches qui sonnent l'alerte, soldats en armes qui affluent dans tous les sens, arbalètes déclenchées par des mécanismes ou des mains humaines, rien ne sera épargné à l'as de la cambriole.

Si vous avez confié cette mission au \hero{Brigand}, rendez-vous au \glink{5}.

Sinon, rendez-vous au \glink{10}.

\gsection{124}

Trouver un lieu de culte en Ylèdre n'est pas si compliqué que cela. Trouver le temple d'une divinité particulière l'est beaucoup plus. En raison du grand nombre de réfugiés qu'elle a accueillis au cours de son histoire, ce pays presque autant de religions différentes que de familles.

Cependant, à cœur vaillant rien d'impossible. Et le moine soldat finit par trouver l'aumônerie qu'il recherche. Une simple grotte dont le fronton a été gravé du signe avec amour à défaut de talent, au fond de laquelle un renflement naturel de la roche a été convertie en autel.

Devant lui est en prières un homme que sa maigreur et sa faible hygiène corporelle désignent comme un ermite. Le nouvel entrant s'agenouille à ses côtés, et ils restent ainsi tous deux à contempler les mystères du divin durant une éternité.

Enfin l'ascète se relève, et s'enquiert des raisons de la présence de son invité. Celui lui explique être à la recherche d'hérétiques, ce à quoi le saint homme s'esclaffe :

« Tout ce pays est hérétique ! Ses habitants vénèrent les puissantes obscures, les bêtes de la lune et les papillons du jour, mais jamais le Vrai Dieu. Mais ce n'est pas pour ces impénitents de bas étage que vous êtes là. Vous cherchez quelque chose de beaucoup plus gros, de beaucoup plus dangereux. Vous cherchez la racine du mal. »

Au milieu de ces paroles hachées, ces aboiements à peine compréhensibles, l'ermite semble peu à peu tomber en transe. Ses yeux deviennent vitreux, la bave lui vient aux lèvres, tandis que sa voix se fait plus grave, prend des accents prophétiques.

« Dans le marais il vit. Dans le marais il complote notre fin à tous. Il est le dieu de l'annihilation, du néant, de la fin. Nos esprits ne sont que des jouets pour lui. Il ne m'a laissé l'entrevoir que pour mieux se nourrir de mes cauchemars. Seul notre Seigneur incarné aurait la force de le vaincre. Mais dans votre orgueil, vous essaierez de l'affronter comme moi j'ai essayé, et vous y laisserez votre âme. Dans le Marais Noir votre mort vous attend. »

Il manque de s'étouffer, au bord de la crise d'épilepsie, mais se calme peu à peu. Épuisé, il met rapidement fin à la conversation, arguant d'en avoir déjà trop dit.

Débloquez la Quête :

\quest{Le marais interdit}{85}{Recherche (\ankh, \caduceus)}{Difficile}{
Les paroles de l'ermite étaient forts énigmatiques par bien des aspects, mais il a explicitement nommé le Marais Noir, une zone de nature hostile désertée par l'homme, comme le repaire du mal. Si la nature exacte de ce qu'il s'y cache vous reste inconnue, ses élucubrations vous font cependant soupçonner que cela doit être passablement dangereux.
}

Rendez-vous au \glink{50}.

\gsection{125}

La partie la plus claire de la prophétie qui vous a été rapportée parle d'un grand livre, et un grand livre vous possédez, même si sa sacralité est sujette à caution.

Vous le fouillez à la recherche de « la clé des souvenirs perdus ». Et vous trouvez effectivement quelque chose qui pourrait correspondre. Sur une des pages jaunies s'étale un puissant sortilège capable de faire remonter à la surface les souvenirs oubliés, même ceux qu'il aurait été préférable de maintenir enfouis.

Vous ignorez dans quel obscur contexte il pourrait servir, mais quand le destin pointe du doigt un atout, bien fou qui l'ignore.

Ajoutez la \emph{Formule du Souvenir} à vos Informations, effacez le Code \emph{Sibylle} et rendez-vous au \glink{50}.

\gsection{126}

L'être respire profondément et se concentre. Pendant quelques instants, il doit cesser d'être tous les humains à la fois pour n'en être plus qu'un. Son visage devient une masse informe alors que des milliers de faces s'y succèdent en un instant. Son corps subit des transformations extrêmes, bouillie de particules en perpétuel réagencement. Il est grand et petit, maigre et gros, homme et femme.

Et après une éternité, le processus s'arrête et il ne reste plus qu'une seule personne, qui s'avance d'un pas sûr dans les couloirs alors que ceux qu'il croise s'écartent de son chemin en le saluant bien bas, avec le respect qui est dû à leur chef. Et les gardes ne font pas de difficulté pour le laisser rentrer dans ses propres appartements, où il va droit au but, car il dispose encore pour un instant des connaissances de son lui actuel.

Rendez-vous au \glink{92}.

\gsection{127}

Ces cavernes sont une anomalie, totalement disjointes de la nature. Les animaux, les plantes les ignorent, car pour eux, c'est comme si elles n'existaient pas. Seule une espèce pensante peut décider de refréner tous ses instincts pour se jeter d'elle-même dans ce trou noir de l'existence.

À l'exact opposé, l'intérieur a lui été conçu comme un piège pour l'homme. Il y est confronté à ses plus grandes peurs derrière le masque de l'obscurité. L'absence de tout danger apparent couplée à la sensation d'un danger permanent est bien plus destructrice pour un esprit logique que n'importe quel monstre.

Mais rien n'a été prévu pour ceux qui naviguent à la lisière entre les deux mondes.

La femme prend une profonde inspiration, laissant l'essence de ses totems couler en elle. La peau de l'ours sur son dos frémit. Non, trop proche de l'homme. Les crocs du tigre autour de son cou s'agitent. Non, les félins sont trop dépendants de leur vue, et vulnérables dans une obscurité aussi parfaite. Les écailles contre son torse tremblent. Pourquoi pas.

Son esprit relève les barrières, et l'humain et le serpent se mélangent. La peau change de teinte, de structure, s'épaissit. Les membres subsistent, mais leur dextérité diminue alors que les vertèbres gagnent en souplesse, que la reptation devient le moyen de transport privilégié. Le toucher est démultiplié tandis que les autres sens, inutiles ici, s'atténuent.

Surtout, l'esprit s'épure, les scories de l'humanité étant remplacées par la froide détermination du patient prédateur. L'hybride se laisse glisser sur la marche suivante, puis encore la suivante, sans se poser plus de questions. Elle sent le sol sous elle, elle sait que l'objectif est en bas, tout ce qu'elle a à faire, c'est donc de descendre calmement jusqu'à l'atteindre.

Rendez-vous au \glink{44}.

\gsection{128}

\success

Les lieux sont en effervescence. Votre émissaire ne tarde pas à comprendre qu'il y a récemment eu un affrontement entre les membres de l'Ordre et les sectateurs de la montagne suite aux événements que vous avez provoqués là-bas. Bien que le duel se soit terminé par la victoire des présents, ils ont essuyé des pertes conséquentes, et doivent encore rattraper certains de leurs ennemis qui se sont enfuis.

Le chaos règne donc en maître avec des entrées et des sorties fréquentes, que ce soient des soldats, des infirmiers, des érudits, des sapeurs... Aucun n'a le temps de s'arrêter pour parler avec un étranger, mais la sécurité est plus lâche avec ce passage constant, et ces individus d'habitude si renfermés, si discrets, si secrets, discutent assez ouvertement dans la frénésie du moment.

Il est donc possible pour un observateur attentif, qui laisserait traîner ses grandes oreilles, de récolter quelques croustillantes informations.

\ellipse

Le rapport complet de tout ce qui a été entendu est au final un peu décevant, car relatant surtout les déboires actuels de l'Ordre. Leurs prises de guerre, desquelles vous avez aussi eu quelques échos, pourraient potentiellement être plus instructives, mais elles sont malheureusement hors de votre portée.
Un détail vous a cependant paru digne d'intérêt et mériterait une enquête.

Débloquez la Quête :

\clearpage

\quest{Le marais interdit}{85}{Recherche (\ankh, \caduceus)}{Difficile}{
Le Marais Noir est un terrain particulièrement inhospitalier, même selon les critères de l'Ylèdre, un bourbier infâme seulement apprécié des moustiques, mais il n'est pas censément plus étrange, plus mystérieux, plus suspicieux qu'un autre endroit du pays. Toutefois, qu'une dizaine de membres de la secte du renouveau, et autant de soldats de l'Ordre à leur recherche, y aient disparu sans laisser de trace vient relativiser cela. Il s'agissait après tout uniquement de personnes entraînées à lutter contre les dangers naturels de ce pays, d'un côté de la barrière ou de l'autre. Qu'aucun n'ait plus donné signe de vie est donc fort inquiétant.
}

Rendez-vous au \glink{50}.

\gsection{129}

La lune brille haut dans le ciel. Ronde et pleine, comme il se doit. C'est une belle nuit pour la chasse. La jeune fille se glisse silencieusement d'ombre en ombre, prédateur à l'affût. Non qu'elle puisse espérer surprendre sa proie sur son propre territoire, mais les habitudes ont la vie dure.

Elle est pieds nus, vêtue seulement d'une tenue légère, basique, une simple tunique à la coupe vieille comme le monde, mais porte à l'épaule un arc absolument superbe. Affichant une courbure délicate mais peu ordinaire, il ne semble pas tant avoir été taillé dans le bois que d'être une unique branche qui aurait naturellement pris la forme d'un arc.

Le carquois qui l'accompagne n'est pas en reste. Les rebords en sont dorés à la feuille, et la représentation aux reflets cuivrés d'une antique déesse pourchassant une biche en fait le tour. Malgré son âge avancé évident, l'objet est encore en excellent état et sa beauté démontre du talent de ses créateurs. Il contient 7 flèches, à la pointe d'argent et à l'empennage en plumes de busard. L'un des projectiles est plus court que les autres, comme si la matière avait commencé à manquer lors de sa création. 

L'ensemble évoque plutôt un équipement rituel, des armes qui sont réservées à un usage sacré, et non destinées à  des choses aussi viles que les guerres des hommes.

La prêtresse est maintenant en vue de l'ennemi, qui transparaît entre les branchages. Elle le contourne, à la recherche d'un angle d'approche préférable. Il n'a aucun point faible apparent, ni même un devant et un derrière. Ce n'est qu'une masse informe, amorphe, sans queue ni tête, palpant l'air de ses excroissances. Elle aimerait ne pas avoir à s'approcher plus, mais le terrain difficile ne lui laisse guère le choix si elle veut pouvoir viser correctement. 

La créature s'excite quelque peu lorsque l'archère se met en position, mais sans plus. Que peut une arme aussi futile contre elle ? Le dérisoire projectile traverse les cieux et s'enfonce mollement dans sa chair sans lui faire le moindre mal. Il n'en reste cependant pas là. Le bois de la tige déborde de son carcan de métal pour éclater en un fouillis de racines qui s'enfoncent toujours profondément, tandis que les plumes sont submergées par un déluge de branches. En une poignée d'instants, c'est un if de bonne taille qui se trouve accroché au flanc de la bête, et il ne cesse de grandir en dévorant son support. 

Le monstre se défend contre l'agression comme un organisme unicellulaire. Sa membrane se déchire et il se désolidarise de la partie malade. Le processus complet ne prend qu'une poignée de secondes, mais un archer entraîné peut tirer une quinzaine de flèches par minute. La corde vibre 5 nouvelles fois, et l'être du fond des âges est noyé sous un déluge de végétation, disparaissant totalement sous le poids d'arbres déjà assez grands pour être plusieurs fois centenaires. 

Et soudain, il explose. Toute son énergie est libérée d'un coup pour le débarrasser des parasites. Les débris volent, racines et cytoplasme entremêlés. Le marais est bombardé d'une pluie de matière vivante encore chaude, et plusieurs des jeunes grands arbres contribuent à la dévastation en s'écrasant avec fracas.

Toutefois, la bête n'est pas morte. Au milieu du cratère ainsi généré, elle existe encore, réduite à sa plus simple expression, un concentré de haine et de malice. Si cette chose a un cœur, cela ne peut être que cette sphère aussi grande qu'un homme qui palpite sur le sol, débarrassé de toutes ces couches protectrices.

Alors l'archère tire son ultime flèche, la tronquée. Après une course aérienne d'un instant seulement, sa pointe s'enfonce profondément dans l'organe vital  avant de le pulvériser de l'intérieur, dans un torrent de verdure. Au lieu d'un véritable arbre, c'est un ensemble de plantes plus modestes, d'herbes, de fleurs, qui poussent sur les restes encore agités de soubresauts de la créature. Et le monstre est bientôt enterré sous un innocent bosquet.

\ellipse

C'est une sensation de froid intense qui réveille la silhouette évanouie. Elle se relève en se frictionnant énergiquement, en claquant des dents. Sa tenue, qui plus est détrempée par l'humidité des lieux, n'est pas adapté à ce climat. Maintenant que la transe est finie, son organisme lui rappelle ce qu'elle lui a fait subir la nuit passée. Elle n'en conserve que de vagues souvenirs, mais le fait qu'elle soit en vie est suffisant pour lui confirmer que la chasseresse céleste a abattu sa cible.

Rendez-vous au \glink{135}.

\gsection{130}

L'histoire se répète. Appâtée par la promesse d'un pouvoir infini, une nouvelle personne vient de s'écarter de sa route à la recherche de la lame de légende. Mais celle-ci a déjà un nouveau gardien, ou plutôt un nouveau jouet.

Il est difficile d'identifier quelqu'un en particulier dans cette silhouette tordue aux yeux vitreux. Les cheveux sont tombés, la peau a été remplacée par des écailles d'acier. Ses formes se sont estompées en faveur d'un élancement généralisé, avec des membres anormalement étirés. Mais quelques signes, que la transformation n'a pas encore pu effacer, rappellent ce que cette chose fut autrefois, parlent d'une époque où elle travaillait pour un vieillard peut-être bienveillant mais surtout inconscient.

Maintenant, ce n'est plus qu'un monstre, taillé pour offrir le plus grand confort à l'épée. Son équilibre est parfait, son allonge extraordinaire. Il lui suffit d'un mouvement gracieux pour projeter la pointe de l'arme droit vers la gorge d'un assaillant avant qu'il ait simplement eu ne serait-ce que le temps de dégainer. C'est un mannequin d'entraînement vivant, la théorie appliquée en mouvement.

Mais la dangerosité extrême de son porteur n'est pas suffisante pour empêcher une autre victime à la tête embrumée de chimère de souhaiter s'en emparer.

Si vous êtes dans un des cas suivants, rendez-vous au \glink{47} :

\begin{itemize}
\item La personne associée au code Folie est également associée au Code Féerie et portait la Couronne des Fées au moment de sa disparition.
\item Il s'agissait du Croisé et vous disposez du Code Mystique.
\item Vous aviez choisi un ou une vétéran, avec à son palmarès trois Quêtes ou plus accomplies avec succès.
\end{itemize}

Sinon, rendez-vous au \glink{16}.

\gsection{131}

\success

« C'est assez délicat, commence-t-il, toute sa répugnance à faire intervenir un extérieur se faisant ressentir dans chaque mot. Nous sommes sur un gros coup, mais il manque encore quelques cordes à notre piège pour nous assurer que notre proie n'en réchappe pas. Malheureusement, j'ai de bonnes raisons de croire qu'elle nous surveille autant que nous la surveillons, et je ne peux pas prendre le risque de l'avertir en envoyant un de mes hommes. J'aimerais donc faire appel à un mercenaire dans cette affaire. Quelqu'un de discret, d'efficace et qui ne puisse pas être directement rattaché à nous. »

\ellipse

Ce n'est pas tout à fait ce que vous espériez, mais c'est mieux que rien. L'Ordre vous laissera accéder  à ses archives si vous accomplissez pour lui une mission d'espionnage contre l'un des groupuscules qu'il combat.

\quest{La secte du renouveau}{56}{Roublardise ou Combat (\ankh, \cross)}{Difficile}{
Une secte hérétique, diabolique, se cache en Ylèdre. L'Ordre a découvert son quartier général, mais ne dispose pas encore d'assez informations pour lancer l'assaut : plan exact des lieux, nombre d'adeptes présents, défenses existantes et échappatoires possibles. À vous d'acquérir ces données pour eux.
}

Notez également le Code \emph{Loi}.

Le \hero{Chasseur} est dorénavant disponible.

Rendez-vous au \glink{50}.

\gsection{132}

La Baba Yaga est surprise. Extrêmement surprise. Si, pour une créature de sa longévité et des ses pouvoirs, elle est restée par bien des aspects fort humaine, ressentant encore un large panel d'émotions, la surprise intense, l'émerveillement, l'étonnement, sont chez elle rares tant elle a tout vu au cours de son existence.

Toutefois, ce n'est pas tous les jours qu'elle peut justement croiser quelqu'un appartenant à la même catégorie d'êtres qu'elle. Ces personnages emblématiques, légendaires, pas assez puissants et intouchables pour être des dieux, mais clairement au-dessus des simples mortels de chair et de sang. Elle doit fouiller loin dans sa mémoire pour se rappeler du nom qui est associé à la masse en perpétuel changement qui affecte la silhouette d'un humain. Mais la Baba Yaga est bien moins gâteuse qu'elle veut bien le faire croire, et cette information remonte des méandres de son passé :

« Dois-tu éternellement payer pour les crimes de ton père, Váli ? »

Les noms ont un pouvoir. Sitôt que ces deux syllabes atteignent les oreilles de celui qui n'est personne, qu'elles remontent le canal auditif et les nerfs jusqu'au cerveau, une métamorphose s'opère. Utilisant ce nom oublié comme un phare dans la mer déchaînée de personnalités, d'existences, qui le composent, il remonte jusqu'à ce qu'il fût, ce qu'il peut être à nouveau.

Ses traits se fixent en un jeune homme à la chevelure rousse et au regard malicieux. Il grandit, en prestance sinon en taille. Les faiblesses de son corps mortel disparaissent derrière l'énergie de son autre nature, et bien sot serait celui qui penserait en le voyant tel qu'il est actuellement qu'il n'est qu'un simple humain.

Spectacle terrifiant, la Baba Yaga sourit de toutes ses dents d'acier, et elle offre avec empressement un siège à celui qui vient de renaître dans sa propre chair.

Ils ont beaucoup à se dire.

Notez le Code \emph{Mille}.

Rendez-vous au \glink{72}.

\gsection{133}

Quelque chose a changé chez l'être que vous avez envoyé chez cette sorcière. Certes, il était déjà fort étrange auparavant, chaotique, incompréhensible. Mais maintenant, vous percevez une volonté nouvelle, une personnalité forte derrière le défilement constant des visages. Pas forcément le mal, mais pas non plus la bonté. Un entre-deux étrange, tout aussi paradoxal que son brouillement passé.

Selon comment vous souhaitez le gérer, cela peut se tourner à votre avantage ou au contraire provoquer des catastrophes.

Exceptionnellement, \hero{Métamorphe} restera disponible après cette mission. Toutefois, notez qu'il aura désormais le coût supplémentaire de diminuer votre Réputation de 1 chaque fois que vous l'emploierez.

Il a également un marché énigmatique à vous proposer. Il vous affirme qu'il peut obtenir une aide supplémentaire précieuse de la Baba Yaga, mais que toute chose a un prix, et que vous n'apprécierez pas forcément celui-ci.

Vous n'arrivez pas à lui arracher plus de détails sur ce qu'il compte entreprendre.

Si vous désirez accepter cette proposition à l'aveuglette, estimant que le jeu en vaut la chandelle, rendez-vous au \glink{19}.

Sinon, rendez-vous au \glink{50}.

\gsection{134}

La vieille femme ne se considère pas comme une grande praticienne de la magie. Elle sait qu'elle n'a qu'une once du pouvoir de certains des êtres qu'elle côtoie. Mais ici au cœur de la nature, elle est sur son territoire. Dans les arbres tout autour d'elle, chez les habitants qui s'y cachent mal, dans son adversaire actuel, circulent le flux de la terre, le mana de la nature. Cette puissance ne lui est pas destinée, mais elle connaît les secrets pour la faire plier.

D'un coup de boutoir mental, elle arrache les fils qui contrôle la marionnette d'argent à leur maîtresse actuelle. Le chevalier fantoche met genou à terre devant elle, puis se relève pour l'escorter tandis qu'elle continue son chemin à la stupéfaction et la terreur des spectateurs.

La réaction des autorités ne tarde évidemment pas, et c'est la reine des lieux qui mène la charge. Elle n'a pas pris la peine de s'auréoler des charmes et des mystères qui la rendent d'habitude si désirable, préférant concentrer son énergie en deux sphères d'énergie crépitantes. Même sous son apparence basique d'humanoïde ailé pas plus grand que le pouce, les deux soleils de mort électrique qui gravitent autour d'elle maintiennent sa superbe. Le reste de ses courtisans et de ses protecteurs qui courent ou volent derrière elle ne sont que des décorations vivantes. Elle seule prend les décisions, elle est le pivot de tout ceci et les autres s'inclinent.

« Mes respects, votre Majesté, déclare votre envoyée, d'une voix contrôlée, sans émotion perceptible. »

« Voilà bien longtemps qu'aucune sorcière n'avait osé s'aventurer en ces lieux, lui répond la reine avec une colère à peine dissimulée. Que me vaut le déplaisir de ta présence ? »

« De vieilles histoires que  ma mère me racontait quand j'étais petite. Des contes de l'âge d'or, avec leur cohorte de nobles chevaliers, défenseurs de la veuve et de l'orphelin. Des jouvenceaux plein d'idéaux et d'amour courtois, dédiant toujours leurs exploits à une gente dame. Et la dame d'entre les dames, celle dont tous se disputaient les faveurs, n'était autre que la grande Titania, reine des fées. »

« Ce sont là les chroniques d'une époque révolue sorcière. Il n'y a plus de héros dignes de porter mes couleurs depuis bien longtemps. »

« C'était effectivement vrai hier. Mais aujourd'hui, le maelstrom du changement est sur nous. De nouveaux héros sont apparus, guidés encore une fois par un destin capricieux. Ils ne portent pas tous l'armure étincelante comme jadis, mais je parierai mon œil droit qu'ils sauront conquérir votre cœur. »

« Et quel est ton rôle dans tout cela, sorcière ? »

« Moi ? Je ne suis que la petite vieille qui leur indique le chemin à prendre quand ils sont perdus. Le personnage que tout le monde oublie, mais sans qui ils n'arriveraient à rien. Car je sais bien plus de choses que ces grands benêts caparaçonnés, mais je suis assez maligne pour ne pas leur dire. »

La fée éclate d'un rire cristallin, plus pur que la neige qui tombe flocon après flocon, et conclue, amusée :

« Certaines choses ne changent jamais. Qu'importe l'époque, nous sommes condamnées à jouer les mêmes rôles dans la même pièce, reprise après reprise, jusqu'à la fin des temps. Et bien soit ! Il est temps pour l'entracte de s'achever et pour le rideau de se lever à nouveau. »

\ellipse

Techniquement, la mission est en échec. Malgré le prix que vous lui avez payé, la jeteuse de sorts n'a pas ramené la moindre information utile. À l'entendre, elle n'a rien trouvé, mais elle insiste étrangement sur le fait qu'un autre aventurier, de préférence un peu charmeur, aurait peut-être plus de chance qu'elle.

Vous vous doutez bien qu'il a anguille sous roche, mais la sorcière est aussi têtue qu'âgée, et vous n'arrivez pas à lui faire avouer la vérité.

Notez que dorénavant, tous les aventuriers, exceptée la \hero{Sorcière} elle-même, sont dorénavant considérés comme disposant de la Spécialité Diplomatie lorsqu'ils accomplissent la Quête \textbf{Le cercle de la paix}. Ce bonus n'est valable que lorsqu'ils effectuent cette mission précise, et uniquement pour la durée de celle-ci.

Inscrivez également le Code \emph{Entente} et augmentez votre Réputation de 1.

Rendez-vous au \glink{50}.

\clearpage

\gsection{135}

Pour la première fois depuis des semaines, votre chambre est bien rangée. Plus de trace de l'impossible capharnaüm qui y régnait encore voilà peu. 

Vos ouvrages de référence, sur l'histoire de cette région, ses mythes et ses légendes, sa langue, sur la morphologie des vampires, loups-garous et démons, la médecine des blessures ordinaires et surnaturelles... Tout cela a été soigneusement empilé et empaqueté, et vous attend dehors, sur le dos d'un solide mulet. 

Vous avez de même trié tout l'équipement hétéroclite accumulé au cours de votre séjour, récupérant ce qui pourrait servir dans une quête future, jetant ou offrant le reste. Ce que vous avez conservé a été calé tant bien que mal au fond de votre sac à dos. 

Vous jetez un œil à la ronde une dernière fois pour vérifier que vous n'avez rien oublié, et votre regard accroche l'omniprésente brume matinale à travers la fenêtre ouverte, laissant à peine deviner les silhouettes délabrées des bâtisses les plus proches. 

La fin de l'influence de la créature des marais n'a pas miraculeusement fait disparaître d'un coup tous les problèmes de la région. La vie y est toujours aussi rude, le climat difficile, les habitants prompts à se renfermer sur eux-mêmes et à réagir à l'inconnu par la violence. Des bêtes défiant les lois de la nature rôdent encore la nuit en quête de chair fraîche, les morts persistent à ne pas le rester, chaque famille a conservé quelques cadavres dans leurs placards. 

Toutefois, avec le temps, peut-être s'avérera-t-il que les habitants dorment mieux, sont moins agressifs, que la nature est plus clémente, le temps moins sombre. Peut-être y aura-t-il une vraie corrélation, peut-être que ne s'agira-t-il que d'une simple coïncidence. 

Mais cela ne dépend plus de vous. Vous avez éliminé la graine du mal plantée en ces terres, rompu la source de pouvoir qui permettait au anciennes horreurs de toujours renaître de leurs cendres, condamnait les choses à ne jamais évoluer. Pour le reste, il va vous falloir faire confiance à la nature humaine, ou parfois inhumaine, à la capacité des habitants à réparer leur contrée affaiblie. 

Vous enfilez votre chapeau et votre manteau, ajustez votre sac, agrippez votre bâton de marche. Vous saluez, et payez, une dernière fois l'aubergiste, puis vous franchissez le seuil de son domaine. Dehors vous attendent le froid et l'humidité, mais le mauvais temps a un parfum de renouveau. 

Votre mission ici est terminée. Votre quête elle est éternelle. Bien d'autres lieux souffrent de problèmes mystiques complexes que seul un expert en votre genre peut espérer démêler. Les dernières rumeurs que vous avez entendu parlent d'une invasion de dragons dans le nord, les nombreuses créatures célestes ayant surgi dont ne sait où pour s'abattre sur les villes et les villages. Un nouveau mystère à résoudre, un nouveau problème à régler. 

Alors, tournant le dos à l'Ylèdre, vous mettez le cap sur le septentrion et la suite de vos aventures.

\theend

\chapter{Succès}

Si vous êtes arrivé jusqu'ici, c'est probablement que vous avez réussi à triompher d'une façon ou d'une autre. Si d'aventure vous désiriez recommencer l'histoire encore une fois, à la recherche de nouvelles possibilités, voici quelques objectifs supplémentaires que vous pouvez vous amuser à essayer d'atteindre.

Ou plus, simplement dit, vous trouverez ci-dessous une liste d'\emph{achievements}, ou Succès en bon français, pour vous aiguiller dans la découverte de tous les secrets de cette histoire.

\begin{description}
\item[Diplomate] Obtenir les Codes \emph{Entente}, \emph{Union} et \emph{Némésis}.
\item[Saint] Terminer avec 12 de Réputation ou plus.
\item[Héros sans nom] Porter le coup de grâce au boss final avec un aventurier anonyme.
\item[Connaissances interdites] Acquérir les deux exemplaires du Tome Scellé, la Formule du Souvenir, et le Code \emph{Yog-Sothoth}.
\item[Vous l'avez fait exprès, n'est-ce pas ?] Obtenir le Code \emph{Paria}.
\item[La violence est la solution] Obtenir le Crâne du Vampire et l'Orbe Transiente, acquérir les Codes \emph{Ante} et \emph{Évolution}.
\item[Radin] Finir l'aventure sans offrir le moindre objet ou la moindre pièce d'or.
\item[Efficace] Finir l'aventure en utilisant 5 aventuriers ou moins en tout.
\item[Fanatisme] Obtenir les Codes \emph{Mystique} et \emph{Purification}, battre le boss final avec le Croisé.
\item[Champion] Triompher de quatre quêtes ou plus avec le même héros.
\item[Combattre le mal par le mal] Acquérir les Codes \emph{Sabbat} et \emph{Tempête} et finir avec une Réputation négative ou nulle.
\end{description}

Pour valider un Succès, vous devez réussir à obéir à ses conditions en une seule tentative. Obtenir un des Codes demandés sur un essai et une Possession lors d'un autre n'est pas forcément compliqué, mais trouver le chemin qui passe par les deux en une seule fois nettement moins facile.

Certains Succès sont mutuellement exclusifs et ne peuvent être obtenus en un seul parcours.

\end{document}

