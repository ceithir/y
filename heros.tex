\documentclass{report}

\usepackage{fontspec}
\usepackage{xltxtra}
\usepackage[french]{babel}
\setmainfont[Mapping=tex-text]{Gentium Book Basic}
\setsansfont{DejaVu Sans}

\usepackage[hidelinks]{hyperref}

\usepackage{xcolor}
\definecolor{light-gray}{gray}{0.8}

\usepackage{array}
\renewcommand{\arraystretch}{2}
\newcolumntype{M}[1]{>{\centering\arraybackslash}m{#1}}

\usepackage{pbox}

%See http://tex.stackexchange.com/questions/109467/footnote-in-tabular-environment#109471
\usepackage{footnote}
\makesavenoteenv{tabular}

\newcommand{\mytextfield}[1]{
    \TextField[bordercolor=,backgroundcolor=,width=#1]{}
}

\newcommand{\myfilledtextfield}[2]{
    \TextField[bordercolor=,backgroundcolor=,width=#1,value=#2,readonly]{}
}

\newcommand{\mycheckbox}{
    \mbox{\CheckBox[width=1.5em,bordercolor=]{}}
}

\newcommand{\mycheckedcheckbox}{
    \mbox{\CheckBox[width=1.5em,bordercolor=,checked,readonly]{}}
}

\newcommand{\mynumberfield}[1]{
    \TextField[bordercolor=,backgroundcolor=,align=1,width=#1]{}
}

\newcommand{\myfillednumberfield}[2]{
    \TextField[bordercolor=,backgroundcolor=,align=1,width=#1,value=#2,readonly]{}
}

\newcommand{\cross}{
    \textsf{☨}
}

\newcommand{\ankh}{
    \textsf{☥}
}

\newcommand{\caduceus}{
    \textsf{☤}
}



\usepackage{fullpage}

\usepackage{framed}

\begin{document}

\tableofcontents

\newcommand{\anonymous}[2]{
    \begin{framed}
        \begin{description}
            \item[Spécialité] #1
            \item[Coût première embauche] #2 pièces d'or
            \item[Coût embauches ultérieures] Non réemployable
            \item[A triomphé de] \begin{Form}\mytextfield{8cm}\end{Form}
        \end{description}

        \vspace{0.25cm}

        \begin{tabular}{ | m{4cm} | m{4cm} | m{4cm} | }
            \hline
            \multicolumn{2}{| m{8cm} |}{Équipement} & Information\\
            \hline
            \mytextfield{4cm} & \mytextfield{4cm} & \mytextfield{4cm}\\
            \hline
        \end{tabular}
    \end{framed}
}

\newcommand{\unlockedhero}{
\begin{tabular}{ | m{4cm} | m{4cm} | m{4cm} | }
    \hline
    Débloqué & Indisponible\\
    \hline
    \mycheckedcheckbox & \mycheckbox \\
    \hline
\end{tabular}
}

\newcommand{\hero}[5]{
    \section{Présentation}

    #1

    \section{Caractéristiques}

    \begin{description}
        \item[Spécialité] #2
        \item[Coût première embauche] #3
        \item[Coût embauches ultérieures] #4
    \end{description}

    \fullbleed{#5}

    \section{Équipement}

    \begin{tabular}{ | m{4cm} | m{4cm} | m{4cm} | }
        \hline
        \multicolumn{2}{| m{8cm} |}{Équipement} & Information\\
        \hline
        \mytextfield{4cm} & \mytextfield{4cm} & \mytextfield{4cm}\\
        \hline
        \mytextfield{4cm} & \mytextfield{4cm} & \mytextfield{4cm}\\
        \hline
        \mytextfield{4cm} & \mytextfield{4cm} & \mytextfield{4cm}\\
        \hline
    \end{tabular}

    \section{Quêtes}

    \begin{tabular}{ | m{7cm} | m{2.5cm} | m{2.5cm} |}
        \hline
        A participé à & En est revenu ? & A triomphé ?\\
        \hline
        \mytextfield{7cm} & \mycheckbox & \mycheckbox \\ 
        \hline
        \mytextfield{7cm} & \mycheckbox & \mycheckbox \\ 
        \hline
        \mytextfield{7cm} & \mycheckbox & \mycheckbox \\ 
        \hline
    \end{tabular}
}

\chapter{Les anonymes}

Les anonymes sont en nombre illimité, mais, pour d'évidentes raisons pratiques, seul un exemplaire de chaque spécialité est ici listé.

\anonymous{Aucune}{2}

\anonymous{Combat}{5}

\clearpage

\anonymous{Diplomatie}{5}

\anonymous{Recherche}{5}

\anonymous{Roublardise}{5}

\chapter{Le Mnémonique}

\unlockedhero

\hero{
L'homme qui se profile dans l'entrée de la taverne ne doit pas avoir plus de trente ans, mais il paraît plus vieux que vous, voire que le monde. Ses cheveux sont blancs et cassants, son regard fou, son corps tremblant. Sa démarche est toutefois assurée, et sitôt que ses yeux vous ont repéré, il s'avance de lui-même vers votre table. S'ensuit une discussion passionnante, où vous comprenez rapidement que l'homme a subi une expérience avec des puissances qu'il ne vaut mieux pas approcher.

Cela l'a laissé marqué, blessé, détruit. Il n'a plus aucun souvenir de sa vie d'avant cette confrontation, mais dispose inversement d'une mémoire parfaite de tout ce qui lui est arrivé depuis (« Je n'oublie plus. Plus rien. Je me souviens de tout. Même l'alcool et les drogues n'arrivent pas à effacer les images de mon esprit. »).

Malgré le mal qu'elle lui fait, la connaissance interdite continue à le fasciner, et il parcourt le monde à la recherche de nouveaux savoirs.
}{
Recherche
}{
Quête de type Recherche : Gratuit / Autre type de Quête : Le Tome Scellé
}{
Comme première embauche
}{images/memory.jpg}

\chapter{L'Homme à la Main d'Acier}

\unlockedhero

\hero{
Les mercenaires patibulaires ne sont pas si rares dans la région, mais celui-ci bat des records. Tout chez lui démontre le vétéran endurci, vivant par la guerre et pour la guerre : nombreuses cicatrices, tonsure réglementaire, démarche militaire, armes en évidence et prêtes à servir. Sa première action à son arrivée est de marcher droit sur le bar et de commander une bouteille de l'alcool le plus fort. Il l'avale cul sec, et vous remarquez alors que s'il a ôté son riche manteau et sa cotte, il a conservé ses gants et une tenue à manches longues qui cachent mal l'hypertrophie de son bras droit.

Ce mystère est résolu plus tard dans la soirée, quand, saoul comme un goret, il vous révèle avoir perdu tout ce qui se trouvait au-delà de son coude droit lors d'une bataille passée. Mais un alchimiste de sa connaissance lui greffa alors un bras d'armure à la place du membre perdu, le gantelet ayant été enchanté pour agir aussi efficacement que s'il avait été de chair. La folle opération réussit, et le guerrier put reprendre ses activités, avec même une efficacité accrue par les avantages de sa transformation (« Ma main de chair peut broyer le cou d'un homme sans mal, mais ma main d'acier peut tordre son armure comme si elle était de papier. »).
}{
Combat
}{
10 pièces d'or
}{
Comme première embauche
}{images/steel.jpg}

\chapter{Le Beau Parleur}

\unlockedhero

\hero{
Aujourd'hui, c'est un aventurier dans beaucoup de sens du terme qui se présente à vous. Beau garçon, cheveux longs, fort amical, il va et vient au gré du vent. Il est là aujourd'hui, et il n'a aucune idée de où il sera demain. Par contre, il aimerait bien être dans le lit de la serveuse cette nuit, et il travaille beaucoup en ce sens.

Vous réussissez à lui toucher quelques mots entre deux flirts, et il s'avère que l'homme est fondamentalement bon, à défaut de particulièrement courageux, et ainsi prêt à vous aider bénévolement pour des tâches de faible dangerosité.
}{
Diplomatie et Roublardise
}{
Difficulté Moyenne ou inférieure : Gratuit / Difficulté supérieure : 5 pièces d'or\footnote{Si vous payez avec un objet, ne l'ajoutez pas à l'Équipement du Beau Parleur, car il sera sous peu converti en courage liquide, et c'est le tavernier qui l'empochera}
}{
Difficulté Moyenne ou inférieure : 5 pièces d'or / Difficulté supérieure : 10 pièces d'or\footnote{Même règle spéciale que précédemment}
}{images/sweet.jpg}

\chapter{Le Brigand}

\unlockedhero

\hero{
Dégoter un homme sans scrupule pour accomplir des tâches pas forcément très légales est facile. En trouver un qui n’essaiera de vous poignarder dans le dos à la première occasion, ou ne s'enfuira pas juste avec votre or, beaucoup moins. Cependant, vous pensez tenir quelque chose avec ce malfrat, sans aucun respect pour la loi mais doté d'un certain code de l'honneur et disposant de vraies compétences.

Voleur d'informations plus que d'objets, espion plus qu'un cambrioleur, il sait lire et comprend plusieurs langues, dont plusieurs formes archaïques encore utilisées dans la noblesse. Maître chanteur à ses heures, vous feriez bien de ne pas lui confier quoi que ce soit qu'il puisse retourner contre vous.
}{
Recherche et Roublardise
}{
Quête sans Combat : 5 pièces d'or / Quête comportant du Combat : 15 pièces d'or
}{
Comme première embauche, mais incrémenté de une pièce supplémentaire à chaque nouvelle embauche (6/16 à la seconde, 7/17 à la troisième etc.)
}{images/rogue.jpg}

\end{document}


