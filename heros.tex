\documentclass{report}

\usepackage{fontspec}
\usepackage{xltxtra}
\usepackage[french]{babel}
\setmainfont[Mapping=tex-text]{Gentium Book Basic}
\setsansfont{DejaVu Sans}

\usepackage[hidelinks]{hyperref}

\usepackage{xcolor}
\definecolor{light-gray}{gray}{0.8}

\usepackage{array}
\renewcommand{\arraystretch}{2}
\newcolumntype{M}[1]{>{\centering\arraybackslash}m{#1}}

\usepackage{pbox}

%See http://tex.stackexchange.com/questions/109467/footnote-in-tabular-environment#109471
\usepackage{footnote}
\makesavenoteenv{tabular}

\newcommand{\mytextfield}[1]{
    \TextField[bordercolor=,backgroundcolor=,width=#1]{}
}

\newcommand{\myfilledtextfield}[2]{
    \TextField[bordercolor=,backgroundcolor=,width=#1,value=#2,readonly]{}
}

\newcommand{\mycheckbox}{
    \mbox{\CheckBox[width=1.5em,bordercolor=]{}}
}

\newcommand{\mycheckedcheckbox}{
    \mbox{\CheckBox[width=1.5em,bordercolor=,checked,readonly]{}}
}

\newcommand{\mynumberfield}[1]{
    \TextField[bordercolor=,backgroundcolor=,align=1,width=#1]{}
}

\newcommand{\myfillednumberfield}[2]{
    \TextField[bordercolor=,backgroundcolor=,align=1,width=#1,value=#2,readonly]{}
}

\newcommand{\cross}{
    \textsf{☨}
}

\newcommand{\ankh}{
    \textsf{☥}
}

\newcommand{\caduceus}{
    \textsf{☤}
}



\usepackage{fullpage}

\usepackage{framed}

\newcommand{\anonymous}[2]{
    \begin{framed}
        \begin{description}
            \item[Spécialité] #1
            \item[Coût première embauche] #2 pièces d'or
            \item[Coût embauches ultérieures] Non réemployable
            \item[A triomphé de] \begin{Form}\mytextfield{8cm}\end{Form}
        \end{description}

        \vspace{0.25cm}

        \begin{tabular}{ | m{4cm} | m{4cm} | m{4cm} | }
            \hline
            \multicolumn{2}{| m{8cm} |}{Équipement} & Information\\
            \hline
            \mytextfield{4cm} & \mytextfield{4cm} & \mytextfield{4cm}\\
            \hline
        \end{tabular}
    \end{framed}
}

\newcommand{\unlockedhero}{
\begin{tabular}{ | m{4cm} | m{4cm} | m{4cm} | }
    \hline
    Débloqué & Indisponible\\
    \hline
    \mycheckedcheckbox & \mycheckbox \\
    \hline
\end{tabular}
}

\newcommand{\hero}[5]{
    \herostats{#1}{#2}{#3}{#4}

    \fullbleed{#5}

    \herosheet
}

\newcommand{\herostats}[4]{
    \section{Présentation}

    #1

    \section{Caractéristiques}

    \begin{description}
        \item[Spécialité] #2
        \item[Coût première embauche] #3
        \item[Coût embauches ultérieures] #4
    \end{description}
}

\newcommand{\herosheet}{
    \section{Équipement}

    \begin{tabular}{ | m{4cm} | m{4cm} | m{4cm} | }
        \hline
        \multicolumn{2}{| m{8cm} |}{Équipement} & Information\\
        \hline
        \mytextfield{4cm} & \mytextfield{4cm} & \mytextfield{4cm}\\
        \hline
        \mytextfield{4cm} & \mytextfield{4cm} & \mytextfield{4cm}\\
        \hline
        \mytextfield{4cm} & \mytextfield{4cm} & \mytextfield{4cm}\\
        \hline
    \end{tabular}

    \section{Quêtes}

    \begin{tabular}{ | m{7cm} | m{2.5cm} | m{2.5cm} |}
        \hline
        A participé à & En est revenu ? & A triomphé ?\\
        \hline
        \mytextfield{7cm} & \mycheckbox & \mycheckbox \\ 
        \hline
        \mytextfield{7cm} & \mycheckbox & \mycheckbox \\ 
        \hline
        \mytextfield{7cm} & \mycheckbox & \mycheckbox \\ 
        \hline
    \end{tabular}
}

\newcommand{\picturelesshero}[4]{
    \herostats{#1}{#2}{#3}{#4}

    \clearpage

    \herosheet
}

\newcommand{\lockedhero}{
\begin{tabular}{ | m{4cm} | m{4cm} | m{4cm} | }
    \hline
    Débloqué & Indisponible\\
    \hline
    \mycheckbox & \mycheckbox \\
    \hline
\end{tabular}
}

\begin{document}

\tableofcontents

\chapter{Les anonymes}

Les anonymes sont en nombre illimité, mais, pour d'évidentes raisons pratiques, seul un exemplaire de chaque spécialité est ici listé.

\anonymous{Aucune}{2}

\anonymous{Combat}{5}

\clearpage

\anonymous{Diplomatie}{5}

\anonymous{Recherche}{5}

\anonymous{Roublardise}{5}

\chapter{Le Mnémonique}

\unlockedhero

\hero{
L'homme qui se profile dans l'entrée de la taverne ne doit pas avoir plus de trente ans, mais il paraît plus vieux que vous, voire que le monde. Ses cheveux sont blancs et cassants, son regard fou, son corps tremblant. Sa démarche est toutefois assurée, et sitôt que ses yeux vous ont repéré, il s'avance de lui-même vers votre table. S'ensuit une discussion passionnante, où vous comprenez rapidement que l'homme a subi une expérience avec des puissances qu'il ne vaut mieux pas approcher.

Cela l'a laissé marqué, blessé, détruit. Il n'a plus aucun souvenir de sa vie d'avant cette confrontation, mais dispose inversement d'une mémoire parfaite de tout ce qui lui est arrivé depuis (« Je n'oublie plus. Plus rien. Je me souviens de tout. Même l'alcool et les drogues n'arrivent pas à effacer les images de mon esprit. »).

Malgré le mal qu'elle lui fait, la connaissance interdite continue à le fasciner, et il parcourt le monde à la recherche de nouveaux savoirs.
}{
Recherche
}{
Quête de type Recherche : Gratuit / Autre type de Quête : Le Tome Scellé
}{
Comme première embauche
}{images/memory.jpg}

\chapter{L'Homme à la Main d'Acier}

\unlockedhero

\hero{
Les mercenaires patibulaires ne sont pas si rares dans la région, mais celui-ci bat des records. Tout chez lui démontre le vétéran endurci, vivant par la guerre et pour la guerre : nombreuses cicatrices, tonsure réglementaire, démarche militaire, armes en évidence et prêtes à servir. Sa première action à son arrivée est de marcher droit sur le bar et de commander une bouteille de l'alcool le plus fort. Il l'avale cul sec, et vous remarquez alors que s'il a ôté son riche manteau et sa cotte, il a conservé ses gants et une tenue à manches longues qui cachent mal l'hypertrophie de son bras droit.

Ce mystère est résolu plus tard dans la soirée, quand, saoul comme un goret, il vous révèle avoir perdu tout ce qui se trouvait au-delà de son coude droit lors d'une bataille passée. Mais un alchimiste de sa connaissance lui greffa alors un bras d'armure à la place du membre perdu, le gantelet ayant été enchanté pour agir aussi efficacement que s'il avait été de chair. La folle opération réussit, et le guerrier put reprendre ses activités, avec même une efficacité accrue par les avantages de sa transformation (« Ma main de chair peut broyer le cou d'un homme sans mal, mais ma main d'acier peut tordre son armure comme si elle était de papier. »).
}{
Combat
}{
10 pièces d'or
}{
Comme première embauche
}{images/steel.jpg}

\chapter{Le Beau Parleur}

\unlockedhero

\hero{
Aujourd'hui, c'est un aventurier dans beaucoup de sens du terme qui se présente à vous. Beau garçon, cheveux longs, fort amical, il va et vient au gré du vent. Il est là aujourd'hui, et il n'a aucune idée de où il sera demain. Par contre, il aimerait bien être dans le lit de la serveuse cette nuit, et il travaille beaucoup en ce sens.

Vous réussissez à lui toucher quelques mots entre deux flirts, et il s'avère que l'homme est fondamentalement bon, à défaut de particulièrement courageux, et ainsi prêt à vous aider bénévolement pour des tâches de faible dangerosité.
}{
Diplomatie et Roublardise
}{
Difficulté Moyenne ou inférieure : Gratuit / Difficulté supérieure : 5 pièces d'or\footnote{Si vous payez avec un objet, ne l'ajoutez pas à l'Équipement du Beau Parleur, car il sera sous peu converti en courage liquide, et c'est le tavernier qui l'empochera}
}{
Difficulté Moyenne ou inférieure : 5 pièces d'or / Difficulté supérieure : 10 pièces d'or\footnote{Même règle spéciale que précédemment}
}{images/sweet.jpg}

\chapter{Le Brigand}

\unlockedhero

\hero{
Dégoter un homme sans scrupule pour accomplir des tâches pas forcément très légales est facile. En trouver un qui n’essaiera de vous poignarder dans le dos à la première occasion, ou ne s'enfuira pas juste avec votre or, beaucoup moins. Cependant, vous pensez tenir quelque chose avec ce malfrat, sans aucun respect pour la loi mais doté d'un certain code de l'honneur et disposant de vraies compétences.

Voleur d'informations plus que d'objets, espion plus qu'un cambrioleur, il sait lire et comprend plusieurs langues, dont plusieurs formes archaïques encore utilisées dans la noblesse. Maître chanteur à ses heures, vous feriez bien de ne pas lui confier quoi que ce soit qu'il puisse retourner contre vous.
}{
Recherche et Roublardise
}{
Quête sans Combat : 5 pièces d'or / Quête comportant du Combat : 15 pièces d'or
}{
Comme première embauche, mais incrémenté de une pièce supplémentaire à chaque nouvelle embauche (6/16 à la seconde, 7/17 à la troisième etc.)
}{images/rogue.jpg}

\chapter{La Sorcière}

\unlockedhero

\hero{
Vous avez un jour parcouru un ouvrage d'un chasseur de sorcières expliquant comment reconnaître une maléfique sorcière. Les critères étaient si vastes et ambigus qu'ils devaient englober à peu près l'intégralité de la population féminine mondiale. Vous avez jeté ce tissu d'âneries pour ne jamais y revenir.

Ce qui ne veut pas dire que vous ne savez pas identifier une jeteuse de sorts puisant ses pouvoirs dans des sources moralement répréhensibles quand vous en croisez une. Bien au contraire. Et la personne assise en face de vous en est clairement une. Oh, en apparence, ce n'est qu'une petite vieille voûtée avançant à petits pas lents, le genre qui ne se remarque pas. Mais si elle cherche à la dissimuler, elle dégage une aura qui ne trompe pas, un mélange d'orgueil naturel et de corruption surnaturelle qui pour un ancien mage comme vous se renifle aussi facilement qu'un parfum capiteux.

Lorsque le tavernier lui sert son repas, il oublie de lui faire payer, mais pas au sens d'un simple étourderie, plutôt qu'au moment de lui demander de sortir son argent, ses yeux sont un instant devenu vitreux et l'idée a soudainement disparu de son esprit, comme arrachée de là par une main invisible. Vous êtes persuadé que ce petit tour peu discret n'était destiné qu'à vous faire une démonstration des capacités que vous allez devoir monnayer.

Elle n'a que faire de l'or, mais est intéressée par tout objet un peu exotique et surtout magique qu'il est bien difficile de se procurer dans cette région reculée.
}{
Recherche
}{
Un objet de coût 3 ou supérieur
}{
Le Fétiche en Os ou tout objet de valeur inestimable ([-])
}{images/witch.jpg}

\chapter{Le Métamorphe}

\unlockedhero

\picturelesshero{
L'esprit humain a un talent naturel pour rejeter les anomalies. C'est probablement pour cela que les autres clients de la taverne ne voit qu'un aventurier parmi d'autres dans l'être assis en face de vous. Mais votre esprit expérimenté remarque lui les minuscules mutations qui se produisent constamment en lui. Les couleurs de ses cheveux, de ses yeux et même de ses vêtements changent en permanence, très légèrement à chaque fois, mais sans interruption. De même sa taille et ses proportions se modifient peu à peu.

Ainsi, vous devez faire des efforts drastiques pour vous souvenir de l'apparence qu'il avait un instant plus tôt. C'est comme si une force extérieure essayait de convaincre votre esprit que la petite brune qui se tient devant vous n'était pas un grand gaillard blond voilà quelques minutes auparavant, mais qu'elle avait au contraire toujours été une rousse flamboyante aux joues tatouées.

Tandis que le colosse à la barbe fleurie commande une pinte à sa mesure qui est lapée par une silhouette aux mains écailleuses dissimulée sous une robe de bure, vous entamez la conversation.

Heureusement, cette personne, à défaut d'un meilleur terme pour la décrire, ne semble pas désireuse d'utiliser son étrange pouvoir pour faire le mal. Bien au contraire, elle accepte de vous aider.
}{
Une au choix parmi Combat, Diplomatie, Recherche et Roublardise\footnote{Choisissez avant de commencer la Quête. Vous pouvez décider la changer de nouveau avant chaque nouvelle Quête.}
}{
Gratuit, mais le Métamorphe devient automatiquement indisponible à la fin de la Quête, quel que soit son résultat.
}{
Comme première embauche
}

\chapter{Le Croisé}

\lockedhero

\hero{
La première chose que vous remarquez chez l'homme est l'imposant symbole religieux s'étalant sur son plastron d'acier. Démesuré, ostentatoire, d'un rouge terni par la poussière du voyage mais qui a dû être éclatant un jour, c'est un message clair à destination du monde. La religion n'est pas un sujet de badinerie chez lui.

Cette particularité mise à part, le reste de son apparence évoque la figure classique du chevalier errant. Une armure complète, de bonne qualité, expurgée de toute fioriture, de tout élément purement décoratif, le recouvre des pieds au cou. L'épée à sa ceinture et son fourreau suivent la même politique d'efficacité dans la simplicité. Il tient son casque d'une main, révélant un visage buriné, dissimulé derrière une barbe poivre et sel soigneusement taillée et surmonté d'un crâne qui a dû être intégralement tondu il y a peu mais sur lequel les cheveux commencent déjà à réapparaître.

Lorsqu'il commande sans ironie aucune une infusion au bar, et non une quelconque boisson alcoolisée, quelques ricanements avinés se font entendre, mais l'homme les ignore avec panache. Il se contente de savourer sa boisson, calmement. Il déguste ensuite son repas dans le même climat de paix intérieure, traitant la maigre chère de l'aubergiste avec un respect qu'elle ne mérite pas, en avalant jusqu'à la dernière miette. Il le conclut pieusement d'une prière murmurée.

Vous avez l'occasion d'échanger avec lui alors qu'il se repose près du feu sans qu'aucun autre client n'ose le déranger. Vous n'avez aucun mal à le convaincre de vous raconter son histoire, et s'ensuit un flot d'aventures, impliquant des voyages dans des pays exotiques et des nombreuses guerres, parfois dans le camp des vainqueurs, parfois dans celui des perdants. Vous le sentez critique, voire dégoûté, honteux sur certains des actes qu'il a pu commettre alors. Le rebondissement attendu est sa révélation divine à l'âge mûr et son entrée dans un ordre de moine combattant.

Pour l'heure, son église l'a envoyé dans ce pays maudit à la recherche de ce qu'il nomme la Larme de Dieu, un artefact sacré de très grand pouvoir, dérobé par un infidèle.

Vous gagnez rapidement sa confiance, et il est prêt à vous accorder son aide. Sa mission est prioritaire sur tout, mais s'il a moyen de la réaliser tout en vous aidant, il le fera avec joie.
}{
Combat
}{
Gratuit pour toute Quête \cross / L'Orbe Transiente pour toute autre Quête
}{
Comme première embauche
}{images/templar.jpg}

\chapter{Le Chasseur}

\lockedhero

\hero{
Cape noire. Bottes noires. Brigandine noire. Chapeau noir. Un éventail complet de lames à la ceinture, du couteau utilitaire à la rapière en passant par toutes les tailles intermédiaires, dont un poignard en argent pur. Une arbalète chargée dans le dos. Un sac rebondi sur l'épaule, débordant d'un attirail disparate, parmi lesquels un piège à loup et ce que vous identifiez comme un attrapeur de rêve.

Telle est l'impressionnante apparence de l'homme qui vient d'entrer. Une telle débauche d'armes pourrait prêter à sourire chez un autre dans sa grossière exagération, mais chez lui, elle ne fait que souligner sa dangerosité. Il a l'air de savoir toutes les manier, et, surtout, il vous semble qu'il n'hésitera pas à toutes les déclencher à la moindre contrariété. La nervosité et la paranoïa se lisent sur son visage alors qu'il observe scrupuleusement chaque recoin de l'auberge, en s'arrangeant pour toujours rester dos au mur.

Vous parvenez tout de même à engager la conservation sans qu'il vous transforme en gruyère. L'homme se présente comme un tueur de monstres, un pourfendeur de démons, un exterminateur de fantômes. Certains de ses combats passés ont laissé leurs marques, aussi bien physiques comme psychiques, et vous comprenez à demi-mot qu'il est persuadé qu'un de ses anciens ennemis a survécu et le pourchasse, raison pour laquelle il se montre aussi circonspect et reste rarement bien longtemps au même endroit.

Néanmoins, ses talents martiaux et son expérience vous seraient d'une grande aide, aussi vous efforcez-vous de le convaincre de faire une halte, même courte, dans sa fuite perpétuelle, pour vous épauler.
}{
Combat et Recherche
}{
10 pièces d'or
}{
La Couronne des Fées, la Cape des Crocs ou le Kriss du Dragon
}{images/hunter.jpg}

\chapter{La Bête}

\lockedhero

\hero{
Le tavernier est un homme pragmatique, qui préfère fermer les yeux sur la nature de ses clients tant qu'ils payent. Mais il est évident qu'il se retient à grande peine de faire une exception pour la femme qui dévore à pleines dents un quartier de viande crue devant lui. Vêtue d'une tenue disparate, mais uniquement tissée dans des peaux de carnivores, du justaucorps en écailles de serpent à la cape en fourrure d'ours, elle arbore également un impressionnant assortiment d'armes d'origine animale, notamment une dent de requin en guise de poignard et une corne de narval comme lance.

Nul besoin d'être grand clerc pour détecter la sauvagerie et la magie bestiale qui se dégagent d'elle. Les autres clients font de leur mieux pour l'ignorer, mais vous êtes cependant obligé d'intervenir pour empêcher qu'une dispute ne dégénère, craignant que tout ceci ne finisse en bain de sang. En récompense, vous êtes pris de vertiges provoqués par votre malédiction, mais cela vous permet d'établir un premier contact.
}{
Combat
}{
Gratuit pour toute Quête \caduceus / Le Kriss du Dragon ou la Cape des Crocs pour toute autre Quête
}{
Gratuit pour toute Quête \caduceus \textbf{de Difficulté Insensée} / Le Kriss du Dragon ou la Cape des Crocs pour toute autre Quête
}{images/beast.jpg}

\end{document}


